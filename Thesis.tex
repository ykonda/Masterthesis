\NeedsTeXFormat{LaTeX2e}

% Rahmenumgebung f�r Studien Diplomarbeiten
% Erstellt von Oomke Weikert und Florian Keiler
% �nderungen im file _changelog.txt

% Titel und Autor der Arbeit unten im
% \hypersetup Kommando �ndern
% ...und nat�rlich auf der Titelseite (title.tex bzw. title_en.tex)

% um pdf-file zu erzeugen:
% compilieren mit:
% pdflatex
% bibtex
% pdflatex
% pdflatex
% (Hinweis: Das pdf-file darf bei Aufruf von pdflatex
% nicht im Adobe Reader ge�ffnet sein. Wenn man das
% pdf-file mit Ghostview �ffnet, muss es nicht geschlossen
% werden, und man kann dort die gerade bearbeitete Seite
% offen lassen)

% Im figures Ordner m�ssen die Bilder z.B. in pdf oder jpg Format liegen,
% mit pdflatex k�nnen *keine* eps Bilder benutzt werden.
% Das pdf file darf beim Kompilieren nicht
% im Acrobat Reader ge�ffnet sein!!!

% um ps-file zu erzeugen:
% compilieren mit:
% latex
% bibtex
% latex
% latex
% dvips

% dvips Aufruf f�r Type-1 Schriften:
% dvips -t a4 -Ppdf
% (alte GhostScript-Version: dvips -t a4 -Ppdf -G0)
% Umwandeln in pdf mit Acrobat Distiller
% oder mit ps2pdf (in GhostScript enthalten)


\documentclass[a4paper, twoside, 11pt]{report}

\newif\ifmakeindex
\makeindextrue % generate index
%\makeindexfalse % don't generate index

\ifmakeindex
% fuer Stichwortverzeichnis
\usepackage{makeidx}

% Stichwortverzeichnis erstellen
\makeindex
\fi


\newif\ifenglish
\englishtrue   % english document
%\englishfalse  % german document

\usepackage[margin=1cm,format=hang,font=small,labelfont=bf,textfont=sl]{caption}
% angepasste Bildunterschriften
% Doku siehe beigef�gtes pdf-file

\usepackage{subfigure}

\usepackage[english]{babel}

\usepackage[latin1]{inputenc}
% Unterstuetzen von deutschen Umlauten

\usepackage{t1enc}
% Verwenden von DC-Fonts (erlaubt richtiges Trennen
% auch in Worten mit Umlauten)

%\usepackage{times}
  % Saves a lot of ouptut space in PDF...
  % ..after conversion with ps2pdf or Adobe Distiller
	% �ndert aber die Schriftart! (Times statt computer modern roman)
\usepackage{courier} % fett *und* typewriter nur damit m�glich?
% �ndert normale Schrift nicht!

\usepackage{cite}
% Automatisches Zusammenfassen von Literaturstellen

\usepackage{config/fancyheadings}
%STY-file fancyheadings.sty nicht standardm��ig in miktex enthalten
%evtl. durch fancyhdr ersetzen

\usepackage{float}
%This package improves the interface for defining floating objects such
%as figures and tables in LaTeX.  It adds the notion of a `float style'
%that governs appearance of floats.  New kinds of floats may be defined
%using a \newfloat command analogous to \newtheorem.  This style option
%also incorporates the functionality of David Carlisle's style option
%`here', giving floating environments a [H] option which means `PUT IT
%HERE' (as opposed to the standard [h] option which means `You may put
%it here if you like').

\usepackage{verbatim}
%This package reimplements the L A T E X verbatim and verbatim* envi-
%ronments. In addition it provides a comment environment that skips any
%commands or text between \begin{comment} and the next \end{comment}.
%It also defines the command verbatiminput to input a whole file verbatim.

\usepackage{amsmath}

%\usepackage{rotating}

\usepackage{array}
% wird benutzt von macro.tex f�r \Case

\usepackage{ifthen}
% wird benutzt von macro.tex \ifthenelse

\usepackage{longtable}
% f�r lange Tabellen gr��er als eine Seite mit \begin{longtable} .. \end{longtable}

\usepackage{varioref}
% Kommando \vref verweist mit Seitenzahl

\usepackage{readfile}
% ASCII-File einbinden mit
% \readit{fftdb.m}{\tt}
% 2. Argument (\tt) gibt zu benutzende Schriftart an
\usepackage{shapepar}
\usepackage{setspace}
\usepackage{colortbl}
\usepackage{color}
\usepackage{xcolor}

\usepackage{listings}
% Einfache Handhabung von Abkürzungen
\usepackage{acronym}

% Einheiten, Units
\usepackage{siunitx}

% Floats (images, tables, etc.) will be put within a section
% and following sections will start after them.
\usepackage[section]{placeins}

\lstloadlanguages{}

\lstloadlanguages{C,C++,Java,Matlab,HTML,TeX,XML}


%\lstset{}% restore default
%# title={Titel}:  gibt einen Titel zur Umgebung an (erscheint �ber dem Code zentriert)
%# caption={Titel}:
%# label=Name:

\lstset{
				frame=lines, %Umrandung (single|none|shadowbox|lines|bottomline|topline|leftline)
				framerule=1pt, %Rahmenbreite
				tabsize=4, %Anzahl der Zeichen f�r ein TAB
				backgroundcolor=\color{lightgray}, %Hintergrundfarbe
				emph={}, %hebt die angegebenen W�rter hervor
				emphstyle=\underbar, %unterstreicht hervorgehobene W�rter
				columns=fixed, %Zeichenabstand (fixed | flexible | fullflexible)
				lineskip=0pt, %Zeilenabstand
				basicstyle=\ttfamily\small,%\ttfamily,
				identifierstyle=\color{black},
        commentstyle=\color{darkgreen},
        stringstyle=\color{viola},
        keywordstyle=\color{darkblue},
				ndkeywordstyle=\color{black},
				showspaces=false,
				showtabs=false,
				numbers=none, %Zeilennummern (none|left|right)
				%numbertype=\ttfamily,
				breaklines=true,
        captionpos=b,
        extendedchars=false
}

%\usepackage{fancybox}
%\usepackage{theorem}
%\usepackage{amsbsy}
%\usepackage{nomencl}        % unterst�tzt Symbolverzeichnisse
%\usepackage{makeidx}
%\usepackage{multind}
\usepackage{amssymb}

\def\boxes{yes}
% \boxes == yes makes boxes around desired formulas by \Mbox
% wird benutzt von macros.tex


% pdf-tex settings:
% ------------------------
% detect automatically if run by latex or pdflatex


\usepackage{ifpdf}
%\newif\ifpdf
%\ifx\pdfoutput\undefined
%   \pdffalse
%\else
%   \pdfoutput=1
%   \pdftrue
%\fi

\ifpdf % compiling with pdflatex
   \usepackage[pdftex]{graphicx}
   \DeclareGraphicsExtensions{.pdf, .png, .jpg}
   \usepackage[pdftex,
	bookmarks,
	%colorlinks=false, % instead of colors, now boxes are used for links
	colorlinks=true,
	urlcolor=black, %blue,
	linkcolor=black, %red, %normal internal links
	citecolor=black, %green, %citation links
	%pagebackref, %link from references back to page of citation
	linktocpage,
	% im Inhaltsverzeichnis Link auf Seitenzahl (sonst Probleme bei langen Zeilen)
	%breaklinks = true, % f�r Links l�nger als 1 Zeile
	%hypertexnames = false,  % f�r Links zu Figures?
	bookmarksopen, %open all bookmark folders
	bookmarksnumbered, %use section numbers with bookmarks
    pdfpagemode=UseOutlines, %show bookmarks
	% http://www.tex.ac.uk/cgi-bin/texfaq2html?label=pdfpagelabels
	plainpages=false, % eigene Seitenanker f�r r�mische/arabische Seitenzahlen
	pdfpagelabels, % im Abode Reader Seitenzahl als z.B. "iii (3 von 20)" anzeigen
%	pdfstartview=FitH
	pdfstartview=FitV
	]{hyperref}
    \pdfadjustspacing=1                %%% force LaTeX-like character spacing
    \pdfcompresslevel=9
	\pdfcatalog{
	% Catalog dictionary of PDF output.
    % /PageMode /UseNone
    % /URI (http://www.fi.muni.cz/)
	%
	% pdfscreen-like setting might look like:
	%     /PageMode /none
	%     /ViewerPreferences <<
	%         /HideToolbar true
	%         /HideMenubar true
	%         /HideWindowUI true
	%         /FitWindow true
	%         /CenterWindow true
	%
	% /PageMode determines how Acrobat displays the document on startup. 
	% The possibilities for the latter are explained below:
	% Supported /PageMode values.
	% /UseNone 		neither outline nor thumbnails visible
	% /UseOutlines 	outline visible
	% /UseThumbs 	thumbnails visible
	% /FullScreen 	full--screen mode
	% In full--screen mode, there is no menu bar, window controls,
	% nor any other window present. The default setting is /UseNone.
	} %end of \pdfcatalog
\else % compiling with latex
  \usepackage[dvips]{graphicx}
%  \usepackage{color}
  \DeclareGraphicsExtensions{.eps}
  \usepackage[
    dvips,
%	ps2pdf,  % statt dvips, Unterschied???
	bookmarks,
  	colorlinks=false,  % for final paper without colors
%  	colorlinks=true,
	linktocpage,
	% im Inhaltsverzeichnis Link auf Seitenzahl (sonst Probleme bei langen Zeilen)
	%breaklinks = true, % f�r Links l�nger als 1 Zeile
	% funktioniert *nicht* korrekt f�r lange Zeilen mit gs < 7.05.3 ?
%	pdfstartview=FitH, % funktioniert NICHT mit Adobe Distiller
	bookmarksopen, %open all bookmark folders
	bookmarksnumbered, %use section numbers with bookmarks
    pdfpagemode=UseOutlines, %show bookmarks
	pdfstartview=FitV
   ]{hyperref}
  % hyperrefs are active is the pdf file after conversion
  \hypersetup{
	pdfcreator  = {LaTeX with hyperref package},
	pdfproducer = {dvips + ps2pdf}
  }
\fi

\usepackage{cleveref}
% \renewcommand\lstlistingname{Sourcecode}
% \crefname{listing}{sourcecode}{sourcecodes}

\usepackage{algorithm}
\usepackage[noend]{algpseudocode}
\usepackage{adjustbox}
\makeatletter
\let\OldStatex\Statex
\renewcommand{\Statex}[1][3]{%
  \setlength\@tempdima{\algorithmicindent}%
  \OldStatex\hskip\dimexpr#1\@tempdima\relax}
\makeatother

% Herausgehobene Todos
\usepackage[colorinlistoftodos, prependcaption, textsize=small]{todonotes}

\usepackage[paperwidth=275.9mm, paperheight=289.4mm]{geometry}

% CHANGE TITLE AND AUTHOR !!!
\hypersetup{
  pdftitle={Whistle Sound Source Localization using Multiple NAO Robotic Systems},
	pdfauthor={Yuria Konda},
	pdfsubject  = {Whistle Sound Source Localization using Multiple NAO Robotic Systems, UniBwH, Professur ANT, \today},
	pdfkeywords = {}
}

% ------------------------

%Breite der Bild/Tabellenbeschriftungen:
%\setlength{\LTcapwidth}{\captionwidth}

%Schriftgr��e/Zeilenvorschub in Tabellen:
\def\tablefontsize{\normalsize}
\def\tablespacing{1.0} % linespacing in tables

\newcommand{\bc}{\begin{center}}
\newcommand{\ec}{\end{center}}

\newcommand{\be}{\begin{equation}}
\newcommand{\ee}{\end{equation}}
\newcommand{\bea}{\begin{eqnarray}}
\newcommand{\eea}{\end{eqnarray}}

\newcommand{\bi}{\begin{itemize}}
\newcommand{\ei}{\end{itemize}}

\newcommand{\bmp}{\begin{minipage}[c]{\columnwidth} \bc}
% argument: vertical position: c, t, b
\newcommand{\emp}{\ec \end{minipage} }

\newcommand{\beg}{\hspace{0.5cm}\begin{minipage}{14cm}}
\newcommand{\eeg}{\end{minipage}\vspace{1cm}}


%%%%%%%%%%%%%% tables %%%%%%%%%%%%%%%%%%%%%%%
\newcommand{\bt}[1]
% #1: table placing: htbp
    {    \begin{table}[#1] %htbp
         \begin{center}
         \renewcommand{\baselinestretch}{\tablespacing}
         \tablefontsize
    }

%Tabelle mit gr��erem Zeilenvorschub:
\newcommand{\btline}[2]
% #1: table placing: htbp
% #2: linespacing in table, e.g. 1.4
{   \def\tablespacing{#2}
    \bt{#1}
}

\newcommand {\et}[2]
% #1: caption
% #2: label without 'tab:'
    {
     \caption{#1.}
     \label{tab:#2}
     \end{center}
     \end{table}
     \def\tablespacing{1.0}
    % reset linespacing for next table
}

\newcommand {\btab}{\begin{tabular}}
\newcommand {\etab} {\end{tabular}}

\newcommand{\mc}{\multicolumn}
\newcommand{\mr}{\multirow}

% columns in math mode:
\newcolumntype{C}{>{$}c<{$}}
\newcolumntype{L}{>{$}l<{$}}
\newcolumntype{R}{>{$}r<{$}}

\newcommand{\pbs}[1]{\let\temp=\\#1\let\\=\temp}
%preserve backslash, s. companion, p108



%%%%%%%%%%%%%% figures %%%%%%%%%%%%%%%%%%%%%%%
\newcommand{\fig}[4]{
% Bild mit 80% der Seitenbreite
% arguments:
% #1: file without extension eps
% #2: caption
% #3: label
% #4: placing of the figure: e.g. htbp
\begin{figure}[#4]
\begin{center}
   \includegraphics[width=.8\columnwidth]{figures/#1}
   \caption{#2.}
   \label{fig:#3}
\end{center}
\end{figure}
}

\newcommand{\figscale}[5]{
% arguments:
% #1: file without extension eps
% #2: caption
% #3: label
% #4: placing of the figure: e.g. htbp
% #5: width of figure
\begin{figure}[#4]
\begin{center}
   \includegraphics[width=#5\columnwidth]{figures/#1}
   \caption{#2.}
   \label{fig:#3}
\end{center}
\end{figure}
}

\newcommand{\figscaletwo}[6]{
% arguments:
% #1: file1 without extension eps
% #2: file2 without extension eps
% #3: caption
% #4: label
% #5: placing of the figure: e.g. htbp
% #6: width of figure
\begin{figure}[#5]			      
\begin{center}
   \includegraphics[width=#6\columnwidth]{figures/#1}~\\
   \includegraphics[width=#6\columnwidth]{figures/#2}
   \caption{#3.}
   \label{fig:#4}
\end{center}
\end{figure}
}

\newcommand{\figtwo}[5]{
% two figures labeled (a) and (b) side by side
% arguments:	
% #1: file1 without extension eps
% #2: file2 without extension eps
% #3: caption without fullstop
% #4: label
% #5: placing of the figure: e.g. htbp
\begin{figure}[#5]			      
\begin{center}
    \btab{cc} 
   \includegraphics[width=0.45\columnwidth]{figures/#1}
    &	 
   \includegraphics[width=0.45\columnwidth]{figures/#2}
    (a)&(b)
    \etab 
	\caption{#3.}
	\label{fig:#4}
\end{center}
\end{figure}
}

\newcommand{\figTwo}[6]{
% two figures labeled (a) and (b) on top/bottom
% arguments:	
% #1: file1 without extension eps
% #2: file2 without extension eps
% #3: caption without fullstop
% #4: label
% #5: placing of the figure: e.g. htbp
% #6: width of figures as ratio, e.g. 0.7
\begin{figure}[#5]			      
\begin{center}
   \includegraphics[width=#6\columnwidth]{figures/#1}
(a)\\
\vspace{1 em}
   \includegraphics[width=#6\columnwidth]{figures/#2}
    (b)\\
    \caption{#3.}
    \label{fig:#4}
\end{center}
\end{figure}
}

%%%%%%%%%%%%%%%%%%%%%%%%%%%%%%%%%%%%%%%%%%%%%%

\newcommand{\C}[1]{\texttt{#1}} %Typewriter-Font
\newcommand{\tild}{\~~\hspace{-1 ex}} %Tilde ~

\newcommand{\equi}{\Leftrightarrow} %�quivalenz <=>
\newcommand{\concl}{\Rightarrow}    %daraus folgt =>
%Matrix:
\newcommand{\matr}[2]{\left ( \begin{array}{#1} #2 \end{array} \right )}
% #1 alignment of colums, e.g. ccc
% #2 contents of matrix
%Vektor:
\newcommand{\vect}[1]{\left ( \begin{array}{c} #1 \end{array} \right )}
%Fettschrift im Mathemodus:
\newcommand{\mbf}[1]{\mathbf{#1}} %boldface
\newcommand{\gbf}[1]{\boldsymbol{#1}} %boldface greek letters
%Verweis auf Gleichungen mit Klammern um Gl.-Nr.
%\newcommand{\eqref}[1]{(\ref{#1})}
% \eqref defined in package amsmath


\newcommand{\freqorig}{\ensuremath{\circ\hspace{-.1em}\mbox{---}\!\bullet}}
\newcommand{\freq}{\mbox{~}{\circ\hspace{-.1em}\mbox{---}\!\bullet}\mbox{~}}
	% correspondence symbol time<->freq 
\newcommand{\freqswap}{\ensuremath{\bullet\!\mbox{---}\hspace{-.1em}\circ}}
\newcommand{\freqv}{\begin{turn}{90} \freqswap \end{turn}}
	% vertical correspondence symbol
	% time 
	% freq
\newcommand{\timeh}{\mbox{~}{\bullet\!\mbox{---}\hspace{-.1em}\circ}\mbox{~}}
	% correspondence symbol freq<->time
\newcommand{\timev}{\begin{turn}{90} \freqorig \end{turn}}
	% vertical correspondence symbol
	% freq
	% time

\newcommand{\Case}[1]{\left\{ \begin{array}{ll} #1
    \end{array}\right. } 


\newcommand{\Mbox}[1]{
\ifthenelse{\equal{\boxes}{yes}}
{%\ensuremath
\fbox{$\displaystyle #1$}}
{#1}}

%box around a equationarray
%argument is the contents of the eqnarray environment
\newcommand{\Mboxarray}[1]{
\ifthenelse{\equal{\boxes}{yes}}
{\bc\fbox{\begin{Beqnarray} #1 \end{Beqnarray}}\ec}
{\bea #1 \eea}
}


\newcommand{\suml}{\sum\limits}
% limits of sum UNDER the sign instead of beside it, e.g. in fractions

\newcommand{\intl}{\int\limits}
% limits of int UNDER the sign instead of beside it


\newcommand{\logd}{\mbox{ld\,}} %base 2 logarithm

\newcommand{\erw}[1]{E\left\{#1\right\}} %Erwartungswert


% make ',' an ordinary Symbol in decimal numbers
% Unterdr�ckung des Zwischenraums hinter einem Komma (z.B. in Dezimalzahlen)
% Steht im Mathemodus hinter dem Komma ein Leerzeichen, wird es als Trennzeichen
% (mit Zwischenraum) benutzt, sonst als Dezimalkomma.
 \mathchardef\CommaOrdinary="013B
 \mathchardef\CommaPunct   ="613B
 \mathcode`,="8000   % , im Math-Mode aktiv ("8000) machen
 {\catcode`\,=\active
  \gdef ,{\obeyspaces\futurelet\next\CommaCheck}}
 \def\CommaCheck{\if\space\next\CommaPunct\else\CommaOrdinary\fi}

% define the size of the region where text can appear on the page
\setlength{\hoffset}{-1in}
\textwidth 15.7cm
\textheight 24cm %24.4cm
\topmargin -1cm %-1.6cm
\oddsidemargin 2.7cm
\evensidemargin 2.6cm
%\oddsidemargin -3cm
%\evensidemargin -3cm
\headheight 15pt%[12]
\headsep 1.2cm%[25pt]
%\setlength{\footheight}{15pt} %[12]
%\columnsep 0.8cm

% setzt den Einzug bei neuem Absatz zu null
\setlength{\parindent}{0pt}
% oder \parindent0cm

%Abstand vor neuem Absatz
%wird in abstract.tex benutzt
\def\parspacing{2.5ex}
\setlength{\parskip}{\parspacing} 

% enable following command if you don't want extra space after points
\frenchspacing

\renewcommand{\textfraction}{0.1}
% Mindestanteil an Text auf einer Seite, 
% in Prozent, default: 0.2
 
\renewcommand{\topfraction}{1}
% Maximaler Anteil der von Gleitobjekten am Kopf der Seite 
% eingenommen werden kann, in Prozent, default: 0.7

\setcounter{topnumber}{10} 
% number of float objects at top of page, default: 2

\setcounter{totalnumber}{10} 
% number of float objects on one page, default: 3


\setcounter{secnumdepth}{3}
% depth of numbering, see companion p.20-21
% for report:   chapter: 0, section: 1, subsection: 2
%       subsubsection:3,paragraph: 4,   subparagraph: 5

\setcounter{tocdepth}{2}

\setlongtables
% for equal column width on different pages of a longtable, see companion, p 122
% has to be disabled to reset, only widest width is saved, 
%so there's an error, if this entry is shortened!


\addto\captionsenglish{%
	\renewcommand{\lstlistlistingname}{List of Listings} % default: "Listings"
	\renewcommand{\lstlistingname}{Listing} % default: "Listing"
}


%\makeglossary

\begin{document}
\definecolor{gold}{rgb}{0.85,.66,0}
\definecolor{viola}{rgb}{0.5,0,0.5}
\definecolor{darkblue}{rgb}{0,0,.6}
\definecolor{darkred}{rgb}{.6,0,0}
\definecolor{darkgreen}{rgb}{0,.6,0}
\definecolor{red}{rgb}{.98,0,0}
\definecolor{lightgray}{rgb}{.99,.99,.97}


%HSU Corporate Design
\definecolor{hsurot}{cmyk}{0,1.00,0.43,0.185}
\definecolor{hsugrau}{cmyk}{0,0.085,0.15,0.43}
\definecolor{hsugelb}{cmyk}{0,0.15,0.60,0}
\definecolor{hsublau}{cmyk}{1.00,0.38,0,0.69}
\definecolor{hsugruen}{cmyk}{1.00,0,0.47,0.47}
\definecolor{hsuoliv}{cmyk}{0,0,1.00,0.47}
\definecolor{hsuorange}{cmyk}{0,0.43,1.00,0.18}
\definecolor{hsuhellrot}{cmyk}{0,0.76,0.83,0.11}

 

% 60%
\colorlet{hsurot60}{hsurot!60}
\colorlet{hsugrau60}{hsugrau!60}
\colorlet{hsugelb60}{hsugelb!60}
\colorlet{hsublau60}{hsublau!60}
\colorlet{hsugruen60}{hsugruen!60}
\colorlet{hsuoliv60}{hsuoliv!60}
\colorlet{hsuorange60}{hsuorange!60}
\colorlet{hsuhellrot60}{hsuhellrot!60}

% 30%
\colorlet{hsurot30}{hsurot!30}
\colorlet{hsugrau30}{hsugrau!30}
\colorlet{hsugelb30}{hsugelb!30}
\colorlet{hsublau30}{hsublau!30}
\colorlet{hsugruen30}{hsugruen!30}
\colorlet{hsuoliv30}{hsuoliv!30}
\colorlet{hsuorange30}{hsuorange!30}
\colorlet{hsuhellrot30}{hsuhellrot!30} % several predefined colors - including the HSU corporate Design.

\pagenumbering{roman} % sonst gibt es 2 mal Seiten 1 und 2 (arabisch)
% so werden Probleme bei hyperlinks vermieden

%\def\voff{1cm}
%\addtolength{\voffset}{-\voff}

%\vspace{-2cm}
\pdfbookmark[1]{Title page}{sec:title}  % Bookmark im pdf file
\begin{titlepage}
    \label{sec:title}

    %\enlargethispage{\voff}
    \large
    \begin{center}

        % Unilogo mit Link auf Webseite
        \href{http://www.unibw-hamburg.de}
        {%\includegraphics[width = 9cm]{figures/Uni_farbe}
            \includegraphics[width = 9cm]{figures/hsu/HSU-logo_farbe}
        }

        \vspace{2cm}

        \begin{minipage}{.9\linewidth}
            {\centering\Huge\bf Whistle Sound Source Localization Using Multiple NAO Robotic Systems\par}
            % \par ist n�tig f�r korrekten Zeilenabstand im Titel
        \end{minipage}

        \large

        %\vspace{1.5cm}
        %{von\\[.3cm] {\bf \LARGE Autor}\\[1cm]
        %\huge	{ \bf \it Diplomarbeit} \\[1cm]}

        \vspace{1.5cm}
        {\huge{\bf \it Master Thesis} %\\[1cm]
        }

        \vspace{0.6cm}
        {
        \Large by \\[.5cm]
        {\bf \LARGE \huge Yuria Konda}\\[1.5cm]
        }

        \vfill
        \begin{tabular}{ll}
            Start date:            & 04. Max 2019                       \\
            End date:              & 04. November 2019                   \\
            Supervisor:            & Dr.-Ing. Martin Holters            \\
            Supervising Professor: & Prof.~Dr.-Ing.~habil.~Udo Z\"olzer \\
        \end{tabular}
    \end{center}
    \vspace{2cm}

\end{titlepage}

%\addtolength{\voffset}{\voff}

\thispagestyle{empty} % Titelr�ckseite (S. 2) ohne Seitennummer

\cleardoublepage

\newcounter{pageno}
\setcounter{pageno}{1} %titlepage = page 1, but pagenumber not printed
\addtocounter{pageno}{1}

\cleardoublepage
\pdfbookmark[1]{\abstractname}{sec:abstract}  % Bookmark im pdf file
\begin{abstract}
\label{sec:abstract}
\addtocounter{pageno}{1}
\setcounter{page}{\arabic{pageno}}
\thispagestyle{plain}
\begin{center}
\begin{minipage}[t]{0.8\linewidth}
\setlength{\parskip}{\parspacing}

% Sound source localization is a ongoing topic in research that receives
The RoboCup Standard Platform League is a competition for prospective
researchers to compete in autonomous robot soccer
with the overall goal to contribute to research in the fields
of humanoid robotics and autonomous multi-agent systems.
According to the rules of this league, implementation is done on NAO robots.
%  which
% come with four microphones attached on the head.
Currently, audio signals are only used as indicator for the kickoff
in form of a whistle sound.
To prevent the detection of false positives from neighboring fields,
a whistle sound source localization is designed and implemented.
%  which is based on a \acf{TDOA} approach.
% This work evaluates the realizability of a sound source localization according
% to these whistles with an approach based on the \acf{TDOA}.
% In order to obtain the time delay between the microphones,
% three methods were investigated
Different methods that are based on the \acf{TDOA} are evaluated
to obtain the direction of the whistle source using the four
microphones attached on the robot's head.
To compute a global position of the acoustic source,
direction estimates of multiple robots are fed into a multi-agent
filter.
The resulting algorithm is shown to allow whistles to be
localized with an \ac{RMSE} of 1\si{\meter} in terms of
Euclidean distance.




\end{minipage}
\end{center}

\end{abstract}



\cleardoublepage
% \begin{center}
% \pdfbookmark[1]{Statement}{sec:statement}  % Bookmark im pdf file
% {\Huge \bf Statement}
% \label{sec:statement}
% \end{center}
\begin{minipage}[t]{0.8\linewidth}
% \setlength{\parskip}{\parspacing}
\vspace{4cm}

\textit{I want to thank the members of the HULKs for the infinite support.}
\todo[inline]{Formulation is weird. Maybe rather: "To the team members of the HULKs, in special gratitude for their tremendous support."}

\end{minipage}


\cleardoublepage
\begin{center}
\pdfbookmark[1]{Statement}{sec:statement}  % Bookmark im pdf file
{\Huge \bf Statement}
\label{sec:statement}
\end{center}
Hereby I do state that this work has been prepared by myself and with the help which is referred within
this thesis.

%Hiermit erkl�re ich, dass die vorliegende Arbeit von mir selbst�ndig und nur unter Verwendung der angegebenen Quellen und Hilfsmittel erstellt wurde.

\vspace{2cm}

Hamburg, November 4th 2019


\cleardoublepage
\pdfbookmark[1]{Foreword}{sec:vorwort}  % Bookmark im pdf file
\chapter*{Foreword}
\label{sec:vorwort}

\todo[inline]{TODO -- Missing}

\vspace{2cm}
Hamburg, xxx


\cleardoublepage
\pagestyle{fancyplain}

% Font of the header
\def\headfont{\bfseries}%\sffamily}

% Gew�nschte Formatierung der Kopfzeile, default: alles in Gro�buchstaben
\renewcommand{\chaptermark}[1]
{\markboth{\chaptername { }\thechapter. #1} 
{\chaptername { }\thechapter. #1}} 
% p 99
% without name ``Chapter'', ``Appendix'' etc:
%\renewcommand{\chaptermark}[1]{\markboth{\thechapter. #1}{\thechapter. #1}} % p 99


% for twoside option:
\renewcommand{\chaptermark}[1]{\markboth{\chaptername { }\thechapter. #1}{}}
%\chaptername { }\thechapter. #1}}
\lhead[\fancyplain{\headfont\thepage}{\headfont\thepage}]
{\fancyplain{\headfont\rightmark}{\headfont\rightmark}}

\renewcommand{\sectionmark}[1]{\markright{\thesection\ #1}}

%\lhead{\fancyplain{\headfont \rightmark}{\headfont\rightmark}}
\rhead[\fancyplain{\headfont\leftmark}{\headfont\leftmark}]
{\fancyplain{\headfont\thepage}{\headfont\thepage}}
%\lfoot{\tt Arbeit \emph{Titel} -- Autor}
%\rfoot{\tt Stand \today} %aktuelles Datum
\cfoot{}
%\setlength{\plainfootrulewidth}{0pt}
%\renewcommand{\footrulewidth}{0pt}
\setlength{\plainheadrulewidth}{0.4pt}
 % set fancy headings
%$$$$$$$$$$$$$$$ tableofcontents $$$$$$$$$$$$$$$$$
\ifenglish
	\newcommand{\chapname}{Chapter}
\else
	\newcommand{\chapname}{Kapitel}
\fi % end of \ifenglish\renewcommand{\chaptername}{} %if name appears in header

\setlength{\parskip}{0ex}
\renewcommand{\baselinestretch}{1}
\normalsize

\cleardoublepage % to get correct page no. in TOC
\pdfbookmark[1]{\contentsname}{toc}  % Bookmark auf Inhaltsverzeichnis im pdf file
\tableofcontents


\setlength{\parskip}{1ex}
\cleardoublepage % to get correct page no. in TOC
\phantomsection
\setcounter{lofdepth}{2}
% subfigures werden mit aufgelistet in der LOF
\addcontentsline{toc}{chapter}{\protect\numberline{\listfigurename}}
\listoffigures

\cleardoublepage % to get correct page no. in TOC
\phantomsection
\addcontentsline{toc}{chapter}{\protect\numberline{\listtablename}}
\listoftables

% \cleardoublepage % to get correct page no. in TOC
% \phantomsection
% \addcontentsline{toc}{chapter}{\protect\numberline{\lstlistlistingname}}
% \lstlistoflistings
% \todo[inline]{Fill or remove?}


\renewcommand{\baselinestretch}{1.0}
\normalsize

\cleardoublepage % to get correct page no. in TOC

\setlength{\parskip}{\parspacing}

\clearpage
\renewcommand{\chaptername}{\chapname} %if name appears in header

\cleardoublepage

% Hier wird das Symbolzeichnis eingef�gt
% Falls es nicht gew�nscht ist, sind die folgenden sechs Zeilen auszukommentieren
\cleardoublepage % to get correct page no. in TOC
\phantomsection
\markboth{\bfseries LIST OF SYMBOLS}{\bfseries LIST OF SYMBOLS}
\addcontentsline{toc}{chapter}{\protect\numberline{List of Symbols}}
\chapter*{List of Symbols}
\label{sec:symbolverzeichnis}


$x(n)=x'(n)+ j\cdot x''(n)$\\
$x'(n)$				Real part\\
$x''(n)$			Imaginary part\\

\begin{table}[htbp]
\begin{tabular}[t]{ll}
$\Psi(f)$       &   Weighting function for generalized cross correlation\\
% HNV					&	Haupt- zu Nebenmaximumverh�ltnis.\\
% MF					&	Merit-Faktor.\\
% $\gamma$		&	Phasenoffsets des Kanals.\\
% $\lambda$		&	Anzahl der Symbole pro Rahmen.\\
% $\xi$				&	Anzahl der Pr�ambelsymbole.\\
% $\psi$			&	Pilotsymbolanzahl.\\
% $\alpha$		&	Zeitkompressionsfaktor.\\
% cp					&	Clock-Precision in Simulink.\\
% $r_{O}$			&	Output-Rate.\\
% $r_{I}$			&	Input-Rate.\\
% $\tau$			&	Anzahl an Pilotsymbolen pro Sequenz.\\
% $S_{i}$			&	i-te - Datensequenz.\\
% $\sigma$		&	Anzahl an Datensequenzen pro Rahmen.\\
% $\beta_{R}$	&	Anstiegsfaktor f�r die Einh�llendenberechnung.\\
% X						& Einschaltgrenze f�r die Rahmenerkennung.\\
% $\beta_{F}$	&	Abfallfaktor f�r die Einh�llendenberechnung..\\
% $t_{F}$			& Abfallzeit in Samples f�r die Einh�llendenberechnung.\\
% $E_{S}$			& mittlere Symbolenergie\\		
% K						&	Schwellwertfaktor f�r die Rahmenerkennung.\\
% $\overline{Corr}$ &	Mittelwert des Korrelationsergebnisses.\\
% $\sigma_{Corr}$	& Standardabweichung des Korrelationsergebnisses.\\
% $\overline{Max}$ &	Mittelwert der Maxima des Korrelationssignals.\\
% $\sigma_{Max}$ &	Standardabweichung der Maxima des Korrelationssignals..\\
% S						&	dynamische Schwelle f�r die Rahmenerkennung.\\
% $P_{F}$			&	Falschdetektionswahrscheinlichkeit bei der Rahmenerkennung.\\
% $P_{N,S}$		&	Nichterkennungswahrscheinlichkeit bei der Rahmenerkennung.\\
% $P_{E}$			&	Falscheinschaltwahrscheinlichkeit f�r die Rahmenerkennung.\\
% $P_{Ab}$		&	Falschabschaltwahrscheinlichkeit f�r die Rahmenerkennung.\\
% $E_{R}$			& Erwartungswert der Rayleighverteilung
\end{tabular}  
%\label{}  
\end{table}


\listoftodos[ToDos]

\cleardoublepage
\setcounter{page}{1}
\renewcommand{\thepage}{\arabic{page}}

%============================================================================
%============================================================================

\cleardoublepage
\chapter{Introduction}

\todo[inline]{
Margins needs to be reduced back in layout.tex to
"oddsidemargin 2.7cm", "evensidemargin 2.6cm"
and "usepackage[paperwidth=275.9mm, paperheight=279.4mm]{geometry}"
in Thesis.tex needs to be removed.}

\cleardoublepage
\chapter{Prerequisites}

- To localize a whistle sound source, it must be detected firstly (was done by previous work
and is implemented)
- calculate the direction of the whistle sound on one Nao\\
- This can be done by determining the \ac{TDOA} of microphone pairs\\
- With the delay, the angle of the sound source relative to angle pair
can be is known.\\
- By combining these, a direction ray is defined on each Nao\\
- The results are filtered by updating it with the singe rays, assuming
gaussian distribution. (known error, Trigonometry)

- why other theories do not fit into this problem
(finger printing, beam forming)

- CC in time domain, because low frequency resolution (44100Hz/512Samples=resolution)
and also \ac{CC} corrupted and we want to detect which signal was first
- reverberation is a problem and assumed as not multi-path system
- entropy for large scale start detection
- zcr and energy for smaller scale (because high precision needed)

\missing[inline]{more content}

\section{Whistle Signal}
\label{sec:02_whistleSignal}

In this work the localization of a whistle sound source is to be to the fore.
Detection of the whistle is done in frequency domain by assuming the whistle
sound to be higher than 2000Hz and lower than 4000Hz.
By comparing the mean of the signal between this band with the overall mean of the
received signal, a peak arising around the whistle frequency can be detected.
For the whistle detection, only one channel of the robot is used and the
mean of the whistle band must exceed the threshold multiple cycles in a row.
If the team takes action due to the detected signal on individual robots is a team decision.

Further on for this work, the mathematical model of a received whistle
signal at one of the four microphone sensor is defined as
\be
x_i(t) = s_i(t) + n_i(t) \text{ for } i \in \{0, 1, 2, 3\}
\label{eq:02_whistleSignal}
\ee
where $s(t)$ represents the signal and $n(t)$ noise.
Both are assumed as real, jointly stationary random processes.
\section{Time Difference Of Arrival}

The direction of a signal source can be computed by the relative time delay between
two microphones.
- describe how the direction is defined by TDOA
It is assumed that the distance from the sensors to the sound source is
significantly large so that the signal waves proceed parallel.
Then, calculations for the direction of the sound source can be done with a simple
geometrical approach like done in \cite{Valin_Michaud}.

With the definition of a whistle signal as stated in \ref{eq:02_whistleSignal},
the microphone sensors $mic_1$ and $mic_2$ will output
\bsub \bal
    x_1(t) &= s(t) + n_1(t)\\
    x_2(t) &= \alpha s(t - D) + n_2(t).
\eal \esub
\label{eq:02_signalTimeDomain}
Here, $D$ is the delay of $x_2$ relative to $x_1$ for which is looked for.
\section{Cross Correlation}
\label{sec:cc}

The \ac{CC} provides information about the similarity of two signals.
Thus, the delay of one signal can be detected where the \ac{CC} $r_{12}(t)$ is largest.
In time domain, the \ac{CC} of two signals $x_1$ and $x_2$ is denoted as
\bal
    R_{x_1x_2}(t) = x_1(t) \circledast x_2(t) = \int^{+\infty}_{-\infty}x_1(\tau-t)x_2(\tau)d\tau.
\eal
Considering the frequency domain, the function can be transformed into
\bal
    \mathcal{F}[R_{12}(t)] = G_{x_1x_2}(f) = X_1^*(f)X_2(f)
\eal
with $\mathcal{F}[x_i(t)] = X_i(f)$ and $X_i^*(f)$ indicating the conjugate complex form.
However, the finite observation time of the received signal corrupts the fourier
transform \cite{K_C_GCC}
and noise of sensors may introduce false peaks in the \ac{CC} \cite{H_B_GCC}.
In frequency domain, the signals $x_1(t)$ and $x_2(t)$ from \ref{eq:02_signalTimeDomain}
can be expressed as
\bsub \bal
    X_1(f) &= S(f) + N_1(f)\\
    X_2(f) &= \alpha S(f) e^{-j2\pi fD}+ N_2(f).
\eal \esub
\label{eq:02_signalFreqDomain}
Thus, the \ac{CC} is
\unsure[]{better formatting?}
\bsub \bal
    G_{x_1x_2}(f) &= \alpha |S(f)|^2 e^{-j2\pi fD} + N_1^*(f)N_2(f) + S^*(f) N_2(f) + \alpha S(f) e^{-j2\pi fD}N_1^*(f)\\
\intertext{which will be shortened as}
    G_{x_1x_2}(f) &= \alpha \phi_s(f) e^{-j2\pi fD} + \phi_n(f) + \phi_c(f) \label{eq_02_Gx1x2_simple} \\
\intertext{where}
\phi_s(f) &= |S(f)|^2 \label{eq:02_phi_s} \\
\phi_n(f) &= N_1^*(f)N_2(f) \label{eq:02_phi_n1n2} \\
\phi_c(f) &= S^*(f) N_2(f)+\alpha S(f)e^{-j2\pi fD}N_1^*(f) \label{eq:02_phi_c}.
\eal \esub
\label{eq:02_Gx1x2}
%\cite{H_B_GCC}
Considering the ideal case where $s(t)$, $n_1(t)$ and $n_2(t)$ are uncorrelated, the terms
$\phi_c$ and $\phi_n$ disappear and the \ac{CC} results in
\bal
    R_{12}(t) = \mathcal{F}^{-1}[\alpha \phi_s(f) e^{-j2\pi fD}] = \alpha \mathcal{F}^{-1}[\phi_s(f)] \circledast \delta(t-D).
\eal
\label{eq_02_R12_noNoise}
\unsure[]{do I fully understand this? Is this correct?}
This means there exists a peak at delay $D$ which is altered by the \ac{iFT}
of the signal spectrum.
In general, $\phi_c$ and $\phi_n$ can neither be neglected nor assumed as uncorrelated to the signal \cite{H_B_prob},
so that they introduce inaccuracies and errors.

\section{Generalized Cross Correlation}

- improve normal cross correlation\\
- whistle frequency between 2000Hz and 4000Hz (low-pass)\\
- thus, the peaks of the cross-correlation are wide\\
- to avoid this circumstance, the generalized cross correlation (GCC)
was introduced by Knapp and Carter 1976 \cite{Knapp_Carter_GCC} \\
- weightings before iFFT to maximize expected signal peak relative
to output noise \cite{Hassab_Boucher_GCC}


- 
"The filters have a function to ensure a large sharp peak in the obtained cross-versions of the two received signals. The filters have a function to ensure a large sharp peak in the obtained cross- correlation thus ensuring a high time delay resolution in
function to ensure a large sharp peak in the obtained cross- correlation thus ensuring a high time delay resolution in generalized cross-correlation method.
correlation thus ensuring a high time delay resolution in generalized cross-correlation method." - 
\section{Signal Start Detection}
\label{sec:02_signalStartDetection}

One focus of the whistle signal localization is the correct choice of the
signal frame, with which the \ac{TDOA} calculation is done.
Assuming that the clearest signal without reverberation and with minimal
multipath propagated samples is at the start of a sound signal,
the frame to examine is chosen to be at the beginning of a whistle-sound.

By knowing the frequency band of a whistle signal, the start can be detected
where these frequencies dominate.
Using this indicator only does not always give the desired accuracy, that is why
different methods are investigated in this work \cite{tdoa_sspd}.
In the next subsections, signal start detection using short time energy,
zero crossing rate and spectral entropy are subject of discussion.
Also, the methods require
various computational power. According to the circumstances, the most suitable
approach can be chosen.
Another point is, that robustness can be increased by considering these methods
in combination.
As a latter, the consensus of the single methods can be passed as information
about the certainty of the computed direction result.

\subsection{Short Time Energy and Zero Crossing Rate}

A common method in signal start and endpoint detection is the evaluation
of the short time energy and \ac{ZCR}.

\subsubsection{Short Time Energy}
The energy
\bal
    E &= \sum_{n=1}^N E_s(n)
    \label{eq:02_energy}
 \intertext{with the energy spectral density}
    E_s(n) &= |x(n)|^2
    \label{eq:02_spectralEnergy}
\eal
of signal frames with length $N$ are expected to be higher than noise frames
and therefore, noise and signal can be distinguished
according to \cite{Z_W_voiceActivity}.
A threshold needs to be specified appropriately dependent on the environment.

\subsubsection{Zero Crossing Rate}

The \ac{ZCR} of one frame $Z$ needs small computational effort in order to
identify a periodic signal in time domain.
Its formula is
\bal
    Z = \sum_{n=2}^N |sgn(x(n)) - sgn(x(n-1))|
    \label{eq:02_zcr}
\eal
with the sign function
\begin{align*}
    sign(x) =
    \begin{cases}
        1 & \quad \text{if } x\geq 0 \\
        -1 & \quad \text{if } x < 0
    \end{cases}
\end{align*}
for a discrete signal $x(k)$ of a frame with length $N$ \cite{Z_W_voiceActivity}.
A higher \ac{ZCR} is an indication for a periodic signal.
To detect the signal start, a threshold is determined dynamically.
The \ac{ZCR} mean of frames which are known to be noise only
are averaged with the mean of those frames, that include the whistle signal.
The signal start is detected at the point in time, where the \ac{ZCR} exceeds
this threshold.

\subsection{Spectral Entropy}
\label{subsec:02_Entropy}

Entropy provides information about the disorder of a system.
From this, one can derive that noise has a high entropy compared to
a whistle-sound, which is a highly structured sound signal and a high
amount of information accordingly.
The spectral entropy of a signal is determined by normalizing the
\ac{PDF} over all frequency components as described in \cite{S_J_entropy}.
When $X(n)$ is the \ac{DFT} of the sampled signal $x(n)$, the \ac{PDF} is
\bal
    P(n) &= \frac{E_s(n)}{E}
    \label{eq:02_pdf}
\eal
with \cref{eq:02_spectralEnergy} as the spectral energy density function for $E_s(n)$
and $E$ as the energy.
Finally, the spectral entropy results in
\bal
    H &= - \sum_{n=1}^N P(n) log_2P(n).
    \label{eq:02_entropy}
\eal
Utilizing some a-priori knowledge about the signal, the entropy estimation
can be improved.
In this work, the frequency of a whistle-sound is known to be
between $2\si{\kilo\hertz}$ and $4\si{\kilo\hertz}$ from \cite{Hasselbring},
Thus, only the frequency components in the whistle range is considered.
% \unsure[]{weighting factors as described in S-J-entropy?}
Differentiating between noise samples where no signal is present and
signal frames, a dynamic threshold can be set to detect the signal start point.
%\section{Signal Phase Difference}
\label{sec:02_phase}

With a different approach to the correlation methods, the \ac{TDOA} can be
detected by observing the phase of one certain frequency $f_c$.
The phase of a signal frequency is easily computable in frequency domain
with
\bal
    \phi(f_c) &= tan^{-1}(\frac{imag(X(f_c))}{real(X(f_c))}).
\eal
With the difference of the phases of two channel, the delay in meters is defined as
\bal
    D &= \frac{\Delta \phi \cdot c_s}{2 \pi \cdot f_c}.
\eal
From that, the direction angle calculation of \cref{eq:02_tdoaAngle} can
be followed.
%\input{content/02_Trigonometry}

%\input{content/02_filter}


\cleardoublepage
\chapter{Implementation}

\missing[inline]{
    * Alsa\\
    * Where in HULKs code and general HULKs framework\\
    * How WhistleLocalization is introduced into framework\\
    * Show that channel signals from front and rear have same mean magnitude\\
    * TDOA\\
    \tab * CC Implementation\\
    \tab * GCC Implementation\\
    \tab * Phase Diff Implementation\\
    * Rear distance calculation\\
    * How TeamWhistleLocalization is implemented\\
    * Also mention things that did not work:\\
        \tab * SNR\\
        \tab * Magnitude to check rough direction
}

\section{Signal Start Detection}
\label{sec:03_signalStartDetection}

As mentioned in \ref{sec:02_signalStartDetection}, the detection of the
signal start is crucial for the localization.
The implementation of the different methods will be presented coupled with
an examination of real measurement data. By knowing the frequency band
of a whistle signal, the start can be detected where these frequencies dominate.
Using this indicator only does not always give the desired accuracy, that is why
different methods are investigated in this work. Also, the methods require
unequal computational power. According to the circumstances, the most suitable
approach can be chosen.
Another point is, that the robustness can be increased by considering these methods
in combination.\change[]{Move this into theory part?}
As a latter, the consensus of the single methods can be passed as information
about the certainty of the computed direction result.
\\
To reduce undesirable effects and demonstrate the simplest form, a sinusoidal
signal of $3000\si{Hz}$ with a sample rate of $44100\si{Hz}$. For this
data, the sound source was placed $2\si{m}$ in front of the robot.
\\
In order to find the time point where the signal starts, information about
smaller fractions are required.
So, the original $44100$ samples that were buffered by the
\change[]{correct font and wording}
"WhistleLocalization" module are divided into several overlapping
frames with size $256$. The computational effort raises with smaller frame size,
but delivers a higher precision in return.
To compute the energy and entropy, the frames are transformed into
frequency domain with the \ac{FFT}.
The \ac{ZCR} does not require such a transformation.
\missing[]{final start index by combination of these 3 methods}
For better visualization, the following data is shortened to $2400$ samples.


\subsubsection*{Spectral Entropy}

The formula to calculate the spectral entropy of a signal is \ref{eq:02_entropy}.
- use non-cleaned signal because of entropy information
- first frames are known as noise only
- mean of noise signal can be set as threshold
- look from back until threshold is exceeded. This is signal start index

% Variable \lstinline!calcFFt!

\begin{figure}[ht]
	\centering
		\includegraphics[]{figures/sine_entropy}
	\caption{Entropy of a sinusoidal signal with 3000Hz.}
\end{figure}
\label{fig:03_entropy}

\subsubsection*{Energy}

\ref{eq:02_energy} results in the energy of each frequency component of each frame.
According to this, the energy of one frame is \ref{eq:02_energy}
As the frames of the whole signal are overlapping, the energy function plotted in
\ref{fig:03_energy} results from overlapping and adding the frame energies.\\
- with a priori knowledge: only look for energy between 2000Hz and 4000Hz
One downside of the energy information is that the threshold can not be
set dynamically. It has to be adapted manually for the related environment.

\begin{figure}[ht]
	\centering
		\includegraphics[]{figures/sine_energy}
	\caption{Energy of a sinusoidal signal with 3000Hz.}
\end{figure}
\label{fig:03_energy}

\code{signal_processing}{7}{19}{Whistle energy}

\subsubsection*{Zero Crossing Rate}

- count the sign changes in frame
- calculate the noise mean at the beginning of signal
- calculate signal mean at end of signal
- mean of both is set as threshold
- start index is detected at that point where zcr is higher than this threshold

\begin{figure}[ht]
	\centering
		\includegraphics[]{figures/sine_zcr}
	\caption{Zero Crossing Rate of a sinusoidal signal with 3000Hz.}
\end{figure}
\label{fig:03_zcr}
\section{Front and Rear Distance}
\label{sec:03_distance}

If the front and rear direction candidates dissolve each other to a small
value, it is assumed that the signal source is on the X-axis of the robot.
The side delays are then used as indicator for the distance to the robot as
explained in \cref{sec:02_distance} and need to be smaller than
\change[]{change name?} $samples_{xz}$.
To estimate the distance, the height of the sound source needs to be set as
constant \change[]{change name?} \lstinline!height_source!.\\
Restrictions of the front and rear distance measurement differ.
For the front case, the maximal angle for a unambiguous distance calculation
is $\frac{\pi}{2}- 2\alpha$.
Thus, the maximal front distance that can be approximated shrinks to
$\Delta x = (\Delta h_{source} - \Delta h_{Nao}) \cdot \tan(\frac{\pi}{2} - 2\alpha)$
according to \cref{eq:02_deltaX}.
To the rear, the maximal value for $\gamma$ is bounded by \change[]{naming?}
\lstinline!xz_delay_limit!.
Setting $\Delta h_{source}$ to 1.5\si{m} and $\Delta h_{Nao}$ to 0.57\si{m},
the maximal measurable distance to the front is about 0.66\si{m}.
With the same values and 5.3 samples as \lstinline!xz_delay_limit!, the
minimum $\Delta x$ value is 33.29\si{m}.
\code{distance}{0}{35}{Pseudo code of distance approximation on X-axis.}{03_distance}
\unsure[]{I am not really sure how pseudo codes should look like and if one is necessary anyway?}


\cleardoublepage
\chapter{Evaluation}

\section{Signal Start Detection}
\label{sec:04_signalStartDetection}

\missing[inline]{Compare the single methods and calculate error}

\cleardoublepage
\chapter{Conclusion}
\label{chap:05_conclusion}

\begin{comment}
    - Objective of this work:
        x a whistle sound locator (in xy-plane) with multiple robots
        x stationary robots and sources
        x evaluated different tdoa methods
        - found a approach to detect direction of source on single robots
\end{comment}

The purpose of this work was to find an approach to localize a whistle
signal with a multi-agent system consisting of multiple NAO robots.
Each of these robots have four microphones attached on their head which
record audio signals within a range of 150\si{\hertz} and 12\si{\kilo\hertz}
with a sample rate of 44.1\si{\kilo\hertz}.
Only stationary sound sources and robots were considered in the scope of this
work.
Roughly, the implementation can be divided into three parts.
Of capital importance is the \acf{WSDE} executed by stand-alone robots which
is realized by computing the \ac{TDOA} between the four channels on the robots' head.
Two fundamentally different approaches were evaluated to obtain a stable method
for the delay estimation.
One obtains the \ac{TDOA} by cross-correlating signal samples with standard
\acf{CC} and \acf{GCC-PHAT} algorithms in frequency domain.
The other analyses recorded samples in frequency domain in terms of
a phase of a reference frequency which is described as \acf{PD} method.
% Correct
It was assumed that the start of a signal is most reliable 
To circumvent impact of multi-path propagation and reverberation, 
was investigated and is true
Thus, \acf{SSD} approaches were compared with regard to accuracy and computational effort.
Finally, 

\section{Start Detection}
\label{sec:05_startDetection}

If the observed signal is known as whistle and computational effort is
irrelevant, 
% the evaluation shows that the start detection with the
% existing whistle detection algorithm performs best.
that a rough start detection by the exiting whistle detection
with a large frame size followed by a \ac{ZCR} detection with small
window size.
This is a good trade-off with regard to the computational effort.
% - entropy for large scale start detection\\
% - zcr and energy for smaller scale (because high precision needed)\\

\section{TDOA Methods}
\label{sec:05_methodComparison}

% Phase method: Much faster!

% GCC is more accurate in regard to each robot result -> This is
% important when less robots are used for localization.
% It can be the case when robots are broken / got penalty
% We want each robot result to be as accurate as possible so
% that we can count on each result
% Less robots means, that each wrong direction is weighted more
% and influences the final position significantly.
\section{Team Filter}
\label{sec:05_teamFilter}

% Team filter: more intelligent filtering (e.g. clustering
% intersections for multiple position candidates)


% Future work ___________________________________
% Separation of multiple sounds
% Moving sources, moving robots

% conclusion -> multimodales filtering, expizite ausreißerfilter, include prior knowledge
% of refree position

\cleardoublepage
this is my daily content

%############# appendix #################
\cleardoublepage
\renewcommand{\chaptername}{\appendixname} %if name appears in header

\begin{appendix}

	\chapter{Anhang 1}
bla bla bla

bla bla bla



	% \cleardoublepage
	% \chapter{Anhang 2}
bla bla bla

bla bla bla



\end{appendix}

\cleardoublepage
\phantomsection
\addcontentsline{toc}{chapter}{\protect\numberline{List of Abbreviations}}
\chapter*{List of Abbreviations}
\label{sec:abbreviations}
\markboth{\MakeUppercase{List of Abbreviations}}{\MakeUppercase{List of Abbreviations}}

\begin{acronym}[HSU-HH]
  \acro{GCC}[GCC]{Generalized Cross Correlation}
  \acro{HSU}[HSU-HH]{Helmut-Schmidt-University/University of the Federal Armed Forces Hamburg}
\end{acronym}


%Softwareverzeichnis
\cleardoublepage
\phantomsection
\addcontentsline{toc}{chapter}{\protect\numberline{List of Software}}
\chapter*{List of Software}
\markboth{\MakeUppercase{List of Software}}{\MakeUppercase{List of Software}} 
\label{sec:software}
\begin{table}[htb]
	\centering
		\begin{tabular}{|l|l||l|}
		\hline
			\bfseries{Name} & \bfseries{Version} & \bfseries{URL}\\
		\hline
		\hline
			Python3 & 3.7.4 & https://docs.python.org/3/\\
		\hline
			NumPy & 1.17.0 & https://numpy.org/\\
		\hline
			\ac{ALSA} & 1.17.0 & https://www.alsa-project.org/\\
		\hline
			HULKs Framework & 2019 & https://github.com/HULKs/HULKsCodeRelease\\
		\hline
		\end{tabular}
	\caption{Utilized Software.}
	\label{tab:VerwendeteSoftware}
\end{table}


\cleardoublepage
%########### Bibliography ##############
\clearpage % to get correct page no. in TOC
\phantomsection
\addcontentsline{toc}{chapter}{\protect\numberline{\bibname}}

\renewcommand{\chaptername}{} %if name appears in header

\renewcommand{\baselinestretch}{1}
\normalsize

% if enabling this command all references of the bib file are listed
% otherwise only the referenced books are listed in Bibliography section
%\nocite{*}

%\bibliographystyle{unsrt}
%\bibliographystyle{alpha}
%\bibliographystyle{ALPHADIN}

\bibliographystyle{IEEEtran} %IEEE Style

% In literature.bib sind die eigentlichen Literaturangaben enthalten
\bibliography{postcontent/literature} % requires file literature.bib


\ifmakeindex
	\cleardoublepage
	\phantomsection
	\addcontentsline{toc}{chapter}{\protect\numberline{Index}}
	% Stichwortverzeichnis endgueltig anzeigen
	\printindex
\fi

\end{document}
