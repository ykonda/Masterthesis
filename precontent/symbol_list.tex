\chapter*{List of Symbols}
\label{sec:symbolverzeichnis}


$x(n)=x'(n)+ j\cdot x''(n)$\\
$x'(n)$				Real part\\
$x''(n)$			Imaginary part\\

\begin{table}[htbp]
\begin{tabular}[t]{ll}
$\Psi(f)$       &   Weighting function for generalized cross correlation\\
% HNV					&	Haupt- zu Nebenmaximumverh�ltnis.\\
% MF					&	Merit-Faktor.\\
% $\gamma$		&	Phasenoffsets des Kanals.\\
% $\lambda$		&	Anzahl der Symbole pro Rahmen.\\
% $\xi$				&	Anzahl der Pr�ambelsymbole.\\
% $\psi$			&	Pilotsymbolanzahl.\\
% $\alpha$		&	Zeitkompressionsfaktor.\\
% cp					&	Clock-Precision in Simulink.\\
% $r_{O}$			&	Output-Rate.\\
% $r_{I}$			&	Input-Rate.\\
% $\tau$			&	Anzahl an Pilotsymbolen pro Sequenz.\\
% $S_{i}$			&	i-te - Datensequenz.\\
% $\sigma$		&	Anzahl an Datensequenzen pro Rahmen.\\
% $\beta_{R}$	&	Anstiegsfaktor f�r die Einh�llendenberechnung.\\
% X						& Einschaltgrenze f�r die Rahmenerkennung.\\
% $\beta_{F}$	&	Abfallfaktor f�r die Einh�llendenberechnung..\\
% $t_{F}$			& Abfallzeit in Samples f�r die Einh�llendenberechnung.\\
% $E_{S}$			& mittlere Symbolenergie\\		
% K						&	Schwellwertfaktor f�r die Rahmenerkennung.\\
% $\overline{Corr}$ &	Mittelwert des Korrelationsergebnisses.\\
% $\sigma_{Corr}$	& Standardabweichung des Korrelationsergebnisses.\\
% $\overline{Max}$ &	Mittelwert der Maxima des Korrelationssignals.\\
% $\sigma_{Max}$ &	Standardabweichung der Maxima des Korrelationssignals..\\
% S						&	dynamische Schwelle f�r die Rahmenerkennung.\\
% $P_{F}$			&	Falschdetektionswahrscheinlichkeit bei der Rahmenerkennung.\\
% $P_{N,S}$		&	Nichterkennungswahrscheinlichkeit bei der Rahmenerkennung.\\
% $P_{E}$			&	Falscheinschaltwahrscheinlichkeit f�r die Rahmenerkennung.\\
% $P_{Ab}$		&	Falschabschaltwahrscheinlichkeit f�r die Rahmenerkennung.\\
% $E_{R}$			& Erwartungswert der Rayleighverteilung
\end{tabular}  
%\label{}  
\end{table}
