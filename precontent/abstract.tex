\pdfbookmark[1]{\abstractname}{sec:abstract}  % Bookmark im pdf file
\begin{abstract}
\label{sec:abstract}
\addtocounter{pageno}{1}
\setcounter{page}{\arabic{pageno}}
\thispagestyle{plain}
\begin{center}
\begin{minipage}[t]{0.8\linewidth}
\setlength{\parskip}{\parspacing}

% Sound source localization is a ongoing topic in research that receives
The RoboCup Standard Platform League provides a platform\todo[inline]{zwei mal "platform". Vielleicht "is a competition"} for prospective
researchers to compete in autonomous robot soccer
with the overall goal to contribute to research in the fields
of humanoid robotics and autonomous multi-agent systems.
Per specification, implementation is done on NAO robots.\todo[inline]{"per specification" unueblich. Vielleicht einfach,"according to the rules of this league," oder ohne den ersten Teil "The hardware used in this competition is the NAO robotic system."}
%  which
% come with four microphones attached on the head.
Currently, audio signals are only used as indicator for the kickoff
in form of a whistle sound.
To prevent the detection of false positives from neighboring fields,
the realizability of a whistle sound source localization
is evaluated.\todo[inline]{Du hast den algorithmus ja auch designed und implementiert. Vielleicht: "...from neighboring fields, a whistle sound source localization is designed and implemented."}
%  which is based on a \acf{TDOA} approach.
% This work evaluates the realizability of a sound source localization according
% to these whistles with an approach based on the \acf{TDOA}.
% In order to obtain the time delay between the microphones,
% three methods were investigated
Different methods that are based on the \acf{TDOA} are implemented\todo[inline]{damit "implemented" nicht doppelt ist, wenn du meine Aenderung oben uebernimmst, vielleicht "studied"}
to obtain the direction of the whistle source using the four
microphones attached on the robot's head.
% time delay
To compute a global position of the acoustic source,
direction estimates of multiple robots are fed into a multi-agent
filter.
The resulting algorithm is shown to allow whistles to be
localized with an average \ac{RMSE} of 1\si{\meter} in terms of
Euclidean distance.\todo[inline]{"average RMSE" ist doppelt gemoppelt. Eins weg.}
% according
% to the distance.




\end{minipage}
\end{center}

\end{abstract}

