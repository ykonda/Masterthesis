\section{Cross Correlation}

The \ac{CC} provides information about the similarity of two signals.
Thus, the delay of one signal can be detected where the \ac{CC} $r_{12}(t)$ is largest.
In time domain, the \ac{CC} of two signals $x_1$ and $x_2$ is denoted as
\bal
    R_{x_1x_2}(t) = x_1(t) \circledast x_2(t) = \int^{+\infty}_{-\infty}x_1(\tau-t)x_2(\tau)d\tau.
\eal
Considering the frequency domain, the function can be transformed into
\bal
    \mathcal{F}[R_{12}(t)] = G_{x_1x_2}(f) = X_1^*(f)X_2(f)
\eal
with $\mathcal{F}[x_i(t)] = X_i(f)$ and $X_i^*(f)$ indicating the conjugate complex form.
However, the finite observation time of the received signal corrupts the fourier
transform \cite{Knapp_Carter_GCC}
and noise of sensors may introduce false peaks in the \ac{CC} \cite{Hassab_Boucher_GCC}.
In frequency domain, the signals $x_1(t)$ and $x_2(t)$ from \ref{eq:02_signalTimeDomain}
can be expressed as
\bsub \bal
    X_1(f) &= S(f) + N_1(f)\\
    X_2(f) &= \alpha S(f) e^{-j2\pi fD}+ N_2(f).
\eal \esub
\label{eq:02_signalFreqDomain}
Thus, the \ac{CC} is
\bsub \bal
    G_{x_1x_2}(f) = \alpha |S(f)|^2 e^{-j2\pi fD} + N_1^*(f)N_2(f) + S^*(f) N_2(f) + \alpha S(f) e^{-j2\pi fD}N_1^*(f)
\eal \esub
which will be shortened as
\bsub \begin{align*}
    G_{x_1x_2}(f) &= \alpha \phi_s(f) e^{-j2\pi fD} + \phi_{n_1n_2}(f) + \phi_c(f)\\
\intertext{where}
    \phi_s(f) &= |S(f)|^2
\end{align*} \esub


- improve this estimation with filtering\\
- noise can not be seen as uncorrelated
