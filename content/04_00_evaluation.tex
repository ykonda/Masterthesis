\chapter{Evaluation}

\change[]{Allgemeine Anmerkung: Ich finde es persoenlich bei diesen
Kapiteleinleitungen schoener, wenn die "section" immer am Anfang des satzes steht
damit man nicht vor und zurück springt. Aber das ist geschmackssache.}

In this chapter, experimental results are presented while setting
focus on different aspects.
Before evaluating the performance of the final localization algorithm the
\ac{WSDE} on single robots is examined by first taking a closer look
at a representative example in \cref{sec:04_tdoaSingle}.
There, the focus is set on the \ac{TDOA} methods itself for each channel pair
before combining them to a robot direction result.
\Cref{subsec:04_singleRobotAngleError} presents the validity of the methods with
regard to a whole set of data.

% Additional Info
In \cref{chap:03_implementation}, the presence of additional
information next to the estimated \ac{WSDE} on individual robots
is discussed.
This includes knowledge about \acp{SNR} of the channel signals, the \ac{PSNR}
of the \ac{GCC} functions and a distance approximation if the signal source
is detected straight in front or from behind.
\Cref{sec:04_additionalInformation} analyses if and to
what extent these factors are useful for the \ac{WSDE} or \ac{WSL}.

% Team filter
To examine the whistle sound source localization of the robots as team,
measurements were taken with five Naos on the field of the HULKs' laboratory
as specified in \cref{subsec:04_labMeasurements}.
In \cref{sec:04_multiAgentLoca} the performance of the \ac{TDOA} methods are presented
and compared for all measurements.
Depending on the accuracy of the individual direction results, the quality
of the team filter is limited.
% Therefore, the angular error of the single robots is demonstrated for each
% measurement and method.

All measurements presented in this chapter were recorded in the
laboratory of the HULKs and are taken from a sound source at a height of
\SI{1.5}{\meter} above ground.
The size of the field used in this work is smaller than the regular \ac{SPL}
field with 7.5\si{m} length and 5\si{m} width instead of 9\si{\meter} and 6\si{\meter}.
This circumstances occur due to lack of space and limited size of the
field room in the laboratory.
Another deviation to competition conditions arise due to walls being next to field borders.
In \cref{subsec:04_labMeasurements} further information about the measurement setup
is introduced.

\subsection{Measurement Setup}
\label{subsec:04_labMeasurements}

Eleven measurement were taken with five robots on the small-sized \ac{SPL} field
of the HULKs laboratory.
\Cref{fig:04_setup} illustrates the positions of the NAO robots
and the positions of the whistle sound sources.
% -------------------------------------------------------------
\begin{figure}[ht]
	\centering
		\includegraphics[]{figures/evaluation/setup}
	\caption{Setup of robots and sound source positions for the evaluation measurement.}
    \label{fig:04_setup}
\end{figure}
% -------------------------------------------------------------

According to these, the x- and y-coordinates of the sound source positions and
robots are listed in \cref{tab:04_robots} and \cref{tab:04_sources}, respectively.
Additionally, the positions are named for easier memorability.
The orientation $\theta$ of the robots are defined relatively to the global
x-axis as defined in \cref{subsec:03_coordinates}.
In the following, the measurements introduced in this section are
referred to as \textit{laboratory-dataset}.
% -------------------------------------------------------------
\btline{ht}{1.2}
\btab{|c|c|c|c|}
\hline
NAO & x [\si{m}] & y [\si{m}] & $\theta$ [\si{deg}]\\
\hline
21 & 3.75 & 2.5 & -40.2\\
\hline
24 & 3.75 & -2.5 & 90\\
\hline
26 & 0 & 0 & 0\\
\hline
27 & -3.75 & -2.5 & 66.06\\
\hline
28 & -2.45 & 0 & 0\\
\hline
\etab
\et{Robot positions of the laboratory-dataset}{04_robots}
% -------------------------------------------------------------
\btline{ht}{1.2}
\btab{|c|c|c|c|}
\hline
Measurement & Position Name & x [\si{m}] & y [\si{m}]\\
\hline
0 & front left & 3.75 & 2.5\\
\hline
1 & front right & 3.75 & -2.5\\
\hline
2 & rear right & -3.75 & -2.5\\
\hline
3,9 & rear left & -3.75 & 2.5\\
\hline
4 & own penalty spot & -2.45 & 0\\
\hline
5 & opponent penalty spot & 2.45 & 0\\
\hline
6,10 & center & 0 & 0\\
\hline
7 & center right & 0 & -2.5\\
\hline
8 & behind own goal & -6.05 & 0\\
\hline
\etab
\et{Positions of the whistle sources in the laboratory-dataset}{04_sources}
% -------------------------------------------------------------

\section{Signal Start Detection}
\label{sec:04_signalStartDetection}

\missing[inline]{Compare the single methods and calculate error}
\section{Single Robot Method Comparison}
\label{sec:04_tdoaSingle}

For comparability of the results, one exemplary measurement is utilized
to present and analyse the \ac{TDOA} methods.
In this data set, the sound source is placed at the right front
of the robot with 4.5\si{m} distance.
This corresponds to an an angle of -33.7\si{\degree} in robot coordinates.
Some samples around the signal start of the received signal data
is plotted in \ref{fig:04_tdoaSignal} for all channels.
% -------------------------------------------------------------
\begin{figure}[ht]
	\centering
		\includegraphics[]{figures/evaluation/cc_frontRight_1_signal}
	\caption{Signal start section of a whistle sound recorded from front right.}
	\label{fig:04_tdoaSignal}
\end{figure}
% -------------------------------------------------------------

As the next sections focus on the performance of the \ac{TDOA} methods,
the start index is set manually.
A frame size is defined with 256 samples and is selected around the start index for
the correlation methods as explained in \cref{sec:03_tdoa}.
For the phase method, the first frame with a size of 64 samples
is chosen where a whistle is detected in all channels.

In the following, the correlation function $R_{x_ax_b}$ of two signals
$x_a$ and $x_b$ as in \cref{chap:02_prerequisites} is shortened
to $R_{ab}$ for simplicity.


\subsection{Cross Correlation}
\label{subsec:04_ccSingle}
% -------------------------------------------------------------

To visualize the result of the \ac{CC}, the correlations are plotted in
\cref{fig:04_cc}. For $R_{23}$ and $R_{13}$ the peak is clearly traceable.
However, for the other \ac{CC} the problem of a low maximum peak
arises as mentioned in \cref{sec:02_cc}.
% -------------------------------------------------------------
\begin{figure}[ht]
	\centering
		\includegraphics[]{figures/evaluation/cc_frontRight_1}
	\caption{Cross correlation results of signal from front right (-33,7\si{\degree}).}
	\label{fig:04_cc}
\end{figure}
% -------------------------------------------------------------
\btline{ht}{1.2}
\btab{|c|c|c|c|c|}
\hline
Base Channel & Next Channel & Delay & Candidate (-) & Candidate (+)\\
\hline
0 & 1 & -8.25 & -144.9 & -35.1\\
\hline
1 & 3 & -4.59 & -17.4 & 78.6\\
\hline
2 & 0 & 9.16 & -30.6 & -30.6\\
\hline
3 & 2 & 3.94 & -150.2 & -29.8\\
\hline
\etab
\et{Cross correlation delay results of singal from front right}{04_cc}
% -------------------------------------------------------------

According to the delays in \cref{tab:04_cc}, the final result of the \ac{CC}
is -26.9\si{\degree} which emerges an error of 6.8\si{\degree}.
The delay between channel 2 and 0 is larger than the maximal delay of 6.85 samples
and therefore cut to the maximal sample delay.
Besides these, the \ac{TDOA} between the channel pairs produce one appropriate
direction candidate which correctly points to the sound source.
% -------------------------------------------------------------

\subsection{Generalized Cross Correlation}
\label{subsec:04_gccSingle}
% -------------------------------------------------------------
\Cref{fig:04_gcc} presents the cross correlation result by the \ac{GCC} method of
the same signal data as in the previous section.
The subsample delays for each channel pair and their resulting direction candidates
are listed in \cref{tab:04_gcc}.
From this, a final direction of -30.0\si{\degree} is determined
resulting in an error of 3,69\si{\degree}.
It is apparent that the peaks of the \ac{GCC} are better to detect than the peaks of the
\ac{CC}.
% -------------------------------------------------------------
\begin{figure}[ht]
	\centering
		\includegraphics[]{figures/evaluation/gcc_frontRight}
	\caption{Generalized cross correlation results of signal from front right.}
	\label{fig:04_gcc}
\end{figure}
% -------------------------------------------------------------
\btline{ht}{1.2}
\btab{|c|c|c|c|c|}
\hline
Base Channel & Next Channel & Delay & Candidate (-) & Candidate (+)\\
\hline
0 & 1 & -8,28 & -144,7 & -35,3\\
\hline
1 & 3 & -4,09 & -22,8 & 84,0\\
\hline
2 & 0 & 7,60 & -30.6 & -30.6\\
\hline
3 & 2 & 4,13 & -148,7 & -31,3\\
\hline
\etab
\et{Generalized cross correlation delay results of singal from front right}{04_gcc}
% -------------------------------------------------------------
\subsection{Phase Difference}
\label{subsec:04_phaseSingle}

For detecting the source direction with phase difference, a smaller frame
size of 64 samples is set.
Previously, two cases were introduced in \cref{subsec:03_phase} where either a
fixed frequency is focused on or the frequency with maximal magnitude is
taken for reference.

First, the result of the dynamically selected frequency is presented.
As stated in the implementation chapter, the frame is chosen where the
frequencies of the maximal amplitudes coincides for all channel which is
at 2756,25\si{\hertz}.
In the upper plot of \cref{fig:04_phaseSingle} one sees the received samples which
will be transformed into frequency domain by \ac{FFT} and using a Hann window.
The resulting phases and amplitudes are listed in \cref{tab:04_phaseSingle}.
For comprehensibility, the determined frequency information visualized by
wave signals with these phases and amplitudes
in the lower subplot of \cref{fig:04_phaseSingle}.
Due to the larger distance between channels 0 and 1, the phase difference
information is neglected.
Outcome from the applied phase differences is -29,2\si{\degree} by the combination of
-17,6\si{\degree}, -30,6\si{\degree} and -39,3\si{\degree}.
% -------------------------------------------------------------
\btline{ht}{1.2}
\btab{|c|c|c|}
\hline
Channel & Phase [\si{\deg}] & Amplitude\\
\hline
0 & -1,55 & 0,00144\\
\hline
1 & -177,7 & 0,00287\\
\hline
2 & 173,4 & 0,00279\\
\hline
3 & -75,0 & 0,00372\\
\hline
\etab
\et{Phase and amplitude of frame signals with 2756,25Hz}{04_phaseSingle}
% -------------------------------------------------------------
\btline{ht}{1.2}
\btab{|c|c|c|c|c|}
\hline
Base Channel & Next Channel & Phase Difference & Candidate (-) & Candidate (+)\\
& & [\si{\deg}] & [\si{\deg}] & [\si{\deg}] \\
\hline
0 & 1 & 176,2 & 33,07 & 146,9\\
\hline
1 & 3 & -102,7 & -17,6 & 78,8\\
\hline
2 & 0 & 173,4 & -30,6 & -30,6\\
\hline
3 & 2 & 113,1 & -140,7 & -39,3\\
\hline
\etab
\et{Phase differences and resulting direction candidates of example data with phase method}{04_phaseDiffSingle}
% -------------------------------------------------------------
\begin{figure}[ht]
	\centering
		\includegraphics[]{figures/evaluation/phase_cos}
	\caption{Frames used for the direction detection by phase method.}
	\label{fig:04_phaseSingle}
\end{figure}
% -------------------------------------------------------------

Secondly, the frequency to examine $f_c$ is set to the first represented frequency
larger than 2600\si{\hertz} which is 2627,1\si{\hertz} at a \ac{FFT} length
of 256.
At this frequency, outcome of the direction candidates listed in \cref{tab:04_fixedFreqResult}
with the considered delays is -29,6\si{\degree}.
% -------------------------------------------------------------
\btline{ht}{1.2}
\btab{|c|c|c|c|c|}
\hline
Base Channel & Next Channel & Phase Difference & Candidate (-) & Candidate (+)\\
& & [\si{\deg}] & [\si{\deg}] & [\si{\deg}] \\
\hline
1 & 3 & -79,1 & -26,8 & 88,0\\
\hline
2 & 0 & 167,7 & -30,6 & -30,6\\
\hline
3 & 2 & 88,5 & -148,7 & -31,3\\
\hline
\etab
\et{Resulting candidates of phase difference method with fixed frequency
	2411,72Hz of example measurement from front right
	(-33,3\si{\degree})}{04_fixedFreqResult}
% -------------------------------------------------------------
\section{Configuration}
\label{04_configuration}

\subsection{Fixed Frequency Value}
\label{subsec:04_fixedFrequencyVal}

In order to decide on a fixed frequency for the phase method,
resulting errors of different frequency were evaluated.
For this, all measurements in \cref{subsec:04_labMeasurements}
of the robot 26 at the center point are utilized.
As the \ac{RMSE} in \cref{fig:04_diffFc} shows, for frequencies larger
than 2024,1\si{\hertz} the errors increase minimally with growing frequency.
With a frequency of 2024,1\si{\hertz}, error is largest.
Therefore, the fixed frequency is set to 2411,7\si{\hertz}.
% -------------------------------------------------------------
\begin{figure}[ht]
	\centering
		\includegraphics[]{figures/evaluation/phase_fc_rmse}
	\caption{Result of all measurements for Nao 26 to compare different
	fixed frequency values.}
	\label{fig:04_diffFc}
\end{figure}
% -------------------------------------------------------------
\newpage
\section{Additional Information}
\label{sec:04_additionalInformation}

On top of the direction detection by \ac{TDOA}, additional information
can be won by received microphone data.
This information can be used to improve the single robot result or
to feed the team filter with beta information.
As \cref{sec:03_distance} has described, the distance of the sound source
can be estimated approximately for nearby signals that are aligned with
the x-axis of the robot's head.
Another intuitive approach is the inspection of the \ac{SNR} which
is assumed as higher for closer sources.
Apart from this, the \ac{PSNR} of the \ac{GCC} defined in \cref{sec:03_snr}
is investigated.

\subsection{Distance Approximation}
\label{subsec:04_distance}

\Cref{sec:03_distance} clarified the conditions to identify the distance of
the sound source by one robot.
To examine the validity of this statement, measurements from the front and
back of the robot are collected and evaluated.
\Cref{tab:04_distance} lists the distances and their results by the
all methods.
For both measurements with zero distance, the orientation of the whistle differed.
180\si{\degree} indicates that the whistle was turned in the opposite direction
of the robot.
In the other case (0\si{\degree}), the whistle was aligned with the robot.
The distance is represented in robot coordinates, so that positive distance
expresses that the source was placed in front of the robot and oriented towards it
and vise versa.
% -------------------------------------------------------------
\btline{ht}{1.2}
\btab{|c|c|c|c|c|c|}
\hline
Nr. & Real Distance [\si{\meter}] & GCC Result [\si{\meter}] & CC Result [\si{\meter}] & Phase Result [\si{\meter}]\\
\hline
1 & 0,9 & $\infty$  & $\infty$ & $\infty$ \\
\hline
2 & 0,6 & $\infty$ & $\infty$ & $\infty$ \\
\hline
3 & 0,3 & 0,35 & 0,25 &  0,13 \\
\hline
4 & 0,0 (180\si{\degree}) & -0,13 & -0,15 & -0,23 \\
\hline
5 & -0,0 (0\si{\degree}) & 0,22 & 0,21 & 0,02 \\
\hline
6 & -0,3 & -0,15 & -0,27 & -0,45 \\
\hline
7 & -0,6 & -0.34 & -0,50 & -0,62 \\
\hline
8 & -0,9 & -0.70 & -0,91 & -0,99 \\
\hline
9 & -1,2 & -1,00 & -1,28 & -1,71 \\
\hline
10 & -1,5 & -1,39 & -1,59 & -2,98 \\
\hline
11 & -1,8 & -1,72 & -2,07 & -3,33 \\
\hline
12 & -2,1 & -2,16 & $\infty$ & -3,02\\
\hline
13 & -2,4 & -2,31 & $\infty$ & $\infty$ \\
\hline
14 & -3,75 & -3,66 & -9,51 & -4,15 \\
\hline
15 & -6,4 & -7,35 & -7,27 & $\infty$  \\
\hline
16 & -9,8 & $\infty$ & $\infty$ & $\infty$ \\
\hline
\etab
\et{Result of front and rear distance for all methods.}{04_distance}
% -------------------------------------------------------------

As the results show, the distance can be approximated with adequate error.
One can see that the \ac{GCC} results are erroneous for small distances,
but gives a correct approximation steadily.
Compared to this, the \ac{CC} method performs better for small distances,
but fails completely for some measurements.
Those failing cases mostly arose when lateral delays exceeded the maximum
lateral samples incorrectly.
The phase methods provides most incorrect results.
Especially measurement 10 stands out by being double the real value.

For all measurements, the resulting direction angle of the sound source were
accurate.
Rightfully, the algorithm detects sources that are out of observable range.

Unfortunately, for real cases one can not rely on the height parameter of the sound
source which varies from the referee's body height.
Having this as approximation only, the distance result should be handled
with care.

\subsection{SNR}
\label{subsec:04_snr}

The validity of a result can be brought in the Bayesian updating algorithm.
Depending on the uncertainty, the covariance value of the incoming result can be adjusted
so that unreliable results have less influence to the filtering.
If a relation between received signal strength and distance to the source exists,
it would be a simple way to predict the sound source position roughly.
This can help for example, when team filter intersections are clustered so that
multiple location results exist.
% Thus, one intuitive hypothesis is assuming the existence of a relation between the
% received signal strength and distance to the source.

Taking the measurements of \ref{subsec:04_labMeasurements}, this hypothesis is
investigated by looking at the relation between distance and relative \ac{SNR}.
By reason of the whistle not blown equal for all measurements,
the \ac{SNR} is scaled by the overall mean of all robots' \acp{SNR} for one
measurement.
% $\text{Relative SNR} = \frac{SNR_{robot}}{Mean(SNR_{measurement})}$.
In \cref{fig:04_snrDistance} no straightforward
link between both values can be found unexpectedly.
% -------------------------------------------------------------
\begin{figure}[ht]
	\centering
	\includegraphics[]{figures/evaluation/snr_scatter}
	\caption{Visualization of relation between SNR and distance.}
	\label{fig:04_snrDistance}
\end{figure}
% -------------------------------------------------------------

Due to the result, further analysis on the \ac{SNR} values
on individual robots is done for evaluation of the microphones' validity.
For this purpose, a signal is played back digitally from fixed distance
and constant volume with different angles.
The main purpose of this measurement is to ensure that the lone channels
are not biased.
14 measurements of a 3\si{\kilo\hertz} sine signal
with are distance of 0.73\si{m} are done but no tendency is detectable.
Further evaluation is done by determining the channel with the maximum
\ac{SNR} of one measurement.
It is expected that the nearest channel to the sound source has maximum \ac{SNR}.
At 85,71\si{\percent} of the cases this assertion could be evidenced.
From this, it can be said that the general recordings of the microphones
are neither biased nor falsified.

The same procedure is done with the real whistle recordings of the
measurement in \ref{subsec:04_labMeasurements}.
Using these measurements, only 54,55\si{\percent} of the maximum \acp{SNR}
match with the expected channels.
Consequentially, one must assume that the environmental circumstances
like multi-path propagation and reflection have large influence
on the signal data.
Thus, neither for a single robot direction result nor for the team filter
the \ac{SNR} can be utilized.

\subsection{PSNR}
\label{subsec:04_psnr}

As referred in \cref{sec:02_gcc}, the main characteristic of the \ac{GCC-PHAT}
algorithm is the emerging sharp peak.
In conclusion, one can assume that the lack of a sharp peak indicates
a less valid delay result of the \ac{GCC}.

\subsubsection*{Informative Value}

This statement is examined by comparing the \ac{PSNR} value
to the error of the direction angle resulting from the \ac{GCC-PHAT} delay.
% Two cases of errors are taken into consideration.
In the following, the \ac{PSNR} is recorded as high if it exceeds 17,5 whereat the
\ac{PSNR} value ranges from 10,1 to 28,8 for measurements in
\cref{subsec:04_labMeasurements}.

Firstly, it is looked at each channel pair by determining the
error between true direction and the most suited direction candidate emerging
from the \ac{GCC} delay.
The results are split by the \ac{PSNR} value.
If the \ac{PSNR} of the \ac{GCC} is greater than the threshold of 17,5,
the \ac{RMSE} of its result is grouped to the errors with high \ac{PSNR}.
Elsewise it pertains to the errors with low \ac{PSNR}.
This valuation is done with the measurements of \cref{subsec:04_labMeasurements}
and manually set start indexes.
Having 76 correlations assessed with low \ac{PSNR}, the \ac{RMSE} of this group
is 35,77\si{\degree}.
Compared to this, the \ac{RMSE} of the remaining 144 measurements
is 15,86\si{\degree} only.

To see the impact on a complete robot result, the same is done
with the final angular errors of single robots results.
Here, the \ac{RMSE} of the lower \ac{PSNR} case results in 25,36\si{\degree}
whereat the error of the other case is 14,41\si{\degree}.
From this, one can identify the \ac{PSNR} as valid enriching information.

\subsubsection*{Frame Selection}

Another perspective is to include the PSNR information into the single
robot whistle direction determination.
In \cref{fig:04_psnr2FrameShift}, the frame to investigate is
shifted before and after the signal start index.
Here, the same measurement as in \cref{sec:04_tdoaSingle} is used where
the robot was positioned at the center point
while the whistle is blown at -33,7\si{\degree} with 4,5\si{\meter}
distance.
The frame size of the \ac{GCC} is set to 256 samples and the shift
was set to a quarter of the frame size.
Samples of the rear left channel are plotted for better understanding
in the upper graph of \cref{fig:04_psnr2FrameShift}.
The second graph shows the \ac{RMSE} of the robot direction result
over the frame shifts.
For the lower graph, the mean over the \acp{PSNR} of all channels
is presented.
One sees, how the error is low with high \ac{PSNR} what confirms
the previous statement in this section.
For shifts smaller than -2, the signal is not represented in the
frames yet what explains the high error.
An important notice is that the result changes with proceeding
frames what indicates that the implemented \ac{GCC-PHAT} method is
not suitable for arbitrary frames.
% Without additional information about the rough direction,
% ambiguity exists due to periodicity of the signal.
% Due to the inconclusive result of the \ac{SNR} in \cref{subsec:04_snr}
% one decides not to take the signal magnitude as such information
% into account.
% Thus, the decisive role of the start of the signal is underlined again.
% -------------------------------------------------------------
\begin{figure}[ht]
	\centering
	\includegraphics[]{figures/evaluation/gcc_frame_shift}
	\caption{
		Relation between \ac{PSNR}
		and selection of the frame in time. Signal data
		of the rear left channel is plotted in the upper window.
		In this measurement, the whistle is positioned at right front
		of the robot.
	}
	\label{fig:04_psnr2FrameShift}
\end{figure}
% -------------------------------------------------------------



\section{Multi-Agent Source Localization}
\label{sec:04_multiAgentLoca}

After the three \ac{TDOA} methods \ac{CC}, \ac{GCC} and phase difference were
evaluated in the preceding sections, the \ac{SSL} algorithm with a multi-agent system
of five robots is evaluated.
The remaining part of this chapter focuses on the performance of the \ac{SSL}
with regard to each \ac{TDOA} method individually.
Input to the \ac{SSL} algorithm are the \ac{WSDE} of each
robot in the team. Based on these results, the multi-agent whistle localization outputs
an absolute sound source position in field coordinates.
To provide a decoupled result to the signal start detection,
the start indexes were set manually.
In the following, the laboratory-measurements of \Cref{subsec:04_labMeasurements}
are utilized for evaluation of all methods.

% -------------------------------------------------------------
\subsection{CC Method}
\label{04_teamCc}


To determine an overall result, each robot computes a direction prediction from
the locally recorded signal using the \ac{GCC-PHAT} method standing
alone.
These local direction estimates of individual robots are fed to the team
decision filter as specified in \cref{sec:03_multiAgentLoca} which estimates the
global sound source position by combining all measurements through Bayesian
updates.
First, the results of the \ac{SSL} algorithm are presented with
the \ac{WSDE} calculated by simple \ac{CC}.
The results for the predictions of this method are reported in
\Cref{tab:04_ccTeamResult} lists the error of the localized position
in regard to the real sound source position in x- and y-coordinates.
Additionally, the angular error in field coordinates is listed.
It indicates if the result has a correct tendency.

% -------------------------------------------------------------
\btline{ht}{1.2}
\btab{|c|c|c|c|c|c|}
\hline
& & Error & Error & Error & Error\\
No. & Measurement & x [\si{\meter}] & y [\si{\meter}] & Abs. Distance [\si{\meter}] & Angle\\
\hline
0 & front left & 0.6 & 1.39 & 1.51 & 8.15\si{\degree}\\
\hline
1 & front right & -0.49 & 1.2 & 1.3 & 11.97\si{\degree}\\
\hline
2 & rear right & 2.32 & 2.11 & 3.13 & 18.28\si{\degree}\\
\hline
3 & rear left & 1.07 & -0.96 & 1.44 & 3.71\si{\degree}\\
\hline
4 & own penalty spot & 1.95 & -0.09 & 1.95 & 10.44\si{\degree}\\
\hline
5 & opponent penalty spot & 0.07 & 0.01 & 0.07 & 0.32\si{\degree}\\
\hline
6 & center & 0.4 & -0.01 & 0.4 & 1.91\si{\degree}\\
\hline
7 & center right & 1.06 & -0.0 & 1.06 & 22.99\si{\degree}\\
\hline
8 & behind own goal & 1.24 & -0.06 & 1.24 & 0.77\si{\degree}\\
\hline
9 & rear left & -0.04 & 0.78 & 0.78 & 7.22\si{\degree}\\
\hline
10 & center & 0.03 & -0.0 & 0.03 & 0.0\si{\degree}\\
\hline
\etab
\et{Whistle localization results of laboratory-measurements with
\ac{CC} method}{04_ccTeamResult}
% -------------------------------------------------------------

Over all measurements, the \ac{CC} predictor has
a \ac{RMSE} of 1.45\si{\meter} in distance and 10.67\si{\degree} angular.

\subsection{GCC Method}
\label{04_teamGcc}

As the \ac{GCC} method provides the best results for the \ac{WSDE} results,
more precise steps of the multi-agent source localization algorithm are presented
here.
For further clarification, measurement 1 of the laboratory-measurements is
selected as example.
\Cref{fig:04_gccResult} illustrates the result of the relative direction
estimates $\gamma_i$ of the individual robots listed in \cref{tab:04_gccResult}
for this example.
For \cref{fig:04_setup,fig:04_gccResult}, robot positions are marked by yellow dots
where a short yellow line indicates each robot's orientation.
The arrows in \cref{fig:04_gccResult} represent the local direction estimates
$\gamma_i$ as predicted by each robot. Finally, the true position of the sound
source is marked with a red star while the joint position estimate over all
robots is visualized by a cross.
% -------------------------------------------------------------
\btline{ht}{1.2}
\btab{|c|c|c|c|}
\hline
NAO & $\gamma_i$ & Abs. Error\\
\hline
21 & -26.22\si{\degree} & 3.71\si{\degree}\\
\hline
24 & -133.77\si{\degree} & 9.32\si{\degree}\\
\hline
26 & -30.19\si{\degree} & 3.50\si{\degree}\\
\hline
27 & -75.26\si{\degree} & 1.71\si{\degree}\\
\hline
28 & -15.90\si{\degree} & 2.53\si{\degree}\\
\hline
\etab
\et{Resulting direction estimates of the individual robots with \ac{GCC-PHAT}
method for a whistle-sound signal in the right front corner of the playing
field}{04_gccResult}
% -------------------------------------------------------------
\begin{figure}[H]
	\centering
		\includegraphics[]{figures/evaluation/gcc_team}
	\caption{Team whistle localization result with \ac{GCC-PHAT}
	method.}
    \label{fig:04_gccResult}
\end{figure}
% -------------------------------------------------------------

The final result and its corresponding errors are listed in
\cref{tab:04_gccTeamResult}.
% -------------------------------------------------------------
\btline{H}{1.2}
\btab{|c|c|c|}
\hline
 & Result & Error\\
\hline
Position x [\si{\meter}] & 3.38 & -0.37\\
\hline
Position y [\si{\meter}] & -1.85 & 0.65\\
\hline
Angle & 33.18\si{\degree} & 1.57\si{\degree}\\
\hline
Distance [\si{\meter}] & 3.85 & 0.74 \\
\hline
\etab
\et{Whistle localization result of measurement 1 with \ac{GCC-PHAT} method}{04_gccTeamResult}
% -------------------------------------------------------------

\Cref{tab:04_gccTeamResult} shows the distance and angle errors
for all laboratory measurements in \cref{subsec:04_labMeasurements}.
The \ac{RMSE} of the localized source positions in distance being 0.87\si{\meter}
and angular \ac{RMSE} being 5.07\si{\degree} one can say that the \ac{GCC-PHAT} algorithm
works well for whistle-sound source localization.
% -------------------------------------------------------------

\btline{ht}{1.2}
\btab{|c|c|c|c|c|c|}
\hline
& & Error & Error & Error & Error\\
No. & Measurement & x [\si{\meter}] & y [\si{\meter}] & Abs. Distance [\si{\meter}] & Angle\\
\hline
0 & front left & 1.31 & 1.06 & 1.68 & 1.45\si{\degree}\\
\hline
1 & front right & 0.13 & 0.06 & 0.15 & 1.57\si{\degree}\\
\hline
2 & rear right & 0.59 & 0.43 & 0.73 & 0.42\si{\degree}\\
\hline
3 & rear left & 0.54 & 0.47 & 0.72 & 9.09\si{\degree}\\
\hline
4 & own penalty spot & 0.27 & 0.0 & 0.27 & 0.01\si{\degree}\\
\hline
5 & opponent penalty spot & 0.15 & 0.14 & 0.21 & 3.18\si{\degree}\\
\hline
6 & center & 0.41 & -0.02 & 0.41 & 2.67\si{\degree}\\
\hline
7 & center right & 0.39 & 0.02 & 0.39 & 8.98\si{\degree}\\
\hline
8 & behind own goal & 1.84 & -0.01 & 1.84 & 0.14\si{\degree}\\
\hline
9 & rear left & 0.58 & 0.52 & 0.78 & 9.89\si{\degree}\\
\hline
10 & center & 0.03 & -0.0 & 0.03 & 0.0\si{\degree}\\
\hline
\etab
\et{Whistle localization results for all laboratory-measurements with
\ac{GCC-PHAT} method}{04_gccTeamResult}
% -------------------------------------------------------------
% - intersections
% - updates
% - covariance
% - PSNR

% -------------------------------------------------------------
\subsection{Phase Method}
\label{04_teamPhase}

Finally, the performance  of the phase method is evaluated.
In this experiment, the reference frequency is set to a minimum of
2700\si{\hertz} due to the result that reference frequency larger than
2600\si{\hertz} obtain best results.
The results of phase method for this experiment are shown in
\cref{tab:04_phaseTeamResult}.
The \ac{RMSE} of the position estimate is close to the prediction accuracy of
the \ac{CC} method with 1.33\si{\meter}.
The \ac{RMSE} of global direction estimate of 74.8\si{\degree}
shows a significantly worse performance than the other two methods.
However, it must be noted that these angular error mainly arise from
measurements 6 and 10. Both measurements are taken at the center point of the
field. Since the absolute position is in an acceptable error range, these
angular results will be treated as outliers for the error calculation. Thus,
the angular \ac{RMSE} of the phase method without measurements 6 and 10 is
11.69\si{\degree}.
\unsure[]{Handle measurements 6 and 10 differently because any angle is correct?}

% -------------------------------------------------------------

\btline{H}{1.2}
\btab{|c|c|c|c|c|c|}
\hline
& & Error & Error & Error & Error\\
No. & Measurement & x [\si{\meter}] & y [\si{\meter}] & Abs. Distance [\si{\meter}] & Angle\\
\hline
0 & front left & -1.07 & -0.98 & 1.45 & 4.13\si{\degree}\\
\hline
1 & front right & 0.21 & 0.27 & 0.34 & 4.27\si{\degree}\\
\hline
2 & rear right & 0.22 & 1.39 & 1.41 & 16.25\si{\degree}\\
\hline
3 & rear left & 1.26 & -0.02 & 1.26 & 11.19\si{\degree}\\
\hline
4 & own penalty spot & 0.16 & -0.04 & 0.17 & 0.99\si{\degree}\\
\hline
5 & opponent penalty spot & -0.42 & 0.19 & 0.47 & 5.46\si{\degree}\\
\hline
6 & center & -0.32 & 0.08 & 0.33 & 166.25\si{\degree}\\
\hline
7 & center right & -0.28 & 1.76 & 1.78 & 20.82\si{\degree}\\
\hline
8 & behind own goal & 2.37 & 0.27 & 2.39 & 4.23\si{\degree}\\
\hline
9 & rear left & 2.04 & 0.29 & 2.06 & 24.89\si{\degree}\\
\hline
10 & center & -0.32 & 0.0 & 0.32 & 180.0\si{\degree}\\
\hline
\etab
\et{Whistle localization results for all measurements in \cref{subsec:04_labMeasurements} with
phase method}{04_phaseTeamResult}

% Another point to take into account is the number of intersections
% yielded in the team filter from the direction rays.
% Compared to 
% more robots that fail -> less intersections
% show number of intersections for each file compared to method
% The rest of algorithm as presented in 03 phase


\subsection{Conclusion}
\label{subsubsec:04_teamConclusion}

The \ac{SSL} algorithm is tested with all \ac{TDOA} methods
in regard to the laboratory-measurements.
\Cref{tab:04_teamMethodComparison} summarizes the results briefly by
the absolute distance \ac{RMSE} of all measurements.

\btline{H}{1.2}
\btab{|c|c|}
\hline
Method & Abs. Distance \ac{RMSE} [\si{\meter}]\\
\hline
\ac{CC} & 1.45\\
\hline
\ac{GCC} & 0.87\\
\hline
Phase Difference & 1.33\\
\hline
\etab
\et{Summarized performance of the multi-agent \ac{SSL} according to the \ac{TDOA}
methods}{04_teamMethodComparison}

Comparing the \ac{SSL} results with three different \ac{TDOA} methods,
the \ac{GCC-PHAT} algorithm performs best.
As expected, the \ac{CC} method yields poorer results what
underlines the statements about the \ac{CC} method at the beginning
of this work in \Cref{sec:02_cc}.
Assessing the \ac{CC} and phase difference results of the \ac{SSL},
the resulting outcomes in \cref{tab:04_teamMethodComparison}
could lead to the conclusion that both are equally valid.

Recollecting the \ac{WSDE} results in \cref{fig:04_compareRmse}
one sees that the single \ac{WSDE} results of the \ac{CC} are more
precise regarding the error and standard deviation.
Small deviation means in this case that the individual robots agree
on the direction roughly. The more outliers exist in the measurement,
the more does the deviation increase.
Having all \ac{WSDE} results of five robots available for the \ac{SSL},
the outliers can be neglected by good filtering.
However, with less number of robots included in the multi-agent \ac{SSL}
the accuracy of each single \ac{WSDE} becomes more important.
With this in mind, the results of the simple \ac{CC} approach is more
reliable than the phase difference method.

Finally with the given results from the laboratory-measurements, one can state that
the \acf{GCC-PHAT} algorithm yields the most
accurate results for the \acf{WSDE} outcome of individual robots.
Through the high reliability of the single results with this approach, the Bayesian updating
filter of the multi-agent system produces appropriate whistle-source position estimates that
differ less than 1\si{\meter} from the real source in average.

% Further information about the source of the error can be obtained
% by looking at the single robot results of each measurement.
% Since the team filter is updated computing the intersections of the
% individual robot's direction estimates, the accuracy of the absolute position
% depends on the number of arising intersections.\todo[inline]{It is not immediately
% clear whether "more intersections" is better or worse.}
% This will be discussed in \cref{subsec:04_singleRobotAngleError}.
% % -------------------------------------------------------------


\todo[inline]{General, use pont instead of comma for numerical values}