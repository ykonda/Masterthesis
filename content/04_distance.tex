\subsection{Distance Approximation}
\label{subsec:04_distance}

\Cref{sec:03_distance} clarified the conditions to identify the distance of
the sound source by one robot.
To examine this, measurements from the front and back of the robot were
collected and evaluated.
\Cref{tab:04_distance} lists the distances and their results by the
all methods.
For the two measurements with zero distance, the whistle was blown
once from the front with opposite orientation than the robot (180\si{\degree}) and
once from directly behind the robot with the same orientation as the robot
(0\si{\degree}).
The distance is represented in robot coordinates, so that a source with positive
distance was placed in front of the robot and oriented towards the robot and
vise versa.

% -------------------------------------------------------------
\btline{ht}{1.2}
\btab{|c|c|c|c|c|c|}
\hline
Nr. & Real Distance [\si{\meter}] & GCC Result [\si{\meter}] & CC Result [\si{\meter}] & Phase Result [\si{\meter}]\\
\hline
1 & 0,9 & $\infty$  & $\infty$ & $\infty$ \\
\hline
2 & 0,6 & $\infty$ & $\infty$ & $\infty$ \\
\hline
3 & 0,3 & 0,35 & 0,25 &  0,13 \\
\hline
4 & 0,0 (180\si{\degree}) & -0,13 & -0,15 & -0,23 \\
\hline
5 & -0,0 (0\si{\degree}) & 0,22 & 0,21 & 0,02 \\
\hline
6 & -0,3 & -0,15 & -0,27 & -0,45 \\
\hline
7 & -0,6 & -0.34 & -0,50 & -0,62 \\
\hline
8 & -0,9 & -0.70 & -0,91 & -0,99 \\
\hline
9 & -1,2 & -1,00 & -1,28 & -1,71 \\
\hline
10 & -1,5 & -1,39 & -1,59 & -2,98 \\
\hline
11 & -1,8 & -1,72 & -2,07 & -3,33 \\
\hline
12 & -2,1 & -2,16 & $\infty$ & -3,02\\
\hline
13 & -2,4 & -2,31 & $\infty$ & $\infty$ \\
\hline
14 & -3,75 & -3,66 & -9,51 & -4,15 \\
\hline
15 & -6,4 & -7,35 & -7,27 & $\infty$  \\
\hline
16 & -9,8 & $\infty$ & $\infty$ & $\infty$ \\
\hline
\etab
\et{Result of front and rear distance for all methods.}{04_distance}
% -------------------------------------------------------------
As the results show, the distance can be determined with small error.
One can see that the \ac{GCC} results are erroneous for small distances,
but gives a correct approximation steadily.
Compared to this, the \ac{CC} method performs better for small distances,
but fails completely for some measurements.
Those incorrect cases without distance result arose due to the computed
lateral delays exceeding the maximum lateral samples.
The phase methods provides most incorrect results with measurement 10 being
double the real value.

For all measurements, the resulting direction angle of the sound source were
accurate.
Rightfully, the algorithm detects sources that are out of observable range.

Unfortunately, for real cases one can not rely on the height parameter of the sound
source which varies from the referee's body height.
Having this as approximation only, the distance result should be handled
with care.