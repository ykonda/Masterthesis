\section{Whistle Signal}
\label{sec:02_whistleSignal}

In this work the localization of a whistle sound source is to be to the fore.
Detection of the whistle is done in frequency domain by assuming the whistle
sound to be higher than 2\si{\kilo\hertz} and lower than 4\si{\kilo\hertz}.
By comparing the mean of the signal between this band with the overall mean of the
received signal, a peak arising around the whistle frequency can be detected.
For the whistle detection, only one channel of the robot is used and the
mean of the whistle band must exceed the threshold multiple cycles in a row.
If the team takes action due to the detected signal on individual robots is a team decision.

Further on for this work, the mathematical model of a received whistle
signal at one microphone sensor is defined as
\be
x_i(t) = s_i(t) + n_i(t) \text{ for } i \in \{0, 1, 2, 3\}
\label{eq:02_whistleSignal}
\ee
where $s(t)$ represents the signal and $n(t)$ noise.
Both are assumed as real, jointly stationary random processes.