\section{SNR and PSNR}
\label{sec:03_snr}

The \acf{SNR} is a common value to express the signal power $P_{signal}$ compared
to power of the background noise $P_{noise}$.
Conveniently, the buffered audio signal in this work always consists of
a clean-cut delimitation between signal and noise which is set by the
start index.
Thus, the \ac{SNR} which is defined as
\bal
    SNR_{db} = 10\log_{10}(\frac{P_{signal} - P_{noise}}{P_{noise}})
    \label{eq:03_snr}
\eal
in decibels can be implemented straightforwardly.
Informational content about this measure is investigated in \cref{subsec:04_snr}.
Expectations are that the \ac{SNR} can be feed into the covariance matrix
of an incoming result in the Bayesian update process.

In image processing, the \acf{PSNR} indicates the quality of a compressed
image. Here in this work, the ratio between the peak of a signal
and its noise is related to the \ac{GCC-PHAT} outcome and called \ac{PSNR}
henceforth.
As stated in \cref{sec:02_gcc}, the most significant characteristic
of the \ac{GCC-PHAT} is the resulting sharp peak which now can be
assessed with one value
\bal
    PSNR_{db} = 10\log_{10}(\frac{P_{peak}}{P_{noise}}).
    \label{eq:03_psnr}
\eal
From the implementation view, the power of the correlation peak is divided by
the power of the remaining correlation signal.
It has to be noted that two adjacent values prior and after the peak
are disregarded as they might belong to the peak.
A validation if and how much the \ac{PSNR} and the accuracy of the \ac{GCC}
delay result are linked is done in \cref{subsec:04_psnr}.