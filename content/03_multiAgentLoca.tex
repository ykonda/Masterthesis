\section{Multi-Agent Source Localization}
\label{sec:03_multiAgentLoca}

The final whistle source localization is realized by collecting the
all \ac{WSDE} results of all active robots and combining the directions
into one xy-position outcome.
Two separate implementation details are covered in the following.
First, the actual localization algorithm that concludes the source localization
task is set in focus.
How the process of the multi-agent decision proceeds including the communication
is dealt within \cref{subsec:03_teamCommunication}.
% -------------------------------------------------------------

As stated, the actual Bayesian updating filter algorithm is addressed first
by paying attention on \cref{alg:03_multiAgentLoca}.
With the specification of the filter made in \cref{subsec:02_2dTeam}, the
whistle source localization is realized when the results of the individual robots
are available.
The multi-agent localization algorithm receives an array of rays that represent
the \ac{WSDE} results in the field coordinate system.
The whistle sound position to return consists of the x- and y-position $\vec{\mu}$ and
a covariance matrix \textbit{C}.
For each combination of the rays $R[j]$ and $R[k]$, the intersecting point position
$\vec{\mu_{jk}}$ along with its covariance matrix $\textbit{C}_{jk}$ is examined.
If such intersection exists, the whistle source position information is updated until
until all intersections were taken into consideration.

% PSEUDO CODE UPDATING ALGORITHM
\begin{algorithm}[H]
    \caption{Bayesian Updating}\label{alg:03_multiAgentLoca}
    \begin{algorithmic}[1]
        \Procedure {WhistleLocalization}{$array<R>$}
            \For{$j \in 0:R.size()$}
                \For{$k \in j+1:R.size()$}
                    \State $\vec{\mu}_{jk} \gets \Call{FindIntersection}{R[j], R[k]}$
                    \State $\textbf{\textit{C}}_{jk} \gets \Call{CalulateCovariance}{R[j], R[k]}$
                    \If{first intersection found}
                        \State $\vec{\mu} \gets \vec{\mu}_{jk}$
                        \State $\textbf{\textit{C}} \gets \textbf{\textit{C}}_{jk}$
                    \ElsIf{intersection found}
                        \State $\textbf{\textit{S}}_{jk} \gets \textbf{\textit{C}} + \textbf{\textit{C}}_{jk}$
                        \State $\textbf{\textit{K}}_{jk} \gets \textbf{\textit{C}} \cdot \textbf{\textit{S}}_{jk}^{-1}$
                        \State $\vec{\mu} \gets \vec{\mu} + \textbf{\textit{K}}_{jk} \cdot (\vec{\mu}_{jk} - \vec{\mu})$
                        \State $\textbf{\textit{C}} \gets \textbf{\textit{C}} - \textbf{\textit{K}}_{jk} \cdot \textbf{\textit{C}}$
                    \EndIf
                \EndFor
            \EndFor
            \State \textbf{return} $\vec{\mu}$
        \EndProcedure\vspace{12pt}
    \end{algorithmic}
\end{algorithm}
% -------------------------------------------------------------

\subsection{Team Communication}
\label{subsec:03_teamCommunication}

To agree on a whistle position as multi-agent system by computing the whistle
localization algorithm \cref{alg:03_multiAgentLoca}, the \ac{WSDE} results of the
stand-alone robots must be collected.
As introduced in \cref{sec:03_naoFramework}, the robots are able to send and receive
information from team mates wirelessly.
By present implementation, the team message includes xy-position information as well as
the orientation of all connected robots.
Thus, only the angular \ac{WSDE} and its additional information (distance, \ac{PSNR}) are to be
appended to the team message.

