\section{Multi-Agent Localization}
\label{sec:03_multiAgentLoca}

The final whistle sound source localization is realized by collecting the
all \ac{WSDE} results of all active robots and combining the directions
into one xy-position outcome.
Two separate implementation details are covered in the following.
First, the actual localization algorithm that concludes the source localization
task is set in focus.
How the process of the multi-agent decision proceeds including the communication
is dealt within \cref{subsec:03_teamCommunication}.

% -------------------------------------------------------------

\begin{figure}[ht]
	\centering
		\includegraphics[width=0.50\textwidth]{figures/rays}
    \caption[Nomenclature for multi-agent localization algorithm]
            {Nomenclature for multi-agent localization algorithm.}
    \label{fig:03_rays}
\end{figure}
% -------------------------------------------------------------


\begin{algorithm}[H]
    \caption{Multi-Agent Localizatiion}\label{alg:multiAgentLoca}
    \begin{algorithmic}[1]
        \Procedure {WhistleLocalization}{$array<R>$}
            \For{$j \in 0:R.size()$}
                \For{$k \in j+1:R.size()$}
                    \State $\vec{i} \gets \Call{FindIntersection}{R[j], R[k]}$
                    \State $\textbf{\textit{C}} \gets \Call{CalulateCovariance}{R[j], R[k]}$
                    \If{first intersection found}
                        \State $\vec{\mu} \gets \vec{i}$
                        \State $\textbf{\textit{Q}} \gets \textbf{\textit{C}}$
                    \ElsIf{intersection found}
                        \State $\textbf{\textit{S}} \gets \textbf{\textit{Q}} + \textbf{\textit{C}}$
                        \State $\textbf{\textit{K}} \gets \textbf{\textit{Q}} \cdot \textbf{\textit{S}}^{-1}$
                        \State $\vec{\mu} \gets \vec{\mu} + \textbf{\textit{K}} \cdot (\vec{i} - \vec{\mu})$
                        \State $\textbf{\textit{Q}} \gets \textbf{\textit{Q}} - \textbf{\textit{K}} \cdot \textbf{\textit{Q}}$
                    \EndIf
                \EndFor
            \EndFor
            \State \textbf{return} $\vec{\mu}$
        \EndProcedure\vspace{12pt}
    \end{algorithmic}
\end{algorithm}


$\Phi_j = \theta_j + \gamma_j$
\bal
\vec{r_j} &= \vec{p_j} + \vec{d_j} % = \begin{pmatrix}p_{jx}\\p_{jy}\end{pmatrix} + \ell \begin{pmatrix}d_{jx}\\d_{jy}\end{pmatrix}
    = \begin{pmatrix}p_{jx}\\p_{jy}\end{pmatrix} + \ell \begin{pmatrix}\cos(\Phi)\\\sin(\Phi)\end{pmatrix}
\eal

\bal
\vec{i} &= \begin{pmatrix}i_{x}\\i_{y}\end{pmatrix}\\
\vec{i} &= \vec{p_1} + u \cdot \vec{d_1} = \vec{p_2} + v \cdot \vec{d_2}\\
u &= \frac{p_{1y} \cdot d_{2x} + d_{2y} \cdot p_{2x} - p_{2y} \cdot d_{2x} - d_{2y} \cdot p_{1x}}
         {d_{1x} \cdot d_{2y} - d_{1y} \cdot d_{2x}}\\
v &= \frac{p_{1x} + d_{1x} \cdot u - p_{2x}}{d_{2x}}\\
\eal

\bal
\frac{ix}{\delta \Phi_1} = \frac{(\cos(\Phi_2) \cdot (\cos(\Phi_1)^2 + \sin(\Phi_1)^2) \cdot ((p_{1y} - p_{2y}) \cdot \cos(\Phi_2) +
                                  (-p_{1x} + p_{2x}) \cdot \sin(\Phi_2)))}
                                  {(-\cos(\Phi_2) \cdot \sin(\Phi_1) + \cos(\Phi_1) \cdot \sin(\Phi_2))^2}
\eal

To agree on a whistle position as multi-agent system, the detected sound source
direction is added into the team message as float.
Depending on the implementation, additional information like distance or
\ac{PSNR} can be appended.

For the evaluation of the methods, the team decision is implemented in
python for offline analysis of the sample data.
Therefore, the result is computed for each robot and introduced into
the team decision filter as direction ray. %with covariance and distance
% is front or rear
Each value input is paired with the other rays. If they cross, the
filter is updated with this intersection until all pairs of direction
rays are being considered.
\missing[]{More info}

For the real case, a team whistle localization module is
introduced into the HULKs' framework.
Here, each robots fills the result of the determined direction into
the team message.
\missing[inline]{How this works.}

\subsection{Team Communication}
\label{subsec:03_teamCommunication}

