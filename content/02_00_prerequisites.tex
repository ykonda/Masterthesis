\chapter{Prerequisites}
\label{chap:02_prerequisites}

For the whistle sound source localization with multiple robots, some
sequential steps needs to interact for a final result.
To work with the implementations in \cref{chap:03_implementation}, the fundamentals
are introduced in this chapter on a general basis.

As this work specializes on the localization of a whistle sound, this sound pattern
must be detected at first.
This was already done by previous work \cite{Hasselbring}.
Hence, the flow of the whistle detection are briefly explained in \cref{sec:02_whistleSignal}.
The position of the sound source is determined by combining the
separate direction results on the single robots.
Utilizing the \ac{TDOA} information as in \cref{sec:02_tdoa} between the pairs
of the Naos' microphones, every robot produces a sound direction ray which
are feed into the team decision filter.
As mentioned in \cref{chap:01_introduction}, different methods exist to identify the \ac{TDOA}
and are terms of content in \cref{sec:02_cc,sec:02_gcc,sec:02_phase}.
Due to the low resolution arising from the sample rate and the distance between the microphones,
a subsample estimation is used based on \cref{sec:02_subsampleShift}.
One of the most significant factors is the selection of the signal frame that is used for
the direction determination.
In order to provide a correct and stable decision process, the different approaches to
consider are shortly mentioned in \cref{sec:02_signalStartDetection}.
After all, the results of the individual robots are filtered by assuming
gaussian distribution to produce a sound source position.
By passing information about the certainty of the single robot result, the
filtering explained in \cref{sec:02_filter} is related to a Kalman filter without
prediction step.

% \missing[]{
% - CC in time domain, because low frequency resolution (44100Hz/512samples=resolution)\\
% and also CC corrupted and we want to detect which signal was first\\
% }

\section{Whistle Signal}
\label{sec:02_whistleSignal}

In this work the localization of a whistle sound source is to be to the fore.
Detection of the whistle is done in frequency domain by assuming the whistle
sound to be higher than 2000Hz and lower than 4000Hz.
By comparing the mean of the signal between this band with the overall mean of the
received signal, a peak arising around the whistle frequency can be detected.
For the whistle detection, only one channel of the robot is used and the
mean of the whistle band must exceed the threshold multiple cycles in a row.
If the team takes action due to the detected signal on individual robots is a team decision.

Further on for this work, the mathematical model of a received whistle
signal at one of the four microphone sensor is defined as
\be
x_i(t) = s_i(t) + n_i(t) \text{ for } i \in \{0, 1, 2, 3\}
\label{eq:02_whistleSignal}
\ee
where $s(t)$ represents the signal and $n(t)$ noise.
Both are assumed as real, jointly stationary random processes.
\section{Time Difference Of Arrival}

The direction of a signal source can be computed by the relative time delay between
two microphones.
- describe how the direction is defined by TDOA
It is assumed that the distance from the sensors to the sound source is
significantly large so that the signal waves proceed parallel.
Then, calculations for the direction of the sound source can be done with a simple
geometrical approach like done in \cite{Valin_Michaud}.

With the definition of a whistle signal as stated in \ref{eq:02_whistleSignal},
the microphone sensors $mic_1$ and $mic_2$ will output
\bsub \bal
    x_1(t) &= s(t) + n_1(t)\\
    x_2(t) &= \alpha s(t - D) + n_2(t).
\eal \esub
\label{eq:02_signalTimeDomain}
Here, $D$ is the delay of $x_2$ relative to $x_1$ for which is looked for.
\section{Cross Correlation}
\label{sec:cc}

The \ac{CC} provides information about the similarity of two signals.
Thus, the delay of one signal can be detected where the \ac{CC} $r_{12}(t)$ is largest.
In time domain, the \ac{CC} of two signals $x_1$ and $x_2$ is denoted as
\bal
    R_{x_1x_2}(t) = x_1(t) \circledast x_2(t) = \int^{+\infty}_{-\infty}x_1(\tau-t)x_2(\tau)d\tau.
\eal
Considering the frequency domain, the function can be transformed into
\bal
    \mathcal{F}[R_{12}(t)] = G_{x_1x_2}(f) = X_1^*(f)X_2(f)
\eal
with $\mathcal{F}[x_i(t)] = X_i(f)$ and $X_i^*(f)$ indicating the conjugate complex form.
However, the finite observation time of the received signal corrupts the fourier
transform \cite{K_C_GCC}
and noise of sensors may introduce false peaks in the \ac{CC} \cite{H_B_GCC}.
In frequency domain, the signals $x_1(t)$ and $x_2(t)$ from \ref{eq:02_signalTimeDomain}
can be expressed as
\bsub \bal
    X_1(f) &= S(f) + N_1(f)\\
    X_2(f) &= \alpha S(f) e^{-j2\pi fD}+ N_2(f).
\eal \esub
\label{eq:02_signalFreqDomain}
Thus, the \ac{CC} is
\unsure[]{better formatting?}
\bsub \bal
    G_{x_1x_2}(f) &= \alpha |S(f)|^2 e^{-j2\pi fD} + N_1^*(f)N_2(f) + S^*(f) N_2(f) + \alpha S(f) e^{-j2\pi fD}N_1^*(f)\\
\intertext{which will be shortened as}
    G_{x_1x_2}(f) &= \alpha \phi_s(f) e^{-j2\pi fD} + \phi_n(f) + \phi_c(f) \label{eq_02_Gx1x2_simple} \\
\intertext{where}
\phi_s(f) &= |S(f)|^2 \label{eq:02_phi_s} \\
\phi_n(f) &= N_1^*(f)N_2(f) \label{eq:02_phi_n1n2} \\
\phi_c(f) &= S^*(f) N_2(f)+\alpha S(f)e^{-j2\pi fD}N_1^*(f) \label{eq:02_phi_c}.
\eal \esub
\label{eq:02_Gx1x2}
%\cite{H_B_GCC}
Considering the ideal case where $s(t)$, $n_1(t)$ and $n_2(t)$ are uncorrelated, the terms
$\phi_c$ and $\phi_n$ disappear and the \ac{CC} results in
\bal
    R_{12}(t) = \mathcal{F}^{-1}[\alpha \phi_s(f) e^{-j2\pi fD}] = \alpha \mathcal{F}^{-1}[\phi_s(f)] \circledast \delta(t-D).
\eal
\label{eq_02_R12_noNoise}
\unsure[]{do I fully understand this? Is this correct?}
This means there exists a peak at delay $D$ which is altered by the \ac{iFT}
of the signal spectrum.
In general, $\phi_c$ and $\phi_n$ can neither be neglected nor assumed as uncorrelated to the signal \cite{H_B_prob},
so that they introduce inaccuracies and errors.

\section{Generalized Cross Correlation}

- improve normal cross correlation\\
- whistle frequency between 2000Hz and 4000Hz (low-pass)\\
- thus, the peaks of the cross-correlation are wide\\
- to avoid this circumstance, the generalized cross correlation (GCC)
was introduced by Knapp and Carter 1976 \cite{Knapp_Carter_GCC} \\
- weightings before iFFT to maximize expected signal peak relative
to output noise \cite{Hassab_Boucher_GCC}


- 
"The filters have a function to ensure a large sharp peak in the obtained cross-versions of the two received signals. The filters have a function to ensure a large sharp peak in the obtained cross- correlation thus ensuring a high time delay resolution in
function to ensure a large sharp peak in the obtained cross- correlation thus ensuring a high time delay resolution in generalized cross-correlation method.
correlation thus ensuring a high time delay resolution in generalized cross-correlation method." - 
\section{Signal Phase Difference}
\label{sec:02_phase}

With a different approach to the correlation methods, the \ac{TDOA} can be
detected by observing the phase of one certain frequency $f_c$.
The phase of a signal frequency is easily computable in frequency domain
with
\bal
    \phi(f_c) &= tan^{-1}(\frac{imag(X(f_c))}{real(X(f_c))}).
\eal
With the difference of the phases of two channel, the delay in meters is defined as
\bal
    D &= \frac{\Delta \phi \cdot c_s}{2 \pi \cdot f_c}.
\eal
From that, the direction angle calculation of \cref{eq:02_tdoaAngle} can
be followed.
\section{Subsample Shift}
\label{sec:02_subsampleShift}

\missing[inline]{Text, Formula, Figure explanation}
Considering the case that the sample frequency $f_s$ is set to 44100\si{Hz}
and the sound speed is 343\si{m/s}, the maximal number of samples
between the rear channels is 14.
This leads to a very low resolution of the direction angle.
Quadratic interpolation is a well known technique to obtain a floating number
shift from a \ac{CC}.
For this, a parabola $y(x) = a(x-p)^2+b$ is fitted into the three values of $R$ around the peak
of the \ac{CC} and the peak of the parabola is assumed as correct, more accurate
delay.
\bal
	D_{sub} = \frac{\alpha - \gamma}{2 \cdot (\alpha - 2\beta + \gamma)}
	\label{eq:02_subsample}
\eal
% -------------------------------------------------------------
\begin{figure}[ht]
	\centering
		\includegraphics[]{figures/subsample_shift}
	\caption{Explanation example of the subsample shift estimation using parabolic interpolation.}
    \label{fig:02_subsampleShift}
\end{figure}
% -------------------------------------------------------------
\section{Signal Start Detection}
\label{sec:02_signalStartDetection}

One focus of the whistle signal localization is the correct choice of the
signal frame, with which the \ac{TDOA} calculation is done.
Assuming that the clearest signal without reverberation and with minimal
multipath propagated samples is at the start of a sound signal,
the frame to examine is chosen to be at the beginning of a whistle-sound.

By knowing the frequency band of a whistle signal, the start can be detected
where these frequencies dominate.
Using this indicator only does not always give the desired accuracy, that is why
different methods are investigated in this work \cite{tdoa_sspd}.
In the next subsections, signal start detection using short time energy,
zero crossing rate and spectral entropy are subject of discussion.
Also, the methods require
various computational power. According to the circumstances, the most suitable
approach can be chosen.
Another point is, that robustness can be increased by considering these methods
in combination.
As a latter, the consensus of the single methods can be passed as information
about the certainty of the computed direction result.

\subsection{Short Time Energy and Zero Crossing Rate}

A common method in signal start and endpoint detection is the evaluation
of the short time energy and \ac{ZCR}.

\subsubsection{Short Time Energy}
The energy
\bal
    E &= \sum_{n=1}^N E_s(n)
    \label{eq:02_energy}
 \intertext{with the energy spectral density}
    E_s(n) &= |x(n)|^2
    \label{eq:02_spectralEnergy}
\eal
of signal frames with length $N$ are expected to be higher than noise frames
and therefore, noise and signal can be distinguished
according to \cite{Z_W_voiceActivity}.
A threshold needs to be specified appropriately dependent on the environment.

\subsubsection{Zero Crossing Rate}

The \ac{ZCR} of one frame $Z$ needs small computational effort in order to
identify a periodic signal in time domain.
Its formula is
\bal
    Z = \sum_{n=2}^N |sgn(x(n)) - sgn(x(n-1))|
    \label{eq:02_zcr}
\eal
with the sign function
\begin{align*}
    sign(x) =
    \begin{cases}
        1 & \quad \text{if } x\geq 0 \\
        -1 & \quad \text{if } x < 0
    \end{cases}
\end{align*}
for a discrete signal $x(k)$ of a frame with length $N$ \cite{Z_W_voiceActivity}.
A higher \ac{ZCR} is an indication for a periodic signal.
To detect the signal start, a threshold is determined dynamically.
The \ac{ZCR} mean of frames which are known to be noise only
are averaged with the mean of those frames, that include the whistle signal.
The signal start is detected at the point in time, where the \ac{ZCR} exceeds
this threshold.

\subsection{Spectral Entropy}
\label{subsec:02_Entropy}

Entropy provides information about the disorder of a system.
From this, one can derive that noise has a high entropy compared to
a whistle-sound, which is a highly structured sound signal and a high
amount of information accordingly.
The spectral entropy of a signal is determined by normalizing the
\ac{PDF} over all frequency components as described in \cite{S_J_entropy}.
When $X(n)$ is the \ac{DFT} of the sampled signal $x(n)$, the \ac{PDF} is
\bal
    P(n) &= \frac{E_s(n)}{E}
    \label{eq:02_pdf}
\eal
with \cref{eq:02_spectralEnergy} as the spectral energy density function for $E_s(n)$
and $E$ as the energy.
Finally, the spectral entropy results in
\bal
    H &= - \sum_{n=1}^N P(n) log_2P(n).
    \label{eq:02_entropy}
\eal
Utilizing some a-priori knowledge about the signal, the entropy estimation
can be improved.
In this work, the frequency of a whistle-sound is known to be
between $2\si{\kilo\hertz}$ and $4\si{\kilo\hertz}$ from \cite{Hasselbring},
Thus, only the frequency components in the whistle range is considered.
% \unsure[]{weighting factors as described in S-J-entropy?}
Differentiating between noise samples where no signal is present and
signal frames, a dynamic threshold can be set to detect the signal start point.
\section{Spectral Subtraction}
\label{sec:02_spectralSubtraction}

\section{Team Filter}
\label{sec:02_filter}
