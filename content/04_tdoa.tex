\section{\acl{WSDE}}
\label{sec:04_tdoaSingle}

Before evaluating the performance of the different methods as a whole,
% For comparability of the results,
one exemplary measurement is utilized
to present and analyse the \ac{TDOA} methods in detail first.
In this recording, the sound source is placed at the right front
of the robot with 4.5\si{m} distance.
This corresponds to an an angle of -33.7\si{\degree} in robot coordinates.
To get an idea about the examined data, according whistle signal samples around the
start are plotted in \cref{fig:04_tdoaSignal} for all channels.
% -------------------------------------------------------------
\begin{figure}[ht]
	\centering
		\includegraphics[]{figures/evaluation/cc_frontRight_1_signal}
	\caption{Signal start section of a whistle sound recorded from front right.}
	\label{fig:04_tdoaSignal}
\end{figure}
% -------------------------------------------------------------

As the next sections focus on the performance of the \ac{TDOA} methods,
the start index is set manually.

For the sake of conciseness, throughout the following sections the correlation
function $R_{x_ax_b}$ of two signals $x_a$ and $x_b$  (\cf
\cref{chap:02_prerequisites}) is denoted as $R_{ab}$.


\subsection{Cross Correlation}
\label{subsec:04_ccSingle}
% -------------------------------------------------------------

To visualize the result of the \ac{CC}, the correlations are plotted in
\cref{fig:04_cc}. For $R_{32}$ and $R_{13}$ a peak is clearly visible.
However, for the other \ac{CC} the problem of a weak peak
arises what was mentioned as downside of the \ac{CC} in \cref{sec:02_cc}.
% -------------------------------------------------------------
\begin{figure}[ht]
	\centering
		\includegraphics[]{figures/evaluation/cc_frontRight_1}
	\caption{Cross correlation results of signal from front right (-33,7\si{\degree}).}
	\label{fig:04_cc}
\end{figure}
% -------------------------------------------------------------
\btline{ht}{1.2}
\btab{|c|c|c|c|c|}
\hline
Base Channel & Next Channel & Delay & Candidate (-) & Candidate (+)\\
\hline
0 & 1 & -8.25 & -144.9 & -35.1\\
\hline
1 & 3 & -4.59 & -17.4 & 78.6\\
\hline
2 & 0 & 9.16 & -30.6 & -30.6\\
\hline
3 & 2 & 3.94 & -150.2 & -29.8\\
\hline
\etab
\et{Cross correlation delay results of signal from front right}{04_cc}
% -------------------------------------------------------------

According to the delays in \cref{tab:04_cc}, two source direction candidates arise
for each channel pair.
By the implementation in \cref{subsec:03_directionCandidates}, the combination of
all options with the smallest error is selected as \ac{WSDE}.
Hence, the algorithm outputs -26.9\si{\degree} what produces an error of 6.8\si{\degree}.
The delay between channel 2 and 0 is larger than the maximum delay of 6.85 samples
and therefore cut to the maximum sample delay.
Besides these, the \ac{TDOA} between the channel pairs produce one appropriate
direction candidate which correctly points to the sound source.
% -------------------------------------------------------------

\subsection{Generalized Cross Correlation}
\label{subsec:04_gccSingle}
% -------------------------------------------------------------
\Cref{fig:04_gcc} presents the cross correlation result by the \ac{GCC} method of
the same signal data as in the previous section.
The subsample delays for each channel pair and their resulting direction candidates
are listed in \cref{tab:04_gcc}.
From this, a final direction of -30.0\si{\degree} is determined
resulting in an error of 3,69\si{\degree}.
It is apparent that the peaks of the \ac{GCC} are better to detect than the peaks of the
\ac{CC}.
% -------------------------------------------------------------
\begin{figure}[ht]
	\centering
		\includegraphics[]{figures/evaluation/gcc_frontRight}
	\caption{Generalized cross correlation results of signal from front right.}
	\label{fig:04_gcc}
\end{figure}
% -------------------------------------------------------------
\btline{ht}{1.2}
\btab{|c|c|c|c|c|}
\hline
Base Channel & Next Channel & Delay & Candidate (-) & Candidate (+)\\
\hline
0 & 1 & -8,28 & -144,7 & -35,3\\
\hline
1 & 3 & -4,09 & -22,8 & 84,0\\
\hline
2 & 0 & 7,60 & -30.6 & -30.6\\
\hline
3 & 2 & 4,13 & -148,7 & -31,3\\
\hline
\etab
\et{Generalized cross correlation delay results of signal from front right}{04_gcc}
% -------------------------------------------------------------
\subsection{Phase Difference}
\label{subsec:04_phaseSingle}

For detecting the source direction with phase difference, a smaller frame
size of 64 samples is defined.
In \cref{subsubsec:03_phase} two variants of this method were introduced that use
different strategies to identify a reference frequency. The first version uses
a static reference frequency that is fixed a-priori by the user. The second version
dynamically estimates a dominant frequency across all four channels. Hereafter,
the performance of both variants is discussed.


\subsubsection*{Static Reference Frequency}

In \cref{subsubsec:03_phase} two different ways to set a reference frequency $f_c$
for the phase difference method were introduced.
First, a suitable value for the reference frequency is specified by
investigating the influence of the chosen value.
Therefore, \ac{WSDE} results with the phase difference method are evaluated by setting
different values for the reference frequency within whistle range.
For this purpose we consider all of eleven measurements of the the laboratory-dataset
recorded with robot nr. 26 at the center point.

According to \cref{subsubsec:03_phase}, the maximum feasible frequency is defined as
2775.08\si{\hertz} by the distance between the channels
and the whistle spectrum ranges between 2\si{\kilo\hertz}
and 4\si{\kilo\hertz}.

As shown in \cref{fig:04_diffFc} shows, the \ac{RMSE} is high for
frequencies smaller than 2600\si{\hertz}.
With a frequency of 2024.12\si{\hertz}, error is largest.
% The result complies with the information in \cref{fig:03_maxFreq} showing that
% frequencies higher than 2500\si{\hertz} are dominant in whistle signals.
With this outcome, the fixed frequency is set to 2670.1\si{\hertz}
for further usage of the direction detection by phase method.
Limitation exists due to the ambiguity of the signal which is
content of \cref{subsubsec:03_phase}.
% -------------------------------------------------------------
\begin{figure}[ht]
	\centering
		\includegraphics[]{figures/evaluation/phase_fc_rmse}
	\caption{Result of all measurements done with robot 26 to compare different
	fixed frequency values in whistle range.}
	\label{fig:04_diffFc}
\end{figure}
% -------------------------------------------------------------

On the basis of the results, the reference frequency was set to a minimum
of 2600\si{\hertz}.
Hence, the reference frequency is 2627.1\si{\hertz} when the a \ac{FFT} length is
set to of 256.
Applying the phase difference method for this reference frequency, the final
direction estimate computed from the candidates listed in
\cref{tab:04_fixedFreqResult} is -29.6\si{\degree} which results in an error of
4.1\si{\degree}.
% -------------------------------------------------------------
\btline{ht}{1.2}
\btab{|c|c|c|c|c|}
\hline
Base Channel & Next Channel & Phase Difference & Candidate (-) & Candidate (+)\\
& & [\si{\deg}] & [\si{\deg}] & [\si{\deg}] \\
\hline
1 & 3 & -79.1 & -26.8 & 88.0\\
\hline
2 & 0 & 167.7 & -30.6 & -30.6\\
\hline
3 & 2 & 88.5 & -148.7 & -31.3\\
\hline
\etab
\et{Resulting candidates of phase difference method with fixed frequency
	2670.1Hz of example measurement from front right
	(-33.7\si{\degree})}{04_fixedFreqResult}
% -------------------------------------------------------------

% Configuration for Phase Method
\subsubsection*{Frame Number}
\label{subsubsec:04_frameNumber}

Not only does the frequency play a major role for the phase method,
but also the samples chosen for analysis.
To evaluate if and how the result changes over time, the frame to
utilize is shifted by half the frame size for all measurements
of \cref{subsec:04_labMeasurements} for the robot at the center
point.
% [ 29.08029745  58.1426789   65.27548831  67.91984464  76.82890558
%   97.33976251  96.63736674 100.2203543  105.14752253  61.28179302
%   60.0017076   50.60063622  53.80158271  49.0830362   62.64453085]
In \cref{fig:04_phaseOverTime} one sees that the first channel frames with zero
shift gives the best result with a \ac{RMSE} of 29.1\si{\degree}. These results
show that the frame position has a significant influence on the prediction
accuracy of the direction detection. Therefore, an accurate signal start
detection is crucial for precise sound source localization.
% -------------------------------------------------------------
\begin{figure}[ht]
	\centering
		\includegraphics[]{figures/evaluation/phase_over_time}
	\caption{Changing results by shifting the frame over the
	samples. All measurements of \cref{subsec:04_labMeasurements}
	with robot nr. 26 are taken into consideration.}
	\label{fig:04_phaseOverTime}
\end{figure}
% -------------------------------------------------------------

\subsubsection*{Dynamic Reference Frequency Selection}

First, the result of the phase difference method using a  dynamically selected
frequency are presented. As stated in the implementation chapter, the frame is
chosen where the frequencies of the maximal\todo{Du schreibst haeufiger
"maximal". Laut dictCC gibt es das aber es ist glaube ich eher unueblich}
amplitudes coincides for all channels. For the running example discussed here
this corresponds to a frequency of 2756,25\si{\hertz}.
In the upper plot of \cref{fig:04_phaseSingle}
\todo{Die figure ist ganz schoen
weit weg (ich musste ein bisschen suchen)} one sees the received microphone
data which will be Hann windowed and then transformed into frequency domain by
\ac{FFT}. The resulting phases and amplitudes are listed in
\cref{tab:04_phaseSingle}.
% -------------------------------------------------------------
\begin{figure}[H]
	\centering
		\includegraphics[]{figures/evaluation/phase_cos}
	\caption{Frames used for the direction detection by phase method.}
	\label{fig:04_phaseSingle}
\end{figure}
% -------------------------------------------------------------
For comprehensibility, the determined frequency information visualized by
wave signals with the detected phases and amplitudes
in the lower subplot of \cref{fig:04_phaseSingle}.
Due to the larger distance between channels 0 and 1 \todo{distance in welcher
Metric? Physical distance of the microphones? Kannst du vielleicht noch einen
satz zur klarstellung dazu schreiben.}, the phase difference information must
be neglected because the phase difference is ambiguous. Following the procedure
discussed in \todo[inline]{section ref} the resulting phase difference estimate
is -29.2\si{\degree} by combining the candidate direction -17.6\si{\degree}.
-30.6\si{\degree} and -39.3\si{\degree} from channel 1, 2 and 3 respectively
(\cref{tab:04_phaseDiffSingle}).
% -------------------------------------------------------------
\btline{ht}{1.2}
\btab{|c|c|c|}
\hline
Channel & Phase [\si{\deg}] & Amplitude\\
\hline
0 & -1.55 & 0.00144\\
\hline
1 & -177.7 & 0.00287\\
\hline
2 & 173.4 & 0.00279\\
\hline
3 & -75.0 & 0.00372\\
\hline
\etab
\et{Phase and amplitude of frame signals with $f_c$ = 2756.25Hz}{04_phaseSingle}
% -------------------------------------------------------------
\btline{ht}{1.2}
\btab{|c|c|c|c|c|}
\hline
Base Channel & Next Channel & Phase Difference & Candidate (-) & Candidate (+)\\
& & [\si{\deg}] & [\si{\deg}] & [\si{\deg}] \\
\hline
1 & 3 & -102.7 & -17.6 & 78.8\\
\hline
2 & 0 & 173.4 & -30.6 & -30.6\\
\hline
3 & 2 & 113.1 & -140.7 & -39.3\\
\hline
\etab
\et{Phase differences and resulting direction candidates of example data with phase method}{04_phaseDiffSingle}
% -------------------------------------------------------------

\subsection{Single Robot Angle Error}
\label{subsec:04_singleRobotAngleError}
\todo[inline]{This seems a bit misplaced (e.g. jumping back to angle errors of
individual robots)}

To identify the difference between the methods, a closer look
into the results of the single robots prior to the team filter is given.
\Cref{fig:04_compareRmse} presents the \ac{RMSE} considering the
direction results of all five robots for each measurement in
\cref{subsec:04_labMeasurements}.
Additionally, the estimated standard deviation\todo{Is it standard deviation or
variance here?} of each measurement
provides insight into the validity of the single robot results.
As one can see, the standard deviation of the relative angle of the \ac{GCC}
method is significantly smaller as compared to the phase method for most
measurements.
How this influences the reliability of the sound localization is
subject of discussion in \cref{sec:05_methodComparison}.
% -------------------------------------------------------------
\begin{figure}[ht]
	\centering
		\includegraphics[]{figures/evaluation/compare_rmse}
	\caption{Angular RMSE and standard error of robot results per
	measurement of \cref{subsec:04_labMeasurements}.}
    \label{fig:04_compareRmse}
\end{figure}
% -------------------------------------------------------------

\subsection{Conclusion}
\label{subsec:04_tdoaConclusion}

\ac{GCC-PHAT} is best. \ac{PSNR} information that is only available for this method turned out
to be very useful .