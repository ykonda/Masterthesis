\section{Single Robot Method Comparison}
\label{sec:04_tdoaSingle}
\todo[inline]{Klingt etwas verkettet. Vielleicht geht sowas wie: "Direction Prediction on a Single Robot"?}
\todo[inline]{Hast du fuer die "single robot method" noch mehr als diese eine Messung? Ich finde es sehr gut, das an Hand einer einzelnen Messung zu zeigen. Fuer eine quantiative Aussage waere es aber auch gut, noch messungen aus anderen Richtungen zu haben, wenn du die hast (nur Tabelle, nicht noch mal diskutieren)}

For comparability of the results, one exemplary measurement is utilized
to present and analyse the \ac{TDOA} methods.
In this data set, the sound source is placed at the right front
of the robot with 4.5\si{m} distance.
This corresponds to an an angle of -33.7\si{\degree} in robot coordinates.
Some\todo{some klingt etwas informell} samples around the signal start of the received signal data
is plotted in \cref{fig:04_tdoaSignal} for all channels.
% -------------------------------------------------------------
\begin{figure}[ht]
	\centering
		\includegraphics[]{figures/evaluation/cc_frontRight_1_signal}
	\caption{Signal start section of a whistle sound recorded from front right.}
	\label{fig:04_tdoaSignal}
\end{figure}
% -------------------------------------------------------------

As the next sections focus on the performance of the \ac{TDOA} methods,
the start index is set manually.
A frame size is defined with 256 samples and is selected around the start index for
the correlation methods as explained in \cref{sec:03_tdoa}.\todo{widerspricht sich mit dem satz davor (?) manuell vs automatisch}
For the phase method, the first frame with a size of 64 samples
is chosen where a whistle is detected in all channels.

For the sake of conciseness, throughout the following sections the correlation
function $R_{x_ax_b}$ of two signals $x_a$ and $x_b$  (\cf
\cref{chap:02_prerequisites}) is denoted as $R_{ab}$.


\subsection{Cross Correlation}
\label{subsec:04_ccSingle}
% -------------------------------------------------------------

To visualize the result of the \ac{CC}, the correlations are plotted in
\cref{fig:04_cc}. For $R_{23}$\todo{In der figure steht "R32" statt "R23"} and
$R_{13}$ a peak is clearly visible.
However, for the other \ac{CC} the problem of a low maximum peak
arises as mentioned in \cref{sec:02_cc}.
% -------------------------------------------------------------
\begin{figure}[ht]
	\centering
		\includegraphics[]{figures/evaluation/cc_frontRight_1}
	\caption{Cross correlation results of signal from front right (-33,7\si{\degree}).}
	\label{fig:04_cc}
\end{figure}
% -------------------------------------------------------------
\btline{ht}{1.2}
\btab{|c|c|c|c|c|}
\hline
Base Channel & Next Channel & Delay & Candidate (-) & Candidate (+)\\
\hline
0 & 1 & -8.25 & -144.9 & -35.1\\
\hline
1 & 3 & -4.59 & -17.4 & 78.6\\
\hline
2 & 0 & 9.16 & -30.6 & -30.6\\
\hline
3 & 2 & 3.94 & -150.2 & -29.8\\
\hline
\etab
\et{Cross correlation delay results of signal from front right}{04_cc}
% -------------------------------------------------------------

According to the delays in \cref{tab:04_cc}, the final result of the \ac{CC}
is -26.9\si{\degree}\todo{wo kommt diese Zahl her? Finde ich die in der
Tabelle?}. Hence, the resulting prediction error is 6.8\si{\degree}.
The delay between channel 2 and 0 is larger than the maximal delay of 6.85 samples
and therefore cut to the maximum sample delay.
Besides these, the \ac{TDOA} between the channel pairs produce one appropriate
direction candidate which correctly points to the sound source.
% -------------------------------------------------------------

\subsection{Generalized Cross Correlation}
\label{subsec:04_gccSingle}
% -------------------------------------------------------------
\Cref{fig:04_gcc} presents the cross correlation result by the \ac{GCC} method of
the same signal data as in the previous section.
The subsample delays for each channel pair and their resulting direction candidates
are listed in \cref{tab:04_gcc}.
From this, a final direction of -30.0\si{\degree} is determined
resulting in an error of 3,69\si{\degree}.
It is apparent that the peaks of the \ac{GCC} are better to detect than the peaks of the
\ac{CC}.
% -------------------------------------------------------------
\begin{figure}[ht]
	\centering
		\includegraphics[]{figures/evaluation/gcc_frontRight}
	\caption{Generalized cross correlation results of signal from front right.}
	\label{fig:04_gcc}
\end{figure}
% -------------------------------------------------------------
\btline{ht}{1.2}
\btab{|c|c|c|c|c|}
\hline
Base Channel & Next Channel & Delay & Candidate (-) & Candidate (+)\\
\hline
0 & 1 & -8,28 & -144,7 & -35,3\\
\hline
1 & 3 & -4,09 & -22,8 & 84,0\\
\hline
2 & 0 & 7,60 & -30.6 & -30.6\\
\hline
3 & 2 & 4,13 & -148,7 & -31,3\\
\hline
\etab
\et{Generalized cross correlation delay results of signal from front right}{04_gcc}
% -------------------------------------------------------------
\subsection{Phase Difference}
\label{subsec:04_phaseSingle}

For detecting the source direction with phase difference, a smaller frame
size of 64 samples is defined.
In \cref{subsec:03_phase} two variants of this method were introduced that use
different strategies to identify a reference frequency. The first version uses
a static reference frequency that is fixed a-priori by the user. The second version
dynamically estimates a dominant frequency across all four channels. Hereafter,
the performance of variants is discussed.

\todo[inline]{Reihenfolge anpassen}

\subsubsection*{Dynamic Reference Frequency Selection}

First, the result of the dynamically selected frequency is presented.
As stated in the implementation chapter, the frame is chosen where the
frequencies of the maximal amplitudes coincides for all channel which is
at 2756,25\si{\hertz}.
In the upper plot of \cref{fig:04_phaseSingle} one sees the received microphone data
which will be Hann windowed and then transformed into frequency domain by \ac{FFT}.
The resulting phases and amplitudes are listed in \cref{tab:04_phaseSingle}.
For comprehensibility, the determined frequency information visualized by
wave signals with the detected phases and amplitudes
in the lower subplot of \cref{fig:04_phaseSingle}.
Due to the larger distance between channels 0 and 1, the phase difference
information must be neglected because the phase difference is ambiguous.
Outcome from the applied phase differences is -29,2\si{\degree} by the combination of
-17,6\si{\degree}, -30,6\si{\degree} and -39,3\si{\degree}.
% -------------------------------------------------------------
\btline{ht}{1.2}
\btab{|c|c|c|}
\hline
Channel & Phase [\si{\deg}] & Amplitude\\
\hline
0 & -1,55 & 0,00144\\
\hline
1 & -177,7 & 0,00287\\
\hline
2 & 173,4 & 0,00279\\
\hline
3 & -75,0 & 0,00372\\
\hline
\etab
\et{Phase and amplitude of frame signals with $f_c$ = 2756,25Hz}{04_phaseSingle}
% -------------------------------------------------------------
\btline{ht}{1.2}
\btab{|c|c|c|c|c|}
\hline
Base Channel & Next Channel & Phase Difference & Candidate (-) & Candidate (+)\\
& & [\si{\deg}] & [\si{\deg}] & [\si{\deg}] \\
\hline
1 & 3 & -102,7 & -17,6 & 78,8\\
\hline
2 & 0 & 173,4 & -30,6 & -30,6\\
\hline
3 & 2 & 113,1 & -140,7 & -39,3\\
\hline
\etab
\et{Phase differences and resulting direction candidates of example data with phase method}{04_phaseDiffSingle}
% -------------------------------------------------------------
\begin{figure}[ht]
	\centering
		\includegraphics[]{figures/evaluation/phase_cos}
	\caption{Frames used for the direction detection by phase method.}
	\label{fig:04_phaseSingle}
\end{figure}
% -------------------------------------------------------------

\subsubsection*{Static Reference Frequency Selection}

Secondly, the frequency to examine $f_c$ is set to the first represented frequency
larger than 2600\si{\hertz} which is 2670,1\si{\hertz} with a \ac{FFT} length
of 256.
At this frequency, outcome of the direction candidates listed in \cref{tab:04_fixedFreqResult}
is -29,6\si{\degree} which results in an error of 4,1\si{\degree}.
% -------------------------------------------------------------
\btline{ht}{1.2}
\btab{|c|c|c|c|c|}
\hline
Base Channel & Next Channel & Phase Difference & Candidate (-) & Candidate (+)\\
& & [\si{\deg}] & [\si{\deg}] & [\si{\deg}] \\
\hline
1 & 3 & -79,1 & -26,8 & 88,0\\
\hline
2 & 0 & 167,7 & -30,6 & -30,6\\
\hline
3 & 2 & 88,5 & -148,7 & -31,3\\
\hline
\etab
\et{Resulting candidates of phase difference method with fixed frequency
	2670,1Hz of example measurement from front right
	(-33,7\si{\degree})}{04_fixedFreqResult}
% -------------------------------------------------------------

% Configuration for Phase Method
\subsubsection*{Static Reference Frequency Value}
\label{subsubsec:04_fixedFrequencyVal}

In order to observe the influence of the chosen frequency $f_c$
the phase method is tested with different frequencies in the whistle range.
To have a more generalized conclusion, more measurements are taken for this
investigation.
From eleven positions, all recordings of the whistle sounds are done
with one robot arranged at the center point.
This data corresponds to the measurements of the robot with number 26
in \cref{subsec:04_labMeasurements}.

As shown in \cref{fig:04_diffFc} shows, the \ac{RMSE} is high for
frequencies smaller than 2600\si{\hertz}.
With a frequency of 2024.12\si{\hertz}, error is largest.
The result complies with the information in \cref{fig:03_maxFreq} showing that
in a whistle signal frequencies higher than 2500\si{\hertz} are dominant.
With this outcome, the fixed frequency is set to 2670,1\si{\hertz}
for further usage of the direction detection by phase method.
Limitation exists due to the ambiguity of the signal which is
content of \cref{subsubsec:03_phase}.
% -------------------------------------------------------------
\begin{figure}[ht]
	\centering
		\includegraphics[]{figures/evaluation/phase_fc_rmse}
	\caption{Result of all measurements done with robot 26 to compare different
	fixed frequency values in whistle range.}
	\label{fig:04_diffFc}
\end{figure}
% -------------------------------------------------------------
\subsubsection*{Frame Number}
\label{subsubsec:04_frameNumber}

Not only does the frequency play a major role for the phase method,
but also the chosen frame.
To evaluate if and how the result changes over time, the frame to
utilize is shifted by half the frame size for all measurements
of \cref{subsec:04_labMeasurements} for the robot at the center
point.
% [ 29.08029745  58.1426789   65.27548831  67.91984464  76.82890558
%   97.33976251  96.63736674 100.2203543  105.14752253  61.28179302
%   60.0017076   50.60063622  53.80158271  49.0830362   62.64453085]
In \cref{fig:04_phaseOverTime} one sees that the first channel frames
with zero shift gives the best result with a \ac{RMSE} of 29,1\si{\degree}.
It validates that not any frame can be chosen for the
direction detection what again confirms that the signal start detection
plays a large role for the correct result.
% -------------------------------------------------------------
\begin{figure}[ht]
	\centering
		\includegraphics[]{figures/evaluation/phase_over_time}
	\caption{Changing results by shifting the frame over the
	samples. All measurements of \cref{subsec:04_labMeasurements}
	with robot nr. 26 are taken into consideration.}
	\label{fig:04_phaseOverTime}
\end{figure}
% -------------------------------------------------------------
