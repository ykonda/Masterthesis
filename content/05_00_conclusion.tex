\chapter{Conclusion}
\label{chap:05_conclusion}

The objective of this work was to find an approach to localize a whistle-sound
source with multiple NAO robots. \todo{why should we want this? One more sentence for motivation}
Each of these robots offers four microphones attached on their head.
In the scope of this work, stationary sound sources and motionless robots were
considered.

In order to solve this task, the \acf{TDOA} method was selected as the solution method
which is popular in the field of acoustic signal source localization. \missing[]{citations}
The fundamental concept of \ac{TDOA} algorithms is to obtain direction information
about acoustic signals by observing the time delay between separated microphone sensors.
Three different approaches were evaluated to obtain a stable method
for the delay estimation.
The \acf{CC} and \acf{GCC-PHAT} approaches are based on cross-correlation theory in terms
of signal processing.
In comparison, the phase difference method performs spectral analysis on the signal
to attain \ac{TDOA} information.

On a whole\todo{"on a whole" sounds weird. The sentence works by just starting it with "The process of...}, the process of the \acf{SSL} has been divided into three parts.
First, different signal start \todo{insert "detection"?}algorithms were examined with regard to accuracy.
By this, appropriate subsignals were selected for the \ac{TDOA} process.
Of capital importance is the \acf{WSDE} executed on stand-alone robots by
using one of the \ac{TDOA} methods.
Finally, a multi-agent decision is executed on a Bayesian updating filter
depending on the \ac{WSDE} results of the single robots.

\todo{It feels weird to have the overview almost half way into the conclusion. However, I understand why you want to emphasize the to previous paragraphs. Maybe put all of them into one? That would probably work best.}The beginning of this work covers the theoretical principles of each component
used for the multi-agent \ac{SSL}.
Based on this, implementation details are presented which are followed by analysis
and evaluation of real measurement data.
According to the results, the \acf{GCC-PHAT} algorithm performs best by
providing most accurate results for the \ac{WSDE} which offers the most precise
whistle-sound source position estimations in consequence.
For the measurement data recorded within this work, an average
distance error between estimated position and real source location
of less than 1\si{\meter} could be achieved.
\missing[]{More validation?}

With the outcome of this work, further research topics arise to solve
more complex \ac{SSL} challenges.
Using the ability to localize signal sources that are not whistles,
differentiation of multiple sounds and separation of those become important.
\missing[]{cite}
Furthermore, consideration of moving sources and robots can be interpreted
as the next essential step for the usage in real case scenarios.
Especially when adopting the \ac{SSL} for verification of other robots' information
like position and orientation, advancement towards these issues are unavoidable\todo{"anticipated", "unavoidable" sounds very negative or ironic.}.
In another perspective, the multi-agent behavior can be improved with various
concepts.
This includes processes like multimodal filtering or the exclusion of outliers
with filtering.
Also, more a-priori \todo{check entire document for " a priori" without "-", I think I saw that somewhere ->around Fig. 3.9} knowledge like the prior position of the sound source can be included.

One sees that wide-ranging progress \todo{...progress and novel applications}can be expected regarding \ac{SSL}
in the field of \todo{maybe insert "humanoid" ?} robotics.

\todo{overall you can probably add a little more detail in the summary of what you did. You can aim for 1.5 pages }



% One obtains the \ac{TDOA} by cross-correlating signal samples with standard
% \acf{CC} and \acf{GCC-PHAT} algorithms in frequency domain.
% The other analyses recorded samples in frequency domain in terms of
% a phase of a reference frequency which is described as phase difference method.