\chapter{Conclusion}
\label{chap:05_conclusion}

\begin{comment}
    - Objective of this work:
        x a whistle sound locator (in xy-plane) with multiple robots
        x stationary robots and sources
        x evaluated different tdoa methods
        - found a approach to detect direction of source on single robots
\end{comment}

The purpose of this work was to find an approach to localize a whistle
signal with a multi-agent system consisting of multiple NAO robots.
Each of these robots have four microphones attached on their head which
record audio signals within a range of 150\si{\hertz} and 12\si{\kilo\hertz}
with a sample rate of 44.1\si{\kilo\hertz}.
Only stationary sound sources and robots were considered in the scope of this
work.
Roughly, the implementation can be divided into three parts.
Of capital importance is the \acf{WSDE} executed by stand-alone robots which
is realized by computing the \ac{TDOA} between the four channels on the robots' head.
Two fundamentally different approaches were evaluated to obtain a stable method
for the delay estimation.
One obtains the \ac{TDOA} by cross-correlating signal samples with standard
\acf{CC} and \acf{GCC-PHAT} algorithms in frequency domain.
The other analyses recorded samples in frequency domain in terms of
a phase of a reference frequency which is described as \acf{PD} method.
% Correct
It was assumed that the start of a signal is most reliable 
To circumvent impact of multi-path propagation and reverberation, 
was examined and is true
Thus, \acf{SSD} approaches were compared with regard to accuracy and computational effort.
Finally, 

\section{Start Detection}
\label{sec:05_startDetection}

If the start of a whistle sound is wanted and computational effort is
neglected, the evaluation shows that the start detection with the
existing whistle detection algorithm performs best.
that a rough start detection by the exiting whistle detection
with a large frame size followed by a \ac{ZCR} detection with small
window size.
This is a good trade-off with regard to the computational effort.
\change[inline]{change for more noisy signals? -> then maybe entropy?}
% - entropy for large scale start detection\\
% - zcr and energy for smaller scale (because high precision needed)\\

\section{TDOA Methods}
\label{sec:05_methodComparison}

% Phase method: Much faster!

% GCC is more accurate in regard to each robot result -> This is
% important when less robots are used for localization.
% It can be the case when robots are broken / got penalty
% We want each robot result to be as accurate as possible so
% that we can count on each result
% Less robots means, that each wrong direction is weighted more
% and influences the final position significantly.
\section{Team Filter}
\label{sec:05_teamFilter}

% Team filter: more intelligent filtering (e.g. clustering
% intersections for multiple position candidates)


% Future work ___________________________________
% Separation of multiple sounds
% Moving sources, moving robots

% conclusion -> multimodales filtering, expizite ausreißerfilter, include prior knowledge
% of refree position