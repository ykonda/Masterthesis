\chapter{Summary and Conclusion}
\label{chap:05_conclusion}

The objective of this work was to find an approach to localize a whistle-sound
source with multiple NAO robots.
Each of these robots offers four microphones attached on their head.
In the scope of this work, stationary sound sources and motionless robots were
considered.
This issue arises in the context of \ac{RoboCup} \ac{SPL} which is a
competition where humanoid robots play soccer autonomously.
In this games, whistle-sounds are used to initiate kickoffs at the current state of the rules.
Due to parallel games at neighboring fields, other games' whistles can be heard and
the risk exists that the robots begin to move illegally.
To circumvent those situations by neglecting whistle-sounds from other
games, a \acf{SSL} algorithm was to be designed and implemented.

The beginning of this work covers the theoretical principles
used for the multi-agent \ac{SSL} which has been divided into three major parts.
Based on this, the implementation details of each of this components are proposed
and the different strategies are evaluated.

First, different signal start detection algorithms were examined with regard to accuracy.
Using this algorithm, appropriate subsignals were selected for the following processes with the
assumption that the beginning of a whistle signal is most unaffected by
reverberation and multi-path propagation.
Evaluation of the recorded data confirmed this assumption.
As accuracy and low computational effort are important on a real-time constraint
system, different approaches were compared to detect the signal start.
Best performance could be obtained by considering the \ac{ZCR}.
% The performance can be further increased by combining it with the entropy method.
% Advantage of both methods is that they do not require knowledge about the
% observed signal's main frequency.

Having appropriate samples of the whistle signal, the relative direction of a whistle source
is computed by each robot of the multi-agent system.
In order to solve this task, the \acf{TDOA} method was selected as solution method
which is popular a approach for acoustic signal source localization.
The fundamental concept of \ac{TDOA} algorithms is to obtain direction information
about acoustic signals by observing the time delay between separate microphones.
Three different approaches were evaluated to achieve a stable algorithm for the delay estimation.
The \acf{CC} and \acf{GCC-PHAT} approaches are based on cross-correlation theory in terms
of signal processing.
In comparison, the phase difference method performs spectral analysis on the signal
to attain \ac{TDOA} information.
This \acf{WSDE} is of significant importance which is why a wide range of examination
was done with real measurement data.
According to the results, the \acf{GCC-PHAT} algorithm performs best by
providing most accurate results for the \ac{WSDE} which offers the most precise
whistle-sound source position estimations in consequence.
Evaluating \ac{WSDE} measurements recorded by the multi-agent system of five robots,
the angular \ac{RMSE} by all robots was 11.84\si{\degree}.

Finally, a multi-agent decision is executed on a Bayesian updating filter
depending on the \ac{WSDE} results of the single robots.
For the measurement data recorded within this work, an average
distance error between estimated position and real source location
of less than 1\si{\meter} could be achieved.

With the outcome of this work, further research topics arise to solve
more complex \ac{SSL} challenges.
Using the ability to localize signals other than whistles,
% this approach may be used to distinguish and attribute sources of
% multiple sounds.
differentiation of multiple sounds and separation of their positions becomes possible.
Furthermore, consideration of moving sources and robots can be interpreted
as the next essential step for the usage in real case scenarios.
Especially when adopting the \ac{SSL} for verification of other robots' information
like position and orientation, advancement towards these issues are
important.
In another perspective, the multi-agent behavior can be improved with various
concepts.
This includes processes like multi-modal filtering or the detection of outliers.
Also, more a-priori knowledge like the prior position of the sound source can be included.

It can be concluded that wide-ranging progress and novel applications can be expected
regarding \ac{SSL} in the field of autonomously operating robotics.

% One obtains the \ac{TDOA} by cross-correlating signal samples with standard
% \acf{CC} and \acf{GCC-PHAT} algorithms in frequency domain.
% The other analyses recorded samples in frequency domain in terms of
% a phase of a reference frequency which is described as phase difference method.
