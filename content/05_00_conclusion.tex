\chapter{Conclusion}
\label{chap:05_conclusion}

The objective of this work was to find an approach to localize a whistle-sound
source with multiple NAO robots.
Each of these robots offers four microphones attached on their head.
In the scope of this work, stationary sound sources and motionless robots were
considered.
This issue originates in the context of \ac{RoboCup} \ac{SPL} which is a
platform where humanoid robots play soccer autonomously.
In this games, whistle-sounds are used to initiate kickoffs at the current state of the rules.
Due to parallel games at neighboring fields, other games' whistles can be heard and
the risk exists that the robots begin to move illegally.
To circumvent those situations by neglecting whistle-sounds from other
games, a \acf{SSL} algorithm was quested.

The beginning of this work covers the theoretical principles of each component
used for the multi-agent \ac{SSL} which has been divided into three major parts.
First, different signal start algorithms were examined with regard to accuracy.
By this, appropriate subsignals were selected for the following processes with the
assumption that the beginning of a whistle signal is most unaffected by
reverberation and multi-path propagation.
Evaluation of the recorded data confirmed this assumption.
As accuracy and low computational effort are important on a real-time capable
system, different approaches to detect the signal start were compared.
Best results could be obtained by a simple \ac{ZCR} approach.
However, in combination with the entropy signal start detection
most reliable results can be expected without the dependency on a
particular signal frequency.

Having appropriate samples of the whistle signal, the direction of a whistle-source
is computed by each robot of the multi-agent system.
In order to solve this task, the \acf{TDOA} method was selected as solution method
which is popular with regard to acoustic signal source localization.
The fundamental concept of \ac{TDOA} algorithms is to obtain direction information
about acoustic signals by observing the time delay between separated microphone sensors.
Three different approaches were evaluated to achieve a stable algorithm for the delay estimation.
The \acf{CC} and \acf{GCC-PHAT} approaches are based on cross-correlation theory in terms
of signal processing.
In comparison, the phase difference method performs spectral analysis on the signal
to attain \ac{TDOA} information.
This \acf{WSDE} is of capital importance which is why a wide range of examination
was done with real measurement data.
According to the results, the \acf{GCC-PHAT} algorithm performs best by
providing most accurate results for the \ac{WSDE} which offers the most precise
whistle-sound source position estimations in consequence.
Evaluating \ac{WSDE} measurements recorded by the multi-agent system of five robots,
the angular \ac{RMSE} by all robots was 11.84\si{\degree}.

Finally, a multi-agent decision is executed on a Bayesian updating filter
depending on the \ac{WSDE} results of the single robots.
For the measurement data recorded within this work, an average
distance error between estimated position and real source location
of less than 1\si{\meter} could be achieved.

With the outcome of this work, further research topics arise to solve
more complex \ac{SSL} challenges.
Using the ability to localize signal sources that are not whistles,
differentiation of multiple sounds and separation of those become important.
Furthermore, consideration of moving sources and robots can be interpreted
as the next essential step for the usage in real case scenarios.
Especially when adopting the \ac{SSL} for verification of other robots' information
like position and orientation, advancement towards these issues are anticipated.
In another perspective, the multi-agent behavior can be improved with various
concepts.
This includes processes like multi-modal filtering or the exclusion of outliers
with filtering.
Also, more a-priori knowledge like the prior position of the sound source can be included.

One sees that wide-ranging progress and novel applications can be expected regarding \ac{SSL}
in the field of autonomously operating robotics.

% One obtains the \ac{TDOA} by cross-correlating signal samples with standard
% \acf{CC} and \acf{GCC-PHAT} algorithms in frequency domain.
% The other analyses recorded samples in frequency domain in terms of
% a phase of a reference frequency which is described as phase difference method.