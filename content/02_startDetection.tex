\section{Signal Start Detection}
\label{sec:02_signalStartDetection}

One focus of the whistle signal localization is the correct choice of the
signal frame, with which the \ac{TDOA} measurement is done.
In order to perform the \ac{FFT} most efficiently, the size of one frame
should be a power of 2.
Assuming that the clearest signal without reverberation and with minimal
multipath propagated subsignals is at the start of a sound signal,
the frame to investigate is chosen to be at the beginning of a whistle sound.
Several methods exist and can be combined at will.
In the next subsections, signal start detection using short time energy,
zero crossing rate and spectral entropy are subject of discussion.

\subsection{Short Time Energy and Zero Crossing Rate}

A common method in signal start and endpoint detection is the evaluation
of the \ac{STE} and \ac{ZCR}.

\subsubsection{Short Time Energy}
The energy
\bal
    E(n) = \sum_{i=1}^N |X(n)|^2
    \label{eq:02_energy}
\eal
of signal frames with length $N$ are expected to be higher than noise frames
and therefore, noise and signal can be distinguished
according to \cite{Z_W_voiceActivity}.
A threshold needs to be specified according to the environment.

\subsubsection{Zero Crossing Rate}

The \ac{ZCR} of one frame $Z$ needs small computational effort in order to
identify a periodic signal in time domain.
Its formula is
\bal
    Z = \sum_{k=2}^N |sgn(x(k)) - sgn(x(k-1))|
    \label{eq:02_zcr}
\eal
with the sign function
\begin{align*}
    sign(x) =
    \begin{cases}
        1 & \quad \text{if } x\geq 0 \\
        -1 & \quad \text{if } x < 0
    \end{cases}
\end{align*}
for a discrete signal $x(k)$ of a frame with length $N$ \cite{Z_W_voiceActivity}.
A higher \ac{ZCR} is an indication for a sinusoidal signal.
To detect the signal start, a threshold is determined dynamically.
The \ac{ZCR} mean of frames which are known to be noise only
are averaged with the mean of those frames, that include the whistle signal.
The signal start is detected at the point in time, where the \ac{ZCR} exceeds
this threshold.

\subsection{Spectral Entropy}

Entropy provides information about the disorder of a system.
From this, one can derive that noise has a high entropy compared to
a whistle sound, which is a highly structured sound signal and a high
amount of information accordingly.
The spectral entropy of a signal is determined by normalizing the
\ac{PDF} over all frequency components as described in \cite{S_J_entropy}.
When $X(n)$ is the \ac{DFT} of the sampled signal $x(k)$, the \ac{PDF} is
\bal
    P(n) &= \frac{E(n)}{\sum_{i=1}^N E(n_i)}
    \label{eq:02_pdf}\\
\eal
with \ref{eq:02_energy} as the energy function and $N$ as the frame length.
Finally, the spectral entropy results in
\bal
    H &= - \sum_{n=1}^N P(n) log_2P(n).
\eal
\unsure[]{How should the energy be written? Discrete time correct?}
Utilizing some a priori knowledge about the signal, the entropy estimation
can be improved.
In this work, the frequency of a whistle sound is known to be
between $2000Hz$ and $4000Hz$ from \cite{Hasselbring},
Thus, only the frequency components in the whistle range is considered.
\unsure[]{weighting factors as described in S-J-entropy?}
Differentiating between noise samples where no signal is present and
signal frames, a dynamic threshold can be set to detect the signal start point.