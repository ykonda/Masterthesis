\section{Signal Start Detection}
\label{sec:02_signalStartDetection}

One focus of the whistle signal localization is the correct choice of the
signal frame, with which the \ac{TDOA} calculation is done.
Assuming that the clearest signal without reverberation and with minimal
multipath propagated subsignals is at the start of a sound signal,
the frame to examine is chosen to be at the beginning of a whistle sound.

By knowing the frequency band of a whistle signal, the start can be detected
where these frequencies dominate.
Using this indicator only does not always give the desired accuracy, that is why
different methods are examined in this work.
In the next subsections, signal start detection using short time energy,
zero crossing rate and spectral entropy are subject of discussion.
Also, the methods require
various computational power. According to the circumstances, the most suitable
approach can be chosen.
Another point is, that robustness can be increased by considering these methods
in combination.
As a latter, the consensus of the single methods can be passed as information
about the certainty of the computed direction result.

\subsection{Short Time Energy and Zero Crossing Rate}

A common method in signal start and endpoint detection is the evaluation
of the \ac{STE} and \ac{ZCR}.

\subsubsection{Short Time Energy}
The energy
\bal
    E &= \sum_{n=1}^N E_s(n)
    \label{eq:02_energy}
 \intertext{with the energy spectral density}
    E_s(n) &= |x(n)|^2
    \label{eq:02_spectralEnergy}
\eal
of signal frames with length $N$ are expected to be higher than noise frames
and therefore, noise and signal can be distinguished
according to \cite{Z_W_voiceActivity}.
A threshold needs to be specified appropriately dependent on the environment.

\subsubsection{Zero Crossing Rate}

The \ac{ZCR} of one frame $Z$ needs small computational effort in order to
identify a periodic signal in time domain.
Its formula is
\bal
    Z = \sum_{n=2}^N |sgn(x(n)) - sgn(x(n-1))|
    \label{eq:02_zcr}
\eal
with the sign function
\begin{align*}
    sign(x) =
    \begin{cases}
        1 & \quad \text{if } x\geq 0 \\
        -1 & \quad \text{if } x < 0
    \end{cases}
\end{align*}
for a discrete signal $x(k)$ of a frame with length $N$ \cite{Z_W_voiceActivity}.
A higher \ac{ZCR} is an indication for a periodic signal.
To detect the signal start, a threshold is determined dynamically.
The \ac{ZCR} mean of frames which are known to be noise only
are averaged with the mean of those frames, that include the whistle signal.
The signal start is detected at the point in time, where the \ac{ZCR} exceeds
this threshold.

\subsection{Spectral Entropy}
\label{subsec:02_Entropy}

Entropy provides information about the disorder of a system.
From this, one can derive that noise has a high entropy compared to
a whistle sound, which is a highly structured sound signal and a high
amount of information accordingly.
The spectral entropy of a signal is determined by normalizing the
\ac{PDF} over all frequency components as described in \cite{S_J_entropy}.
When $X(n)$ is the \ac{DFT} of the sampled signal $x(n)$, the \ac{PDF} is
\bal
    P(n) &= \frac{E_s(n)}{E}
    \label{eq:02_pdf}
\eal
with \cref{eq:02_spectralEnergy} as the spectral energy density function for $E_s(n)$
and $E$ as the energy.
Finally, the spectral entropy results in
\bal
    H &= - \sum_{n=1}^N P(n) log_2P(n).
    \label{eq:02_entropy}
\eal
Utilizing some a-priori knowledge about the signal, the entropy estimation
can be improved.
In this work, the frequency of a whistle sound is known to be
between $2\si{\kilo\hertz}$ and $4\si{\kilo\hertz}$ from \cite{Hasselbring},
Thus, only the frequency components in the whistle range is considered.
% \unsure[]{weighting factors as described in S-J-entropy?}
Differentiating between noise samples where no signal is present and
signal frames, a dynamic threshold can be set to detect the signal start point.