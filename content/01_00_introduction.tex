\chapter{Introduction}
\label{chap:01_introduction}

% \todo[inline]{
% Margins needs to be reduced back in layout.tex to
% "oddsidemargin 2.7cm", "evensidemargin 2.6cm"
% and "usepackage[paperwidth=275.9mm, paperheight=279.4mm]{geometry}"
% in Thesis.tex needs to be removed.}

% - why other theories do not fit into this problem (like finger printing, beam forming)
% -> TDOA between robots -> one does not want to send signal samples
%    via message. (Too large information)
%   -> Approach of Dortmund -> too inaccurate time point when whistle is detected. NTP too
%      inaccurate, offline synchronization can be done or PTP implementation
% Another approach would be the TDOA between the robots. Therefore, the robots needs to
% be synchronized in time. The whistle detection time point can then be taken for calculating
% the time difference of arrival. The distance between the robots needs to be known
% Robots needs to be synchronized (which is difficult with NTP only)

\acf{SSL} is content of research for many years with
application scenarios in various environments due to its
validity for all sorts of signals (sonar, radar, acoustic, seismology,
geophysics, ultrasonics, communications \cite{Chen2006}) 
Explicitly for audio signals, use cases emerge with the progressing
ordinariness of technical equipment on daily basis.
% As technical devices gained in importance more and more,
Assuming robots as forthcoming everyday object, reaction to acoustic
input is one essential step for natural human-robot interaction.
For example, one plausible scenario is a humanoid robot keeping the eye
contact with an interacting human.
An even more tangible case can be seen in conference rooms of business environments
where remote participation is commonplace in these days.
Functions of communication systems like speaker identification and tracking of active talkers
get crucially important to provide smooth operation \cite{Brandstein96apractical}.
In general, \ac{SSL} algorithms can be divided into three categories
which are based on beamforming, eigenvalue decomposition or \ac{TDOA} \cite{Brandstein96apractical}.
% intuitive handling with acoustic input
% Especially, many use cases exist regarding non-acoustic signals
% like sonar, radar,
% investigated by researchers a broad topic which is 
% In this field of signal processing, signal source localization is 

This thesis investigates different implementations of the \ac{TDOA} algorithm for \acp{SSL}
focussing on whistle sounds in the context of the \ac{SPL} \textit{RoboCup}.






\section{Motivation}

The \ac{SPL} team \textit{HULKs} is a student association of the \ac{TUHH}.
