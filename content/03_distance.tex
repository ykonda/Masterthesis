\section{Front and Rear Distance}
\label{sec:03_distance}

\change[]{Image! Coordinate is not correct for set coordinate system}

For simplicity, the direction of the sound source is determined in
horizontal plane only.
However, the front and rear microphones differ 0.0212\si{m} in height
what can provide rough information about the distance of the sound source
for the particular case when the signal comes straight from front or behind.
If $delay_{01}$ and $delay_{32}$ are both very small, the signal comes from
the front or the back most likely.
Figure ...\missing[]{reference to nao channel figure} evinces, that one of
the sidewise delays need to be larger or equal to 5.41 samples
\change[]{number to variable?} for a signal to arrive from sideways.
When the delays of the side channels are smaller, they can be used to
detect the angle of the sound source in vertical YZ plane.

\Cref{fig:02_headSideTdoa} is a lateral illustration of
the NAO's right microphones $channel_1$ and $channel_3$.
% -------------------------------------------------------------
\begin{figure}[ht]
	\centering
		\includegraphics[width=0.45\columnwidth]{figures/side_head_tdoa}
	\caption{Illustration of arriving sound for sources from near behind.}
    \label{fig:02_headSideTdoa}
\end{figure}
% -------------------------------------------------------------
With the delay $D$, the angle of the sound source $\gamma$ relative
to the Z-axis can be determined with
% -------------------------------------------------------------
\bsub \bal
\gamma &= \alpha + \gamma'\\
\gamma' &= sign(D) \cdot sin^{-1}(\frac{D}{D_{max}})\\
\intertext{with}
\alpha &= tan^{-1}(\frac{\Delta z_{channel}}{\Delta x_{channel}})
\eal \esub
which is the angle to the orthogonal axis to the plane though
front and rear channels.
% -------------------------------------------------------------
\begin{figure}[ht]
	\centering
		\includegraphics[width=0.6\columnwidth]{figures/x_distance}
	\caption{}
    \label{fig:02_xDistance}
\end{figure}
% -------------------------------------------------------------
Known that the sound source is positioned above of the robot, the distance
of the sound source can be approximated with an assumed height $\Delta h_{source}$
of the source.
So, the distance to the sound source is
% -------------------------------------------------------------
\bal
\Delta x &= (\Delta h_{source} - \Delta h_{NAO}) \cdot \tan(\gamma).
\label{eq:02_deltaX}
\eal
% -------------------------------------------------------------
If the front and rear direction candidates dissolve each other to a small
value, it is assumed that the signal source is on the X-axis of the robot.
The side delays are then used as indicator for the distance to the robot
%  as
% explained in \cref{sec:02_distance} and 
\change[]{removed cref}
need to be smaller than
\change[]{change name?} $samples_{xz}$.
To estimate the distance, the height of the sound source needs to be set as
constant \change[]{change name?} \lstinline!height_source!.\\
Restrictions of the front and rear distance measurement differ.
For the front case, the maximal angle for a unambiguous distance calculation
is $\frac{\pi}{2}- 2\alpha$.
Thus, the maximal front distance that can be approximated shrinks to
$\Delta x = (\Delta h_{source} - \Delta h_{NAO}) \cdot \tan(\frac{\pi}{2} - 2\alpha)$
according to \cref{eq:02_deltaX}.
To the rear, the maximal value for $\gamma$ is bounded by \change[]{naming?}
\lstinline!xz_delay_limit!.
Setting $\Delta h_{source}$ to 1,5\si{m} and $\Delta h_{NAO}$ to 0,57\si{m},
the maximal measurable distance to the front is about 0.66\si{m}.
With the same values and 5.3 samples as \lstinline!xz_delay_limit!, the
minimum $\Delta x$ value is 33.29\si{m}.
\missing[]{removed code}
% \code{distance}{0}{35}{Pseudo code of distance approximation on X-axis.}{03_distance}
\unsure[]{I am not really sure how pseudo codes should look like and if one is necessary anyway?}
