\section{Front and Rear Distance}
\label{sec:03_distance}

If the front and rear direction candidates dissolve each other to a small
value, it is assumed that the signal source is on the X-axis of the robot.
The side delays are then used as indicator for the distance to the robot as
explained in \ref{sec:02_distance} and need to be smaller than
\change[]{change name?} $samples_{xz}$.
To estimate the distance, the height of the sound source needs to be set as
constant \change[]{change name?} \lstinline!height_source!.\\
Restrictions of the front and rear distance measurement differ.
For the front case, the maximal angle for a unambiguous distance calculation
is $\frac{\pi}{2}- 2\alpha$.
Thus, the maximal front distance that can be approximated shrinks to
$\Delta x = (\Delta h_{source} - \Delta h_{Nao}) \cdot \tan(\frac{\pi}{2} - 2\alpha)$
according to \ref{eq:02_deltaX}.
To the rear, the maximal value for $\gamma$ is bounded by \change[]{naming?}
\lstinline!xz_delay_limit!.

