\section{Front and Rear Distance}
\label{sec:03_distance}

If the front and rear direction candidates dissolve each other to a small
value, it is assumed that the signal source is on the X-axis of the robot.
The side delays are then used as indicator for the distance to the robot as
explained in \cref{sec:02_distance} and need to be smaller than
\change[]{change name?} $samples_{xz}$.
To estimate the distance, the height of the sound source needs to be set as
constant \change[]{change name?} \lstinline!height_source!.\\
Restrictions of the front and rear distance measurement differ.
For the front case, the maximal angle for a unambiguous distance calculation
is $\frac{\pi}{2}- 2\alpha$.
Thus, the maximal front distance that can be approximated shrinks to
$\Delta x = (\Delta h_{source} - \Delta h_{Nao}) \cdot \tan(\frac{\pi}{2} - 2\alpha)$
according to \cref{eq:02_deltaX}.
To the rear, the maximal value for $\gamma$ is bounded by \change[]{naming?}
\lstinline!xz_delay_limit!.
Setting $\Delta h_{source}$ to 1.5\si{m} and $\Delta h_{Nao}$ to 0.57\si{m},
the maximal measurable distance to the front is about 0.66\si{m}.
With the same values and 5.3 samples as \lstinline!xz_delay_limit!, the
minimum $\Delta x$ value is 33.29\si{m}.
\code{distance}{0}{35}{Pseudo code of distance approximation on X-axis.}{03_distance}
\unsure[]{I am not really sure how pseudo codes should look like and if one is necessary anyway?}
