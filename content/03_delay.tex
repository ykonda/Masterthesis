\section{Delay Estimation}
\label{sec:03_delay}

To ascertain the direction of the sound source on one robot, the
delay of the signal from each channel to it's neighboring
next channel is the key information.
The implementation of the \ac{CC} from \cref{sec:02_cc} and the \ac{GCC-PHAT}
introduced in \cref{sec:02_gcc}, as well as the phase difference method
of \cref{sec:02_phase} will be presented shortly.

For all methods performed in frequency domain, the \ac{FFT} of the
signal is required. Hence the library NumPy, known for it's high-level
mathematical functions also provides a module with fundamental
functions for the \ac{DFT}, \lstinline!numpy.fft.fft()! and
\lstinline!numpy.fft.ifft()! can be utilized.
To estimate the delay between \lstinline!signal_0! and \lstinline!signal_1!,
both frames of the signals were Hann windowed in advance.
% -------------------------------------------------------------
\subsection*{Correlation}
\label{subsec:03_cc}

In Python, using the \ac{DFT} functions given by NumPy the \ac{CC} and \ac{GCC} calculation
can be implemented with little lines of code.
Function \lstinline!correlation! of \cref{lst:03_cc} itemizes the steps to accomplish
a correlation in Python with input signals \lstinline!x0! and \lstinline!x1!.
Determining the index of the maximal value of the correlation relative to the zero delay
index delivers the delay of \lstinline!x1! compared to \lstinline!x1! as the function
\lstinline!compute_delay! demonstrates.
% -------------------------------------------------------------
\code{delay}{1}{18}{Source code for delay calculation with correlation method.}{03_cc}
% -------------------------------------------------------------
\subsection*{Subsample Delay}
\label{subsec:03_subsample}

Integer delays only offers resulting direction angles with low resolution.
To avoid this, the subsample shift estimation as in \cref{sec:02_subsampleShift}
is added to the delay estimation.
% -------------------------------------------------------------
\code{delay}{20}{21}{Subsample shift source code.}{03_subsample}
% -------------------------------------------------------------
In \cref{lst:03_subsample} the maximum value and it's previous and subsequent values
are buffered in \lstinline!r!. The calculation is then alike to \cref{eq:02_subsample}
and added to the integer delay $D$.

\subsection*{Phase Difference}
\label{subsec:03_phase}

Periodicity is a known problem and is bypassed with following conditions:\\
- don't consider channel 0 and 1\\
- only frequencies above xxx\\
- assume that channel with higher amplitude was first\\
This helps, because known that phase difference can not be larger than $\pi/2$

\missing[]{content}