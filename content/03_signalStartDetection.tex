\section{Signal Start Detection}
\label{sec:03_signalStartDetection}

As mentioned in \cref{sec:02_signalStartDetection}, the detection of the
signal start is crucial for the localization.
The implementation of the different methods will be presented coupled with
an examination of real measurement data.
\\
To reduce undesirable effects and demonstrate the simplest form, a sinusoidal
signal of $3000\si{Hz}$ was recorded with a sample rate of $44100\si{Hz}$.
For the following data, the sound source was placed $2\si{m}$ in front of the robot.
\\
In order to find the time point where the signal starts, information about
smaller fractions are required.
So, the original $44100$ samples that were buffered by the
\lstinline!WhistleLocalization! module are divided into several overlapping
frames with size $256$. The computational effort raises with smaller frame size,
but delivers a higher precision in return.
In order to perform the \ac{FFT} most efficiently, the size of one frame
should be a power of 2.
To compute the energy and entropy, the frames are transformed into
frequency domain with the \ac{FFT}.
The \ac{ZCR} does not require such a transformation.
In the evaluation section \cref{sec:04_signalStartDetection}, the result of the single
methods are compared to each other.
For better visualization, the following data is shortened to $2400$ samples.

\missing[]{Start detection by frequency}

\subsubsection*{Spectral Entropy}

The formula to calculate the spectral entropy of a signal is introduced as \cref{eq:02_entropy}
in the previous chapter.
For the entropy information, the signal must not be cleaned previously.
The first frames are known as noise only, as well as the last frames are assumed
to contain signal and noise.
By comparing the mean of the received noise at the beginning of the measurement and
the mean of the signal part, a dynamic threshold can be defined and the beginning of the
signal is detectable.
In this work, the crossing of the entropy threshold is observed from the last
value to the first. The reason is, that the noise entropy subceeds the threshold with higher risk.
Of course, this approach does not work for real-time systems.
Figure \cref{fig:03_entropy} is a plot of the recorded sine signal with the corresponding
entropy.
According to the frame size, the accuracy of the start index can be increased.
However, the frame size is limited by the required number of samples for one \ac{FFT} and
its computational effort.

\begin{figure}[ht]
	\centering
		\includegraphics[]{figures/sine_entropy}
	\caption{Entropy of a sinusoidal signal with 3000Hz.}
\end{figure}
\label{fig:03_entropy}

\subsubsection*{Energy}

As mentioned above, the total signal is divided into multiple overlapping frames.
Equation \cref{eq:02_spectralEnergy} represents the energy of each frequency
component.
According to this, the energy of one of those frames is \cref{eq:02_energy}.
Assuming that the energy holds for the whole frame, overlapping and adding the energy
results in \cref{fig:03_energy}.
If the frequency of the examined signal is known as for the whistle, only energy values
of the relevant frequencies needs to be considered.
One downside of the energy information is that the threshold has to be adapted
manually for the related environment.
Especially at the tournament, parameters like these should be avoided.

\begin{figure}[ht]
	\centering
		\includegraphics[]{figures/sine_energy}
	\caption{Energy of a sinusoidal signal with 3000Hz.}
\end{figure}
\label{fig:03_energy}

% \code{signal_processing}{9}{19}{Whistle energy}
% Variable \lstinline!calcFFt!

\subsubsection*{Zero Crossing Rate}

To receive a higher accuracy, the frames of the \ac{ZCR} are set to 80 samples.
In each frame, the the sign changes are counted.
It is known that only noise is received at the beginning of the measurement and on the
contrary, signal noise is present at the end.
The same as for the spectral entropy, a dynamic threshold can be set.
In figure \cref{fig:03_zcr} and in this work , the threshold is set between the
noise and signal mean.
The implementation of the \ac{ZCR} in Python is very simple.
\code{zcr}{1}{2}{Calculation of Zero Crossing Rate in Python}{03_zcr}

\begin{figure}[ht]
	\centering
		\includegraphics[]{figures/sine_zcr}
	\caption{Zero Crossing Rate of a sinusoidal signal with 3000Hz.}
\end{figure}
\label{fig:03_zcr}