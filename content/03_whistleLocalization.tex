\section{Whistle Source Direction Estimation}
\label{subsec:03_directionEstimation}

% started with \ac{WSDE}
By one stand-alone system in form of a robot with four microphones
attached on the head, only the direction of a sound source can be
estimated.
As shortly introduced in \cref{subsec:03_whistleLocalizationStructure},
the \lstinline!WhistleDirectionEstimation! module is responsible to
determine the direction of the whistle sound detected by the \lstinline!WhistleDetection!.
Before going into detail, the overall structure of the \ac{WSDE} on a single robot
is summarized by means of \cref{fig:04_stateMachine}.
% -------------------------------------------------------------

\begin{figure}[ht]
	\centering
		\includegraphics[height=0.7\textheight]{figures/state_machine}
	\caption{Concept of whistle localization on single robot.}
	\label{fig:04_stateMachine}
\end{figure}
% -------------------------------------------------------------

\lstinline!WhistleDirectionEstimation! depends on the data type \lstinline!WhistleData!
which is a production of \lstinline!WhistleDetection!.
Currently, it contains two values. A timestamp when the whistle was last detected
and a boolean value if a whistle start is found.
Until this value is set to true by the \lstinline!WhistleDetection!, the
\lstinline!WhistleDirectionEstimation! buffers up to
44100 audio samples per channel.
As a whistle needs to be detected for multiple cycles in a short time, one can
be sure that the buffer contains a number of whistle samples.

Out of the buffered samples, the signal start is calculated by the \ac{SSD}
which will be further described in \cref{sec:02_signalStartDetection}.

According to the \ac{TDOA} method, frames considered for the delay estimation
are chosen differently.
Detailed descriptions are given in the corresponding method
\cref{subsubsec:03_cc,subsubsec:03_phase}.
By \ac{TDOA} the delays between the two microphone channels are
determined by either \ac{GCC-PHAT}, \ac{CC} or phase difference method.
Each delay generates two potential source direction candidates as stated
in \cref{sec:02_tdoa}.
In the event of candidates indicating a nearby signal from straight forward or
backwards, the distance to the source is estimated in addition.
More precise explanation is given in \cref{subsec:03_distance}.
For a final direction, the mean of the candidates with smallest difference
is formed.

If one of the correlation methods are used, the \ac{TDOA} is calculated
multiple times shifting the frame over samples.
The range in which the frame is shifted around the start index is parameterized
as well as the size of the shift.
By saving the calculated \ac{TDOA} values and looking up certain conditions per frame shift,
the most promising result can be chosen.
This way, potential start estimation inaccuracies can be corrected and
decision process is optimized.
In case of the \ac{GCC-PHAT}, the \ac{PSNR} provides information about
the certainty of the \ac{TDOA} estimation as \cref{subsec:04_psnr} proves.
A frame shift delivers \ac{TDOA} values for each channel pair and a
mean of all \acp{PSNR} from the \ac{GCC} functions.
Comparing the frame shifts, the \ac{TDOA} results with
greatest \ac{PSNR} mean value is assumed as best performing.
The same procedure is done with the \ac{CC} method but by examining
the greatest \ac{CC} function value.

Regardless of the method, the production of this module is a
\lstinline!WhistleDirection! data type which contains calculated
direction outcome in radians and additional information like distance or \ac{PSNR}.

All transformations into frequency domain and inverse transformations
are executed with the \ac{FFTW} library just as the \lstinline!WhistleDetection!.
For parallel development in Python, the widespread package \textit{NumPy} delivers
all fundamental functions for computation in frequency domain.

% -------------------------------------------------------------

\section{Signal Start Detection}
\label{sec:03_signalStartDetection}

As mentioned in \ref{sec:02_signalStartDetection}, the detection of the
signal start is crucial for the localization.
The implementation of the different methods will be presented coupled with
an examination of real measurement data. By knowing the frequency band
of a whistle signal, the start can be detected where these frequencies dominate.
Using this indicator only does not always give the desired accuracy, that is why
different methods are investigated in this work. Also, the methods require
unequal computational power. According to the circumstances, the most suitable
approach can be chosen.
Another point is, that the robustness can be increased by considering these methods
in combination.\change[]{Move this into theory part?}
As a latter, the consensus of the single methods can be passed as information
about the certainty of the computed direction result.
\\
To reduce undesirable effects and demonstrate the simplest form, a sinusoidal
signal of $3000\si{Hz}$ with a sample rate of $44100\si{Hz}$. For this
data, the sound source was placed $2\si{m}$ in front of the robot.
\\
In order to find the time point where the signal starts, information about
smaller fractions are required.
So, the original $44100$ samples that were buffered by the
\change[]{correct font and wording}
"WhistleLocalization" module are divided into several overlapping
frames with size $256$. The computational effort raises with smaller frame size,
but delivers a higher precision in return.
To compute the energy and entropy, the frames are transformed into
frequency domain with the \ac{FFT}.
The \ac{ZCR} does not require such a transformation.
\missing[]{final start index by combination of these 3 methods}
For better visualization, the following data is shortened to $2400$ samples.


\subsubsection*{Spectral Entropy}

The formula to calculate the spectral entropy of a signal is \ref{eq:02_entropy}.
- use non-cleaned signal because of entropy information
- first frames are known as noise only
- mean of noise signal can be set as threshold
- look from back until threshold is exceeded. This is signal start index

% Variable \lstinline!calcFFt!

\begin{figure}[ht]
	\centering
		\includegraphics[]{figures/sine_entropy}
	\caption{Entropy of a sinusoidal signal with 3000Hz.}
\end{figure}
\label{fig:03_entropy}

\subsubsection*{Energy}

\ref{eq:02_energy} results in the energy of each frequency component of each frame.
According to this, the energy of one frame is \ref{eq:02_energy}
As the frames of the whole signal are overlapping, the energy function plotted in
\ref{fig:03_energy} results from overlapping and adding the frame energies.\\
- with a priori knowledge: only look for energy between 2000Hz and 4000Hz
One downside of the energy information is that the threshold can not be
set dynamically. It has to be adapted manually for the related environment.

\begin{figure}[ht]
	\centering
		\includegraphics[]{figures/sine_energy}
	\caption{Energy of a sinusoidal signal with 3000Hz.}
\end{figure}
\label{fig:03_energy}

\code{signal_processing}{7}{19}{Whistle energy}

\subsubsection*{Zero Crossing Rate}

- count the sign changes in frame
- calculate the noise mean at the beginning of signal
- calculate signal mean at end of signal
- mean of both is set as threshold
- start index is detected at that point where zcr is higher than this threshold

\begin{figure}[ht]
	\centering
		\includegraphics[]{figures/sine_zcr}
	\caption{Zero Crossing Rate of a sinusoidal signal with 3000Hz.}
\end{figure}
\label{fig:03_zcr}
\subsection{Time Difference of Arrival}
\label{subsec:03_tdoa}

The \ac{TDOA} estimation is the main component to identify the
whistle source location.
Theoretical background to this approach of source localization is
given in \cref{sec:02_tdoa}.
As stated there, the \ac{TDOA} of a signal measured between two microphone sensors
provides details about the direction of the source.
Having four channels attached on a NAO's head, an overdetermined system it
given where each channel pair provides \ac{TDOA} information.

The \ac{GCC-PHAT} method is a modification of the \ac{CC} method.
Due to their implementation being equal except of the weighting function,
they are discussed in \cref{subsubsec:03_cc} collectively.
\Cref{subsubsec:03_phase} presents the implementation details with it's
circumstances.
% -------------------------------------------------------------

\subsubsection{Correlation}
\label{subsubsec:03_cc}

In theory, \acf{CC} in time domain is usually illustrated by shifting
two signals about each other and recording the similarity for each shift.
Thus, a peak will arise at that shift where signals are most similar.
Imaging two equal signals, one can image a peak at the middle of the \ac{CC} function
\lstinline!R!.
The index in this case is called \lstinline!zeroIndex! and calculated with
\lstinline!int(length(R))-1!.
If one signal is alike the other but delayed by some samples, the peak will
occur at so many samples next to the \lstinline!zeroIndex!.
This delay which is directly related to the \ac{TDOA} is computed
by the \ac{CC} and \ac{GCC} in the unit of samples.
As \cref{sec:02_cc,sec:02_gcc} have shown, the \ac{GCC} is commonly performed
in frequency domain.
For unification, both \ac{CC} and \ac{GCC} are implemented in frequency domain.
% Hereinafter, \ac{CC} and \ac{GCC} will be summarized as \textit{correlation} in this section.

The samples for the \ac{CC} are defined by the start index and the frame shift
according to \cref{subsec:03_directionEstimation} and originate from the data which was
cleaned by spectral subtraction previously.
The frame size in this work is set to 256 samples typically and the samples are Hann-windowed
prior to the correlation.
By zero padding the \ac{FFT} resolution can be increased, but is refrained from for now.
For two real signals, the \ac{CC} can be realized by time-reversing one of the signals before
the \ac{FFT} and then multiplying each component.
In the event of \ac{GCC-PHAT}, each component of the multiplication is divided by the absolute
value as the weighting function \cref{eq:02_gccPhat} defines.
After this, the cross-correlated signal is transformed back into time domain and
index of the peak \lstinline!peakIndex! is found.
The delay in samples is then computed by \lstinline!peakIndex - zeroIndex!.
In conformity with the definitions in thesis, a positive delay \lstinline!d_01!
between \lstinline!x_0! (signal at channel 0) and \lstinline!x_1! (signal at channel 1)
indicates that the signal was at channel 0 first.
% -------------------------------------------------------------

\subsubsection*{Subsample Delay}
\label{subsubsec:03_subsample}

Integer delays only offers resulting direction angles with low resolution.
To avoid this, the subsample shift estimation as in \cref{sec:02_subsampleShift}
is added to the delay estimation for both \ac{CC} and \ac{GCC}.

\subsubsection{Phase Difference}
\label{subsubsec:03_phase}

% Frame size of 64 samples -> better result
% periodicity -> maximal possible frame with given distance

There are two ways to determine the angle by phase difference
between two channels.
One can either look at the phase of a fixed frequency or set the frequency
dynamically.
For both, the signal must be divided into multiple frames.
The implementation for the fixed frequency case is straight forward.
From the result of the signal start detection, the frames
are set with an appropriate frame size and then transformed into
frequency domain.
As known from \cite{Hasselbring}, the frequency of the whistle is between 2\si{\kilo\hertz}
and 4\si{\kilo\hertz}.
Thus, a suitable frequency in this range is chosen for analysis.
In the other case, the frame is chosen by doing a
frequency analysis.
For each channel frame in frequency domain after signal start,
the maximal absolute value and its belonging frequency is determined.
If this frequency is equal for all channels, these frames
are chosen.
\Cref{fig:03_maxFreq} shows that such frames exists for signals that
were transformed into frequency domain with a frame size of 256.
The frequency resolution changes with zero padding the signal prior
to the transformation.
\change[]{Finalize!}
% -------------------------------------------------------------
\begin{figure}[ht]
	\centering
		\includegraphics[]{figures/maxFreq}
	\caption{}
    \label{fig:03_maxFreq}
\end{figure}
% -------------------------------------------------------------

If the \ac{TDOA} is determined by phase difference, it must be ensured
that the maximal difference between two channels must not overflow $\pi$.
Meeting this condition, the signed difference is ascertainable.
As \cref{tab:03_maxFrequncies} presents, the distance between channel 0 and 1
is too large because its maximal frequency is not in whistle frequency range.
For this reason, the phase difference information between this pair is neglected.
% -------------------------------------------------------------
\btline{ht}{1.2}
\btab{|c|c|c|}
\hline
Channel Pairs & Absolute Distance [\si{\meter}] & Max. Frequency [\si{\hertz}]\\
\hline
0 and 1 & 0,116 & 1536,75\\
\hline
1 and 3 & 0,0533 & 3217,11\\
\hline
2 and 0 & 0,0533 & 3217,11\\
\hline
2 and 3 & 0.0618 & 2775,08\\
\hline
\etab
\et{Maximal feasible frequencies for unambiguous phase difference detection}{03_maxFrequncies}
% -------------------------------------------------------------

To facilitate the implementation, the phase difference is easily convertible into
delay samples $D_s$ with
\bal
	D_s = \frac{f_s \cdot \Delta \psi}{2 \pi \cdot f_c}.
\eal
\subsection{Direction Estimation}
\label{subsec:03_directionCandidates}

Delay samples are computed by the \ac{TDOA} methods introduced in the last
sections.
Using \cref{eq:02_tdoaAngle}, one positive and one negative signed angle $\gamma'$
arise relative to the vector between the channels.
\Cref{fig:03_tdoaCode} is used to illustrate the terms and visualize the
circumstances for better understanding.
The definition of \textit{base channel} and \textit{next channel} stay as introduced
in \cref{subsec:03_microphones}.
Positive delay between two channels implies that a signal source
is closer to the base channel than the next channel.
For example if the detected delay has the same value as the maximum possible number
of samples between those channels, the source direction is equal to the
\textit{max-delay vector} in the figure.
In this case, $\gamma'$ is zero.
For all smaller delays greater than zero, two direction candidates \textit{candidate 0}
and \textit{candidate 1} result in the range of the max-delay vector $\pm \pi$.
The same applies mirrored for negative delays.
% -------------------------------------------------------------

\begin{figure}[ht]
	\centering
		\includegraphics[width=0.6\columnwidth]{figures/tdoa_code}
	\caption{Illustration of the resulting candidates of \ac{TDOA} implementation.}
	\label{fig:03_tdoaCode}
\end{figure}

% -------------------------------------------------------------

By this implementation, all candidates are represented in the robot coordinate system.
Having four channels, each neighboring channel pair returns two candidate directions.
Diagonal channels can be paired as well for the correlation methods.
However, this case is non-observed profoundly in this work due to the overdetermined
system by four pairs.
\missing[]{Extra: evaluation of diagonal \acp{GCC}?}
Out of these eight candidates, a final direction angle $\gamma$ is chosen
by computing all combinations and selecting the one with smallest sum of angle difference.
During research, different factors like signal strength were tested to include
more a priori knowledge.
Due to lack of reliability, no additional signal properties are taken into account.

There exists one exceptional case, where the signal source is detected straight in front
or behind the robot and distance can be estimated.
If this is the case, the direction angle is corrected to 0 or $\pi$.
How the distance information is handled is content of the next section.

\subsection{Front and Rear Distance}
\label{subsec:03_distance}

% \change[]{Image! Coordinate is not correct for set coordinate system}

% For simplicity, the direction of the sound source is determined in
% horizontal plane only.
% However, the front and rear microphones differ 0.0212\si{m} in height
% what can provide rough 
For the particular case when the signal comes straight from the front or behind,
the distance of the sound source can be estimated\todo{estimated und approx. haben die selbe Aussage}.
If $delay_{01}$ (delay between channel 0 and 1) and $delay_{32}$ (delay between channel 3 and 2)
are both very small, the signal comes from
the front or the back most likely. \todo{why "most likely" i think you can skip that and say it comes from front or back} \todo{comes most likely from the fround or rear.}
In theory, for this case the lateral delays are larger or equal the maximal
sample difference between the front and rear channels on the x-axis which is
to\todo{to?} 5.41 samples.
With smaller lateral delays and some assumptions \todo{the assumptions given below / or which assumptions are you talking about?}, the angle of the sound source in the XZ
plane can be estimated.
\Cref{fig:02_headSideTdoa} illustrates the NAO's head from the right with
channels 1 and 3.
% -------------------------------------------------------------
\begin{figure}[ht]
	\centering
		\includegraphics[width=0.45\columnwidth]{figures/side_head_tdoa}
    \caption{Illustration of arriving sound for sources from near behind.
             Adapted from \cite{nao_docu}.}
    \label{fig:02_headSideTdoa}
\end{figure}
% -------------------------------------------------------------

Assuming, that $delay_{01}$ and $delay_{32}$ are small,
the angle of the sound source $\gamma$ relative
to the Z-axis can be determined with delay $D$ as
% -------------------------------------------------------------
\bsub \bal
\gamma &= \alpha + \gamma'\\
\gamma' &= sign(D) \cdot sin^{-1}\left(\frac{D}{D_{max}}\right)\\
\intertext{whereby}
\alpha &= tan^{-1}\left(\frac{\Delta z_{channel}}{\Delta x_{channel}}\right) \approx 26.73\si{\degree}
\eal \esub
which is the angle to the orthogonal axis to \todo{zweimal to? muss da von einmal eins "from..." .."to.." sein? So ergibt sich mir die Aussage nicht } the plane though
front and rear channels.
$\Delta z_{channel}$ and $\Delta x_{channel}$ are the distances between the channels
in z- and x-direction in \si{\meter}.
% -------------------------------------------------------------
\begin{figure}[ht]
	\centering
		\includegraphics[width=0.6\columnwidth]{figures/x_distance}
	\caption{Illustration of distance estimation.}
    \label{fig:02_xDistance}
\end{figure}
% -------------------------------------------------------------

Knowing that the sound source is ordinarily positioned above of the robot, the distance
of the sound source can be approximated with an assumed height $\Delta h_{source}$
of the source. $\Delta h_{source}$ differs from referee \todo{waere das dann "actual" height? Wuerde ich dann auch einmal als delta(hactual) of the referee oder so bezeichnen} and thus, is only
an averaged value \todo{average of what? or approximate}.
So, the distance in x-direction $\Delta x$ is
% -------------------------------------------------------------
\bal
\Delta x &= (\Delta h_{source} - \Delta h_{NAO}) \cdot \tan(\gamma).
\label{eq:02_deltaX}
\eal
% -------------------------------------------------------------

The distance estimation is triggered, if the direction candidates of
$delay_{01}$ and $delay_{32}$ are smaller than $\pm$ 10\si{\degree}
or larger than $\pm$ 170\si{\degree}.
Additionally, the lateral delays must be smaller than 5.41 samples as mentioned above.

Restrictions of the front and rear distance measurement differ.
For the front case, the maximum angle for a unambiguous distance calculation
is $\frac{\pi}{2}- 2\alpha$.
Thus, the maximum front distance that can be approximated shrinks to
$\Delta x = (\Delta h_{source} - \Delta h_{NAO}) \cdot \tan(\frac{\pi}{2} - 2\alpha)$
according to \cref{eq:02_deltaX}.
To the rear, the maximal value for $\gamma$ is bounded by the 5.41 samples that
are set as condition.
Setting $\Delta h_{source}$ to 1.5\si{m} and $\Delta h_{NAO}$ to 0.57\si{\meter} for example,
the maximal measurable distance to the front is about 0.66\si{\meter}.
With the same values the maximal distance backwards is more than 50\si{\meter} in theory.
However, measurements in \cref{subsec:04_distance} show that 7\si{\meter} is the
limit for the real case. \todo{ich würde hier nochmal etwas klarer darstellen, woher der Unterschied zwischen front und rear kommt. Ist für dich vielleicht klar, für den Lesen, gerade beim ersten Mal aber nicht. Ein sollte da als Begründung reichen}

% -------------------------------------------------------------

\subsection{SNR}
\label{subsec:03_snr}

The \acf{SNR} is a common value to express the signal power $P_{signal}$ compared
to power of the background noise $P_{noise}$.
Conveniently, the buffered audio signal in this thesis always consists of
a clean-cut delimitation between signal and noise which is set by the
start index.
Thus, the \ac{SNR} which is defined as
\bal
    SNR_{db} = 10\log_{10}\left(\frac{P_{signal} - P_{noise}}{P_{noise}}\right)
    \label{eq:03_snr}
\eal
in decibels can be implemented straightforwardly.
Informational content about this measure is investigated in \cref{subsec:04_snr}.
Expectations are that the \ac{SNR} can be fed into the covariance matrix
of an incoming result in the Bayesian update process introduced in \cref{subsec:02_2dTeam}.

\subsection{PSNR}
\label{subsec:03_psnr}
In image processing, the \acf{PSNR} indicates the quality of a compressed
image. Here in this work, the ratio between the peak of a signal
and its noise is related to the \ac{GCC-PHAT} outcome and called \ac{PSNR}
henceforth.
As stated in \cref{sec:02_gcc}, the most significant characteristic
of the \ac{GCC-PHAT} is the resulting sharp peak which now can be
assessed with one value
\bal
    PSNR_{db} = 10\log_{10}\left(\frac{P_{peak}}{P_{noise}}\right).
    \label{eq:03_psnr}
\eal
From the implementation view, the power of the correlation peak is divided by
the power of the remaining correlation signal.
It has to be noted that two adjacent values prior and after the peak
are disregarded as they might belong to the peak.
A validation if and how much the \ac{PSNR} and the accuracy of the \ac{GCC}
delay result are linked is done in \cref{subsec:04_psnr}.
% -------------------------------------------------------------