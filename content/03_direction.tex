\section{Source Direction Estimation}
\label{sec:03_direction}

After the \lstinline!WhistleData! was filled by the \lstinline!WhistleDetection!,
the \lstinline!WhistleLocalization! start to process the buffered microphone data.
First, the frames to consider for the calculation of the \ac{TDOA} are chosen.
Different criteria apply for the correlation methods and phase method, but the
resulting \lstinline!start_index! of the signal start detection implementation
\cref{sec:03_signalStartDetection} is utilized in both.

For the correlation methods in \cref{sec:02_cc} and \cref{sec:02_gcc}, the beginning
of the signal is chosen where the change from noise to signal is visible preferably.
\lstinline!frame_size/2! samples before and after the \lstinline!start_index! are defined
as frame.
It gets undecidable which signal came first for later frames. Thus, if the start detection
is inaccurate the resulting \ac{TDOA} is not reliable.

Regarding the phase method,
%\cref{sec:02_phase}%,
\missing[]{reference to 02 phase}
the frames with the same maximal frequency for each channel are used for the
phase difference calculation.
The phase difference of this frequency is then considered for the phase difference
calculation. %\cref{sec:03_phase}
\missing[]{reference to 03 phase}

Before determining the delays, the frames are windowed with a Hann window and can be
normalized optionally.
According to the method delays between the channels and the consequential direction
candidates are computed.
For each delay between neighboring channels, one positive signed and one negative
signed candidate arise.
The relative angle $\gamma'$ to the vector from one channel to the other is defined
by \cref{eq:02_tdoaAngle}.
Depending on the sign of the delay, both candidates needs to be transformed
into robot coordinates.
In code \ref{lst:03_direction}, the 

\code{tdoa}{0}{10}{Calculation of whistle direction in robot coordinates.}{03_direction}

As there exists the exceptional case that the sidewise and forward delays are both small
as considered in \cref{sec:02_distance}, the invalidity of
\lstinline!small_y_shift && small_x_shift! must be confirmed.
If so, not only the direction but also an approximate distance can be determined
as depicted in \cref{sec:03_distance}.
Otherwise, the direction with the smallest difference between the channel
candidates is submitted as final result.
