\section{Time Difference Of Arrival}
\label{sec:02_tdoa}

The direction of a signal source $\gamma'$ can be determined by the time
delay of the received signal.
Calculations for the direction of the sound source can be done with a
geometrical approach like in \cite{Valin_Michaud}.
\Cref{fig:02_tdoa} illustrates how a the delay $D$ is introduced by the direction angle
of the sound source relative to a vector between channels 0 and 1.
Here, the signal arrives first at channel 0. Ideally, the same signal values
are measured at channel 1 with delay $D$.
If the delay is zero, the signal is perpendicular to the channels vector.
Its value can be $D_{max}$ maximally which in that case delivers the
result of the source direction vector being opposite to the \textit{channels vector}.
The channels vector is defined as vector directing from one channel to another which
in this case starts at channel 0 and points towards channel 1.
It is assumed that the distance from the sensors to the sound source is
significantly large so that the signal waves proceed parallel which is a necessary
criterion for the approach to be valid.
\begin{figure}[ht]
    \centering
    \subfigure[Illustration of TDOA principle where signal comes from front.]{
        \includegraphics[width=0.4\columnwidth]{figures/tdoa_waves}
        \label{fig:02_tdoaFront}
    }
    % \qquad
    \\
    \subfigure[Illustration of \ac{TDOA} principle for the second possible case
               with equal delay where signal comes from rear.]{
        \includegraphics[width=0.4\columnwidth]{figures/tdoa_waves_rear}
        \label{fig:02_tdoaRear}
    }
    \caption[Illustration of TDOA]{Picturing both possible sources of the sound when a delay D is
               measured between channels 0 and 1.}
    \label{fig:02_tdoa}
\end{figure}
% -------------------------------------------------------------

Specifying the speed of sound $c_s$ being 343\si{m/s} in air, the angle
$\gamma'$ can be defined as
\bsub \bal
    \gamma' &= cos^{-1}\left(\frac{D}{D_{max}}\right)
    \label{eq:02_tdoaAngle}\\
    \intertext{with}
    D_{max} &= \frac{f_s \cdot d_{01}}{c_s}
\eal \esub
where $f_s$ is the sampling rate and $d_{01}$ is the distance between both channels.
Not to forget is the ambiguity of the result by observing two channels only.
Having a look at \cref{fig:02_tdoaFront} once more and assuming that the sound source
is positioned in front in this case, the same delay can be the result of a
sound source from behind.
For quick understanding, one can find an illustration of the second possible
source directions in \cref{fig:02_tdoaRear}.

% -------------------------------------------------------------

With the definition of a whistle signal as stated in \cref{eq:02_whistleSignal},
the microphone sensors $channel_0$ and $channel_1$ will output
\bsub
\label{eq:02_signalTimeDomain}
\bal
    x_0(t) &= s(t) + n_0(t)\\
    x_1(t) &= \alpha s(t - D) + n_1(t).
\eal \esub
Again, $D$ is the delay of $x_1$ relative to $x_0$ for what is looked for.
As introduced in \cref{chap:01_introduction}, different methods to detect this delay
were implemented and evaluated in this work.
In the following sections, the theoretical background of these will be
explained in detail.