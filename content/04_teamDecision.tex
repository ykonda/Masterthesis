\newpage
\section{Team Evaluation}
\label{sec:04_teamEvaluation}

In the last chapters the performance of the three \ac{TDOA} methods
were validated on standalone robots independently.
Here, the focus is on the final whistle localization algorithm
as multi agent system.
With the results of the single operations as input,
the team whistle localization outputs an absolute sound source position.
To provide a decoupled result to the signal start detection,
the start indexes were set manually.

The size of the field used in this work is smaller than the regular \ac{SPL}
field with 7.5\si{m} length and 5\si{m} width.
Further information about the measurement setup is introduced in \cref{subsec:04_labMeasurements}.
% It is looked at the result of one exemplary measurement where the behavior
% of the team filter is of prime importance.

\subsection{Measurement Setup}
\label{subsec:04_labMeasurements}

11 measurement were taken with five robots on the small-sized \ac{SPL} field
of the HULKs laboratory.
\Cref{fig:04_setup} illustrates the positions of the NAO robots
and the positions of the whistle sound sources.
According to these. the x and y values of the sources and robots are
listed in \cref{tab:04_robots} and \cref{tab:04_sources}.
The orientation $\theta$ of the robots are defined relatively to the x-axis and
correspond to the definition in \cref{subsec:03_coordinates}.
In the following sections, the signal data from these measurements
will be used mainly.
% -------------------------------------------------------------
\btline{ht}{1.2}
\btab{|c|c|c|c|}
\hline
NAO & x [\si{m}] & y [\si{m}] & $\theta$ [\si{deg}]\\
\hline
21 & 3,75 & 2,5 & -40,2\\
\hline
24 & 3,75 & -2,5 & 90\\
\hline
26 & 0 & 0 & 0\\
\hline
27 & -3,75 & -2,5 & 66,06\\
\hline
28 & -2,45 & 0 & 0\\
\hline
\etab
\et{Positions of the robots for evaluation measurement}{04_robots}
% -------------------------------------------------------------
\btline{ht}{1.2}
\btab{|c|c|c|}
\hline
Measurement & x [\si{m}] & y [\si{m}]\\
\hline
[0] & 3,75 & 2,5\\
\hline
[1] & 3,75 & -2,5\\
\hline
[2] & -3,75 & -2,5\\
\hline
[3,9] & -3,75 & 2,5\\
\hline
[4] & -2,45 & 0\\
\hline
[5] & 2,45 & 0\\
\hline
[6,10] & 0 & 0\\
\hline
[7] & 0 & -2,5\\
\hline
[8] & -6.05 & 0\\
\hline
\etab
\et{Positions of the whistle sound sources for evaluation measurement}{04_sources}
% -------------------------------------------------------------
\begin{figure}[ht]
	\centering
		\includegraphics[]{figures/evaluation/setup}
	\caption{Setup of robots and sound source positions for the evaluation measurement.}
    \label{fig:04_setup}
\end{figure}
% -------------------------------------------------------------

\subsection{GCC Method}
\label{04_teamGcc}

To determine an overall result, each robot executes the sound source direction
detection with the \ac{GCC-PHAT} method standing alone.
After that, the results are input into the team decision filter as specified
in \cref{sec:03_teamDecision} which determines a final sound source position
result with Bayesian updating.

For further clarification, details of the result for measurement 1 of
\cref{subsec:04_labMeasurements} is presented extensively here.
\Cref{fig:04_gccResult} illustrates the result of the
relative direction angle outputs $\gamma$ of the individual robots
listed in \cref{tab:04_gccResult}.
Same as in \cref{fig:04_setup}, the yellow dots symbolizes the robots with
its orientation as short yellow line.
The resulting arrows represent the direction $\gamma$ where the sound source
is predicted.
In the graphic the real source is marked as star and the resulting
position as cross.
% -------------------------------------------------------------
\btline{ht}{1.2}
\btab{|c|c|c|c|}
\hline
NAO & $\gamma$ [\si{\deg}] & Abs. Error [\si{\deg}] & Mean PSNR \\
\hline
21 & -26,22 & 3,71 & 18,3\\
\hline
24 & -133,77 & 9,32 & 16,8\\
\hline
26 & -30,19 & 3,50 & 19,6\\
\hline
27 & -75,26 & 1,71 & 17,4\\
\hline
28 & -15,90 & 2,53 & 15,1\\
\hline
\etab
\et{Resulting directions of the single robots with \ac{GCC-PHAT} method for a
whistle sound signal in the right front corner of the playing field}{04_gccResult}
% -------------------------------------------------------------
\begin{figure}[ht]
	\centering
		\includegraphics[]{figures/evaluation/gcc_team}
	\caption{Team whistle localization result with \ac{GCC-PHAT}
	method.}
    \label{fig:04_gccResult}
\end{figure}
% -------------------------------------------------------------

The final result and its corresponding errors are listed \cref{tab:04_gccTeamResult}.
With an absolute distance error of less than 1\si{\meter} and small angular error,
the source localization works sufficiently for this case of application.
% -------------------------------------------------------------
\btline{ht}{1.2}
\btab{|c|c|c|}
\hline
 & Result & Error\\
\hline
Position x [\si{\meter}] & 3,38 & -0,37\\
\hline
Position y [\si{\meter}] & -1,85 & 0,65\\
\hline
Angle & 33,18\si{\degree} & 1,57\si{\degree}\\
\hline
Distance [\si{\meter}] & 3,85 & 0,74 \\
\hline
\etab
\et{Whistle localization result of measurement 1 with \ac{GCC-PHAT} method}{04_gccTeamResult}
% -------------------------------------------------------------

\Cref{tab:04_gccTeamResult} shows the distance and angle errors
for all measurements in \cref{subsec:04_labMeasurements}.
The \ac{RMSE} in distance being 0.87\si{\meter} and angular \ac{RMSE}
being 5,07\si{\degree} one can say that the \ac{GCC-PHAT} algorithm
works well for whistle sound source localization.
% -------------------------------------------------------------
\btline{ht}{1.2}
\btab{|c|c|c|c|c|}
\hline
Measurement & Error x [\si{\meter}] & Error y [\si{\meter}] & Abs. Distance Error [\si{\meter}] & Angle Error\\
\hline
[0] & 1,31 & 1,06 & 1,68 & 1,45\si{\degree}\\
\hline
[1] & 0,13 & 0,06 & 0,15 & 1,57\si{\degree}\\
\hline
[2] & 0,59 & 0,43 & 0,73 & 0,42\si{\degree}\\
\hline
[3] & 0,54 & 0,47 & 0,72 & 9,09\si{\degree}\\
\hline
[4] & 0,27 & 0,0 & 0,27 & 0,01\si{\degree}\\
\hline
[5] & 0,15 & 0,14 & 0,21 & 3,18\si{\degree}\\
\hline
[6] & 0,41 & -0,02 & 0,41 & 2,67\si{\degree}\\
\hline
[7] & 0,39 & 0,02 & 0,39 & 8,98\si{\degree}\\
\hline
[8] & 1,84 & -0,01 & 1,84 & 0,14\si{\degree}\\
\hline
[9] & 0,58 & 0,52 & 0,78 & 9,89\si{\degree}\\
\hline
[10] & 0,03 & -0,0 & 0,03 & 0,0\si{\degree}\\
\hline
\etab
\et{Whistle localization results for all measurements in \cref{subsec:04_labMeasurements} with
\ac{GCC-PHAT} method}{04_gccTeamResult}
% -------------------------------------------------------------
% - intersections
% - updates
% - covariance
% - PSNR

% -------------------------------------------------------------
\subsection{CC Method}
\label{04_teamCc}

Evaluation is done of the same measurements in \cref{subsec:04_labMeasurements}
with normal \ac{CC}.
The results show that the normal \ac{CC} without weighting performs
poorer compared to the \ac{GCC-PHAT} method with an \ac{RMSE} of
1,45\si{\meter} in distance and 10,67\si{\degree} angular.
Indeed, the statements in \cref{sec:02_cc} prove to be true.

% -------------------------------------------------------------
\btline{ht}{1.2}
\btab{|c|c|c|c|c|}
\hline
Measurement & Error x [\si{\meter}] & Error y [\si{\meter}] & Abs. Distance Error [\si{\meter}] & Angle Error\\
\hline
[0] & 0,6 & 1,39 & 1,51 & 8,15\si{\degree}\\
\hline
[1] & -0,49 & 1,2 & 1,3 & 11,97\si{\degree}\\
\hline
[2] & 2,32 & 2,11 & 3,13 & 18,28\si{\degree}\\
\hline
[3] & 1,07 & -0,96 & 1,44 & 3,71\si{\degree}\\
\hline
[4] & 1,95 & -0,09 & 1,95 & 10,44\si{\degree}\\
\hline
[5] & 0,07 & 0,01 & 0,07 & 0,32\si{\degree}\\
\hline
[6] & 0,4 & -0,01 & 0,4 & 1,91\si{\degree}\\
\hline
[7] & 1,06 & -0,0 & 1,06 & 22,99\si{\degree}\\
\hline
[8] & 1,24 & -0,06 & 1,24 & 0,77\si{\degree}\\
\hline
[9] & -0,04 & 0,78 & 0,78 & 7,22\si{\degree}\\
\hline
[10] & 0,03 & -0,0 & 0,03 & 0,0\si{\degree}\\
\hline
\etab
\et{Whistle localization results for all measurements in \cref{subsec:04_labMeasurements} with
\ac{CC} method}{04_ccTeamResult}

Further information about the origin of the error can be obtained
by looking at the single robot results of each measurement.
Because the team filter is updated by the intersections of the
single robot rays, the accuracy of the absolute position
depends on the number of arising intersections.
This will be discussed in \cref{subsec:04_singleRobotAngleError}.
% -------------------------------------------------------------
\subsection{Phase Method}
\label{04_teamPhase}

For the results by phase method, the minimal frequency value
parameter is set to 2700\si{\hertz}.
Just like for the other methods, the results of the
measurements are shown in \cref{tab:04_phaseTeamResult}.
The \ac{RMSE} according to the distance is in the same
order of magnitude as the \ac{CC} method with 1,33\si{\meter}.
Having a much larger \ac{RMSE} regarding the angle with 74,8\si{\degree},
especially these angular errors of measurements 6 and 10 catch one's eye.
Both measurements are taken at the center point of the field.
Because the absolute position is in an acceptable
error range, these both angular results
will be neglected from the error calculation.
Thus, the angular \ac{RMSE} of the phase method without
measurements 6 and 10 is 11,69\si{\degree}.

% -------------------------------------------------------------
\btline{ht}{1.2}
\btab{|c|c|c|c|c|}
\hline
Measurement & Error x [\si{\meter}] & Error y [\si{\meter}] & Abs. Distance Error [\si{\meter}] & Angle Error\\
\hline
[0] & -1,07 & -0,98 & 1,45 & 4,13\si{\degree}\\
\hline
[1] & 0,21 & 0,27 & 0,34 & 4,27\si{\degree}\\
\hline
[2] & 0,22 & 1,39 & 1,41 & 16,25\si{\degree}\\
\hline
[3] & 1,26 & -0,02 & 1,26 & 11,19\si{\degree}\\
\hline
[4] & 0,16 & -0,04 & 0,17 & 0,99\si{\degree}\\
\hline
[5] & -0,42 & 0,19 & 0,47 & 5,46\si{\degree}\\
\hline
[6] & -0,32 & 0,08 & 0,33 & 166,25\si{\degree}\\
\hline
[7] & -0,28 & 1,76 & 1,78 & 20,82\si{\degree}\\
\hline
[8] & 2,37 & 0,27 & 2,39 & 4,23\si{\degree}\\
\hline
[9] & 2,04 & 0,29 & 2,06 & 24,89\si{\degree}\\
\hline
[10] & -0,32 & 0,0 & 0,32 & 180,0\si{\degree}\\
\hline
\etab
\et{Whistle localization results for all measurements in \cref{subsec:04_labMeasurements} with
phase method}{04_phaseTeamResult}

\subsection{Single Robot Angle Error}
\label{subsec:04_singleRobotAngleError}

To identify the difference between the methods, a closer look
into the results of the single robots prior to the team filter is given.
\Cref{fig:04_compareRmse} presents the \ac{RMSE} considering the
direction results of all five robots for each measurement in
\cref{subsec:04_labMeasurements}.
Additionally, the standard deviation of each measurement allows insight
into the validity of the single robot results.
As one can see, the standard deviation in degrees of the \ac{GCC} method is
significantly smaller in comparison to the phase method for most
measurements.
How this influences the reliability of the sound localization is
subject of discussion in \cref{sec:05_methodComparison}.
% -------------------------------------------------------------
\begin{figure}[ht]
	\centering
		\includegraphics[]{figures/evaluation/compare_rmse}
	\caption{Angular RMSE and standard error of robot results per
	measurement of \cref{subsec:04_labMeasurements}.}
    \label{fig:04_compareRmse}
\end{figure}
% -------------------------------------------------------------

% Another point to take into account is the number of intersections
% yielded in the team filter from the direction rays.
% Compared to 
% more robots that fail -> less intersections
% show number of intersections for each file compared to method
% The rest of algorithm as presented in 03 phase